%Version control information
%$HeadURL: https://ejercicioscalculo.googlecode.com/svn/trunk/compendio_ejercicios_calculo.tex $} {$LastChangedDate: 2008-07-09 16:26:20 +0200 (mi�, 09 jul 2008) $
%$LastChangedRevision: 6 $
%$LastChangedBy: asalber $


\section{Derivada y Diferencial} 
%---------------------------------------------------------------------slide----
\begin{frame}
\frametitle{Derivada y diferencial}
\tableofcontents[sectionstyle=show/hide,hideothersubsections]
\end{frame}

\subsection{El concepto de derivada}

%---------------------------------------------------------------------slide----
\begin{frame}
\frametitle{Tasa de variación media}
\begin{definicion}[Incremento]
Dada una función $f$, se llama \emph{incremento} de $f$ en un intervalo $[a,b]$ a la diferencia entre el valor de $f$ en cada uno de los extremos del intervalo, y se nota
\[\Delta y= f(b)-f(a).\]
\end{definicion}

Cuando $f$ es la función identidad $y=x$, se cumple que
\[\Delta x=\Delta y= f(b)-f(a)=b-a,\]
y por tanto, el incremento de $x$ en un intervalo es la amplitud del intervalo. Esto nos permite escribir el intervalo $[a,b]$ como $[a,a+\Delta x]$.

\begin{definicion}[Tasa de variación media]
Se llama \emph{tasa de variación media} de $f$ en el intervalo $[a,a+\Delta x]$, al cociente entre el incremento de $y$ y el incremento de $x$ en dicho intervalo, y se escribe
\[
\textrm{TVM}\;f[a,a+\Delta x]=\frac{\Delta y}{\Delta x}=\frac{f(a+\Delta x)-f(a)}{\Delta x} 
\]
\end{definicion}
\end{frame}


%---------------------------------------------------------------------slide----
\begin{frame}
\frametitle{Tasa de variación media: Ejemplo}
Consideremos la función $y=x^2$ que mide el área de un cuadrado de chapa metálica de lado $x$.

Si en un determinado instante el lado del cuadrado es $a$, y sometemos la chapa a un proceso de calentamiento que aumenta el lado del cuadrado una cantidad $\Delta x$, ¿en cuánto se incrementará el área del cuadrado?
\begin{columns}
\begin{column}{0.3\textwidth}
\begin{align*}
\Delta y &= f(a+\Delta x)-f(a)=(a+\Delta x)^2-a^2=\\
&= a^2+2a\Delta x+\Delta x^2-a^2=2a\Delta x+\Delta x^2.
\end{align*}
\end{column}
\begin{column}{0.3\textwidth}
\begin{center}
\scalebox{1}{\psset{unit=0.6}
\begin{pspicture*}(0,-0.5)(4,4)
\scriptsize
\psframe(0,0)(3,3)
\rput[t](1.5,-0.1){$a$}
\rput[t](1.5,1.7){$a^2$}
\psframe[fillstyle=solid,fillcolor=orange](0,3)(3,4)
\psframe[fillstyle=solid,fillcolor=orange](3,3)(4,4)
\psframe[fillstyle=solid,fillcolor=orange](3,0)(4,3)
\rput[t](3.5,-0.1){$\Delta x$}
\rput[t](1.5,3.7){$a\Delta x$}
\rput[t](3.5,3.7){$\Delta x^2$}
\rput[t](3.5,1.7){$a\Delta x$}
\end{pspicture*}}
\end{center}
\end{column}
\end{columns}
¿Cuál será la tasa de variación media del área en el intervalo $[a,a+\Delta x]$?
\[
\textrm{TVM}\;f[a,a+\Delta x]=\frac{\Delta y}{\Delta x}=\frac{2a\Delta x+\Delta x^2}{\Delta x}=2a+\Delta x. 
\]
\end{frame}


%---------------------------------------------------------------------slide----
\begin{frame}
\frametitle{Interpretación geométrica de la tasa de variación media}
La tasa de variación media de $f$ en el intervalo $[a,a+\Delta x]$ es la pendiente de la recta \emph{secante} a $f$ en los puntos $(a,f(a))$ y $(a+\Delta x,f(a+\Delta x))$.
\begin{center}
\scalebox{1}{\psset{xunit=2,algebraic}
\begin{pspicture*}(-1,-0.5)(3,4)
\psaxes[ticksize=-3pt 0,labelsep=3pt,ticks=none]{<->}(0,0)(-0.3,-0.5)(2.5,3.8)[$x$,-90][$y$,180]
\psplot[linecolor=blue]{0.2}{1.8}{x^3-x^2+x+0.5}
\footnotesize
\rput[r](1.4,3.5){$f(x)$}
\psxTick[ticksize=-3pt 0,labelsep=3pt](0.5){a}
\psxTick[ticksize=-3pt 0,labelsep=3pt](1.5){a+\Delta x}
\psline[linewidth=0.5pt,linestyle=dashed,linecolor=gray](0.5,0)(0.5,0.875)
\psline[linewidth=0.5pt,linestyle=dashed,linecolor=gray](1.5,0.875)(0,0.875)
\psline[linewidth=0.5pt,linestyle=dashed,linecolor=gray](1.5,0)(1.5,3.125)(0,3.125)
\psyTick[ticksize=-3pt 0,labelsep=3pt](0.875){f(a)}
\psyTick[ticksize=-3pt 0,labelsep=3pt](3.125){f(a+\Delta x)}
\psplot[linecolor=red]{0.2}{1.8}{2.25*x-0.25}
\psline[arrows=|*-|*,linecolor=green](1.5,3.125)(1.5,0.875)
\psline[arrows=|*-|*,linecolor=green](0.5,0.875)(1.5,0.875)
\rput[l](1.6,2){$\Delta y=f(a+\Delta x)-f(a)$}
\rput[t](1,0.8){$\Delta x$}
\end{pspicture*}}
\end{center}
\end{frame}


%---------------------------------------------------------------------slide----
\begin{frame}
\frametitle{Tasa de variación instantánea}
En muchas ocasiones, es interesante estudiar la tasa de variación que experimenta una función, no en intervalo, sino en un punto.

Conocer la tendencia de variación de una función en un instante puede ayudarnos a predecir valores en instantes próximos.

\begin{definicion}[Tasa de variación instantánea y derivada]
Dada una función $f$, se llama \emph{tasa de variación instantánea} de $f$ en un punto $a$, al límite de la tasa de variación media de $f$ en el intervalo $[a,a+\Delta x]$, cuando $\Delta x$ tiende a 0, y lo notaremos
\[
\textrm{TVI}\;f (a)=\lim_{\Delta x\rightarrow 0} \textrm{TVM}\; f[a,a+\Delta x]=\lim_{\Delta x\rightarrow 0}\frac{\Delta y}{\Delta x}=\lim_{\Delta x\rightarrow 0}\frac{f(a+\Delta x)-f(a)}{\Delta x} 
\]
Cuando este límite existe, se dice que la función $f$ es derivable en el punto $a$, y al valor del mismo se le llama derivada de $f$ en $a$, y se nota como $f'(a)$.
\end{definicion}
\end{frame}


%---------------------------------------------------------------------slide----
\begin{frame}
\frametitle{Tasa de variación instantánea: Ejemplo}
Consideremos de nuevo la función $y=x^2$ que mide el área de un cuadrado de chapa metálica de lado $x$.

Si en un determinado instante el lado del cuadrado es $a$, y sometemos la chapa a un proceso de calentamiento que aumenta el lado del cuadrado, ¿cuál es la tasa de variación instantánea del área del cuadrado en dicho instante?
\begin{align*}
\textrm{TVI}\;f(a)]&=\lim_{\Delta x\rightarrow 0}\frac{\Delta y}{\Delta x}=\lim_{\Delta x\rightarrow 0}\frac{f(a+\Delta x)-f(a)}{\Delta x} =\\
&=\lim_{\Delta x\rightarrow 0}\frac{2a\Delta x+\Delta x^2}{\Delta x}=\lim_{\Delta x\rightarrow 0} 2a+\Delta x= 2a. 
\end{align*}
Así pues,
\[
f'(a)=2a.
\]

El signo de $f'(a)$ indica la tendencia de crecimiento de $f$ en el punto $a$:
\begin{itemize}
\item[--]  $f'(a)>0$ indica que la tendencia es creciente.
\item[--]  $f'(a)<0$ indica que la tendencia es decreciente.
\end{itemize}
\end{frame}


%---------------------------------------------------------------------slide----
\begin{frame}
\frametitle{Interpretación geométrica de la tasa de variación instantánea}
La tasa de variación instantánea de $f$ en el punto $a$ es la pendiente de la recta \emph{tangente} a $f$ en el punto $(a,f(a))$.
\begin{center}
\scalebox{1}{\psset{xunit=2,algebraic}
\begin{pspicture*}(-1.5,-0.5)(3,3.5)
\psaxes[ticks=none,labels=none]{<->}(0,0)(-0.3,-0.5)(2,3.5)
\psplot[linecolor=blue]{0.2}{1.5}{x^3-x^2+x+0.5}
\footnotesize
\rput[r](1.4,3.2){$f(x)$}
\psxTick[ticksize=-3pt 0,labelsep=3pt](0.5){a}
\psxTick[ticksize=-3pt 0,labelsep=3pt](1.5){x}
\psline[linewidth=0.5pt,linestyle=dashed,linecolor=gray](0.5,0)(0.5,0.875)
\psline[linewidth=0.5pt,linestyle=dashed,linecolor=gray](1.5,0.875)(0,0.875)
\psline[linewidth=0.5pt,linestyle=dashed,linecolor=gray](1.5,0)(1.5,1.625)(0,1.625)
\psyTick[ticksize=-3pt 0,labelsep=3pt](0.875){f(a)}
\psyTick[ticksize=-3pt 0,labelsep=3pt](1.625){f(a)+f'(a)(x-a)}
\psplot[linecolor=red]{0.2}{1.8}{0.75*x+0.5}
\psline[arrows=|*-|*,linecolor=green](1.5,1.625)(1.5,0.875)
\psline[arrows=|*-|*,linecolor=green](0.5,0.875)(1.5,0.875)
\rput[l](1.6,1.25){$f'(a)(x-a)$}
\rput[t](1,0.8){$(x-a)$}
\end{pspicture*}}
\end{center}
\end{frame}



\subsection{Rectas tangente y normal}
%---------------------------------------------------------------------slide----
\begin{frame}
\frametitle{Tasa de variación instantánea}
\begin{definicion}[Recta tangente y normal a una función en un punto]
Dada una función $f$, se llama \emph{recta tangente} a $f$ en un punto $(a,f(a))$, a la recta de ecuación 
\[
y=f(a)+f'(a)(x-a).
\]
Se llama \emph{recta normal} a $f$ en un punto $(a,f(a))$, a la recta de ecuación 
\[
y=f(a)-\frac{1}{f'(a)}(x-a).
\]
\end{definicion}

\begin{center}
\scalebox{1}{\psset{unit=0.9,algebraic}
\begin{pspicture*}(-2.5,-0.5)(4.5,3.5)
\psaxes[ticks=none,labels=none]{<->}(0,0)(-0.3,-0.5)(2,3.5)
\psplot[linecolor=blue]{0.2}{1.5}{x^3-x^2+x+0.5}
\footnotesize
\rput[r](1.4,3.3){$f(x)$}
\psxTick[ticksize=-3pt 0,labelsep=3pt](1){a}
\psline[linewidth=0.5pt,linestyle=dashed,linecolor=gray](1,0)(1,1.5)(0,1.5)
\psyTick[ticksize=-3pt 0,labelsep=3pt](1.5){f(a)}
\psplot[linecolor=red]{0.2}{1.8}{2*x-0.5}
\psplot[linecolor=red]{0.2}{1.8}{2-0.5*x}
\rput[l](1.6,2.5){$y=f(a)+f'(a)(x-a)$}
\rput[l](1.6,0.8){$y=f(a)-\dfrac{1}{f'(a)}(x-a)$}
\end{pspicture*}}
\end{center}
\end{frame}



\subsection{El concepto de diferencial}
%---------------------------------------------------------------------slide----
\begin{frame}
\frametitle{El concepto de diferencial}
\begin{definicion}[Diferencial de una función en un punto]
Dada una función $f$, se llama \emph{diferencial} de $f$ en un punto $a$, al la función
\[
\begin{array}{rccc}
dy=df(a): & \mathbb{R} & \longrightarrow & \mathbb{R} \\
& \Delta x & \longrightarrow & f'(a)\Delta x
\end{array}
\]
\end{definicion}

Cuando $f$ es la función identidad $y=x$,  entonces $f'(a)=1$, y se cumple que
\[ dx=dy=f'(a)\Delta x=\Delta x,\]
de modo que también podemos definir el diferencial como 
\[dy=df(a)=f'(a)dx.\]

De aquí se deduce otra forma de escribir la derivada de $f$ en $a$
\[f'(a)=\frac{dy}{dx}=\frac{df(a)}{dx}.\]
\end{frame}


%---------------------------------------------------------------------slide----
\begin{frame}
\frametitle{Aproximación de una función mediante su diferencial}
El diferencial de una función $f$ en un punto $a$, permite aproximar la variación de $f$ cerca de $a$. 

\begin{center}
\scalebox{1}{\psset{xunit=2,yunit=1.2,algebraic}
\begin{pspicture*}(-1.2,-0.5)(4.3,4)
\psaxes[ticks=none,labels=none]{<->}(0,0)(-0.3,-0.5)(2,4)
\psplot[linecolor=blue]{0.2}{1.6}{x^3-x^2+x+0.5}
\rput[r](1.4,3.5){$f(x)$}
\psxTick[ticksize=-3pt 0,labelsep=3pt](0.5){a}
\psyTick[ticksize=-3pt 0,labelsep=3pt](0.875){f(a)}
\psline[linewidth=0.5pt,linestyle=dashed,linecolor=gray](0.5,0)(0.5,0.875)(0,0.875)
\uncover<2->{
\psxTick[ticksize=-3pt 0,labelsep=3pt](1.5){a+dx}
\psline[linewidth=0.5pt,linestyle=dashed,linecolor=gray](1.5,0)(1.5,3.125)(0,3.125)
\psyTick[ticksize=-3pt 0,labelsep=3pt](3.125){f(a+dx)}
\psline[arrows=|*-|*,linecolor=green](0.5,0.875)(1.5,0.875)
\rput[t](1,0.8){$dx$}
}
\uncover<3->{
\psline[linestyle=dotted,linecolor=gray](1.5,0.875)(2.6,0.875)
\psline[linestyle=dotted,linecolor=gray](1.5,3.125)(2.6,3.125)
\psline[arrows=|*-|*,linecolor=orange](2.6,0.875)(2.6,3.125)
\rput[l](2.7,1.9){$\Delta y=f(a+dx)-f(a)$}
}
\uncover<4->{
\psplot[linecolor=red]{0.2}{1.8}{0.75*x+0.5}
}
\uncover<5->{
\psline[linewidth=0.5pt,linestyle=dashed,linecolor=gray](1.5,1.625)(0,1.625)
\rput[r](-0.1,1.625){$f(a)+f'(a)dx$}
}
\uncover<6->{
\psline[arrows=|*-|*,linecolor=green](1.5,1.625)(1.5,0.875)
\rput[l](1.6,1.25){$dy=f'(a)dx$}
\rput[r](-0.4,2.4){\rotatebox{90}{$\approx$}}
}
\end{pspicture*}}
\end{center}
\end{frame}


%---------------------------------------------------------------------slide----
\begin{frame}
\frametitle{Aproximación de una función mediante su diferencial: Ejemplo}
Consideremos otra vez la función $y=x^2$ que mide el área de un cuadrado de chapa metálica de lado $x$.

Si el lado del cuadrado es $a$, y sometemos la chapa a un proceso de calentamiento que aumenta el lado del cuadrado, ¿cuál  será aproximadamente la variación que habrá experimentado el área, cuando el lado aumente una cantidad $dx$?
\begin{columns}
\begin{column}{0.3\textwidth}
\begin{align*}
\Delta y &= f(a+dx)-f(a)=(a+dx)^2-a^2=\\
&= a^2+2adx+dx^2-a^2=2adx+dx^2,\\
\only<2->{dy &= f'(a)dx= 2adx.}
\end{align*}
\uncover<3->{Además, \[\lim_{dx\rightarrow 0}\Delta y-dy=\lim_{dx\rightarrow 0}dx^2=0.\]}
\end{column}
\begin{column}{0.3\textwidth}
\begin{center}
\scalebox{1}{\psset{unit=0.6}
\begin{pspicture*}(0,-0.5)(4,4)
\footnotesize
\psframe(0,0)(3,3)
\rput[t](1.5,-0.1){$a$}
\rput[t](1.5,1.7){$a^2$}
\psframe[fillstyle=solid,fillcolor=gray](0,3)(3,4)
\psframe[fillstyle=solid,fillcolor=gray](3,3)(4,4)
\psframe[fillstyle=solid,fillcolor=gray](3,0)(4,3)
\rput[t](3.5,-0.1){$dx$}
\rput[t](1.5,3.7){$adx$}
\rput[t](3.5,3.7){$dx^2$}
\rput[t](3.5,1.7){$adx$}
\uncover<2->{
\psframe[fillstyle=solid,fillcolor=orange](0,3)(3,4)
\psframe[fillstyle=solid,fillcolor=orange](3,0)(4,3)
\rput[t](1.5,3.7){$adx$}
\rput[t](3.5,1.7){$adx$}
}
\end{pspicture*}}
\end{center}
\end{column}
\end{columns}
\end{frame}



\subsection{Álgebra de derivadas}
%---------------------------------------------------------------------slide----
\begin{frame}
\frametitle{Propiedades del diferencial}
Si $y=c$, es una función constante, entonces $dy=0$.
Si $y=x$, es la función identidad, entonces  $dy=dx$.

Si $u=f(x)$ y $v=g(x)$ son dos funciones diferenciables, entonces
\begin{itemize}
\item $d(u+v)=d(u)+d(v)$
\item $d(u-v)=d(u)-d(v)$
\item $d(u\cdot v)=d(u)\cdot v+ u\cdot d(v)$
\item $d\left(\dfrac{u}{v}\right)=\dfrac{du\cdot v-u\cdot dv}{v^2}$
\end{itemize}
\end{frame}

\subsection{Derivada de una función compuesta: La regla de la cadena}

%---------------------------------------------------------------------slide----
\begin{frame}
\frametitle{Diferencial de una función compuesta}
\framesubtitle{La regla de la cadena}
Si $y=f\circ g$ es la composición de dos funciones $y=f(z)$ y $z=g(x)$, entonces
\[dy=f'(z)dz=f'(g(x))g'(x)dx,\]
de donde se deduce
\[
(f\circ g)'(x)=\frac{dy}{dx}=\frac{f'(g(x))g'(x)dx}{dx}=f'(g(x))g'(x),
\]
o bien
\[
(f\circ g)'(x)=\frac{dy}{dx}=\frac{dy}{dz}\frac{dz}{dx}=f'(z)g'(x)=f'(g(x))g'(x).
\]
\end{frame}



\subsection{Derivada de la inversa de una función}
%---------------------------------------------------------------------slide----
\begin{frame}
\frametitle{Derivada de la función inversa}
Si $y=f(x)$ es una función y $x=f^{-1}(y)$ es su inversa, entonces
\[dy=f'(x)dx \quad \textrm{y} \quad dx=\left(f^{-1}\right)'(y)dy,\]
de donde se deduce
\[
\left(f^{-1}\right)'(y)=\frac{dx}{dy}=\frac{dx}{f'(x)dx}=\frac{1}{f'(x)}=\frac{1}{f'(f^{-1}(y))}
\]
o bien
\[
\left(f^{-1}\right)'(y)=\frac{dx}{dy}=\frac{1}{dy/dx}=\frac{1}{f'(x)}=\frac{1}{f'(f^{-1}(y))}
\]
\end{frame}



\subsection{Derivada de una función implícita}
%---------------------------------------------------------------------slide----
\begin{frame}
\frametitle{Derivada de una función implícita}
Si $F(x,y)=0$ es una función implícita entonces 
\[
dF(x,y)=d0=0.
\]

Si $F(x,y)=0$ es una función implícita en la que $y$ depende de $x$, entonces podemos calcular la derivada de $y$ a partir del diferencial
\[
\frac{dF(x,y)}{dx}=\frac{d0}{dx}=0.
\]
\structure{\textbf{Ejemplo}}. Consideremos la función implícita de la circunferencia de radio 1, $x^2+y^2=1$. Entonces su diferencial es
\[
d(x^2+y^2)=d1=0 \Leftrightarrow d(x^2)+d(y^2)=2x\;dx+2y\;dy=0.
\]
A partir de aquí podemos calcular fácilmente la derivada de $y$:
\[
\frac{d(x^2+y^2)}{dx}= \frac{2x\;dx+2y\;dy}{dx}=2x\frac{dx}{dx}+2y\frac{dy}{dx}= 2x+2y\frac{dy}{dx}=0 \Leftrightarrow \frac{dy}{dx}=\frac{-x}{y}.
\]

\end{frame}



\subsection{Derivada de una función paramétrica}
%---------------------------------------------------------------------slide----
\begin{frame}
\frametitle{Derivada de una función parametrica}

Dada una función paramétrica
\[
\left\{%
\begin{array}{l}
x=f(t) \\
y=g(t) 
\end{array}%
\right.  
\]
podemos calcular su derivada a partir de las derivadas de $f$ y $g$:
\[\frac{dy}{dx}=\frac{g'(t)\,dt}{f'(t)\,dt}=\frac{g'(t)}{f'(t)}.\]

\structure{\textbf{Ejemplo}}. Consideremos la elipse
\[
\left\{%
\begin{array}{l}
x=2\sen t \\
y=\cos t
\end{array}%
\right. 
\]
Entonces 
\[
\frac{dy}{dx}=\frac{-\sen t\; dt}{2 \cos t\; dt}=\frac{-1}{2}\tg t.
\]
\end{frame}
