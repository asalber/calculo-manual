%Version control information
%$HeadURL: https://ejercicioscalculo.googlecode.com/svn/trunk/compendio_ejercicios_calculo.tex $} {$LastChangedDate: 2008-07-09 16:26:20 +0200 (mi�, 09 jul 2008) $
%$LastChangedRevision: 6 $
%$LastChangedBy: asalber $


\section{Cálculo diferencial en una variable} 
%---------------------------------------------------------------------slide----
\begin{frame}
\frametitle{Cálculo diferencial en una variable}
\tableofcontents[sectionstyle=show/hide,hideothersubsections]
\end{frame}


\subsection{El concepto de derivada}
%---------------------------------------------------------------------slide----
\begin{frame}
\frametitle{Tasa de variación media}
\begin{definicion}[Incremento]
Dada una función $y=f(x)$, se llama \emph{incremento} de $f$ en un intervalo $[a,b]$ a la diferencia entre el valor de $f$ en cada uno de los extremos del intervalo, y se nota
\[\Delta y= f(b)-f(a).\]
\end{definicion}

Cuando $f$ es la función identidad $y=x$, se cumple que
\[\Delta x=\Delta y= f(b)-f(a)=b-a,\]
y por tanto, el incremento de $x$ en un intervalo es la amplitud del intervalo. Esto nos permite escribir el intervalo $[a,b]$ como $[a,a+\Delta x]$.

\begin{definicion}[Tasa de variación media]
Se llama \emph{tasa de variación media} de $f$ en el intervalo $[a,a+\Delta x]$, al cociente entre el incremento de $y$ y el incremento de $x$ en dicho intervalo, y se escribe
\[
\textrm{TVM}\;f[a,a+\Delta x]=\frac{\Delta y}{\Delta x}=\frac{f(a+\Delta x)-f(a)}{\Delta x} 
\]
\end{definicion}
\end{frame}


%---------------------------------------------------------------------slide----
\begin{frame}
\frametitle{Tasa de variación media: Ejemplo}
Consideremos la función $y=x^2$ que mide el área de un cuadrado de chapa metálica de lado $x$.

Si en un determinado instante el lado del cuadrado es $a$, y sometemos la chapa a un proceso de calentamiento que aumenta el lado del cuadrado una cantidad $\Delta x$, ¿en cuánto se incrementará el área del cuadrado?
\begin{columns}
\begin{column}{0.3\textwidth}
\begin{align*}
\Delta y &= f(a+\Delta x)-f(a)=(a+\Delta x)^2-a^2=\\
&= a^2+2a\Delta x+\Delta x^2-a^2=2a\Delta x+\Delta x^2.
\end{align*}
\end{column}
\begin{column}{0.3\textwidth}
\begin{center}
\scalebox{1}{\psset{unit=0.6}
\begin{pspicture*}(0,-0.5)(4,4)
\scriptsize
\psframe(0,0)(3,3)
\rput[t](1.5,-0.1){$a$}
\rput[t](1.5,1.7){$a^2$}
\psframe[fillstyle=solid,fillcolor=orange](0,3)(3,4)
\psframe[fillstyle=solid,fillcolor=orange](3,3)(4,4)
\psframe[fillstyle=solid,fillcolor=orange](3,0)(4,3)
\rput[t](3.5,-0.1){$\Delta x$}
\rput[t](1.5,3.7){$a\Delta x$}
\rput[t](3.5,3.7){$\Delta x^2$}
\rput[t](3.5,1.7){$a\Delta x$}
\end{pspicture*}}
\end{center}
\end{column}
\end{columns}
¿Cuál será la tasa de variación media del área en el intervalo $[a,a+\Delta x]$?
\[
\textrm{TVM}\;f[a,a+\Delta x]=\frac{\Delta y}{\Delta x}=\frac{2a\Delta x+\Delta x^2}{\Delta x}=2a+\Delta x. 
\]
\end{frame}


%---------------------------------------------------------------------slide----
\begin{frame}
\frametitle{Interpretación geométrica de la tasa de variación media}
La tasa de variación media de $f$ en el intervalo $[a,a+\Delta x]$ es la pendiente de la recta \emph{secante} a $f$ en los puntos $(a,f(a))$ y $(a+\Delta x,f(a+\Delta x))$.
\begin{center}
\scalebox{1}{\psset{xunit=2,algebraic}
\begin{pspicture*}(-1,-0.5)(3,4)
\psaxes[ticksize=-3pt 0,labelsep=3pt,ticks=none]{<->}(0,0)(-0.3,-0.5)(2.5,3.8)[$x$,-90][$y$,180]
\psplot[linecolor=blue]{0.2}{1.8}{x^3-x^2+x+0.5}
\footnotesize
\rput[r](1.4,3.5){$f(x)$}
\psxTick[ticksize=-3pt 0,labelsep=3pt](0.5){a}
\psxTick[ticksize=-3pt 0,labelsep=3pt](1.5){a+\Delta x}
\psline[linewidth=0.5pt,linestyle=dashed,linecolor=gray](0.5,0)(0.5,0.875)
\psline[linewidth=0.5pt,linestyle=dashed,linecolor=gray](1.5,0.875)(0,0.875)
\psline[linewidth=0.5pt,linestyle=dashed,linecolor=gray](1.5,0)(1.5,3.125)(0,3.125)
\psyTick[ticksize=-3pt 0,labelsep=3pt](0.875){f(a)}
\psyTick[ticksize=-3pt 0,labelsep=3pt](3.125){f(a+\Delta x)}
\psplot[linecolor=red]{0.2}{1.8}{2.25*x-0.25}
\psline[arrows=|*-|*,linecolor=green](1.5,3.125)(1.5,0.875)
\psline[arrows=|*-|*,linecolor=green](0.5,0.875)(1.5,0.875)
\rput[l](1.6,2){$\Delta y=f(a+\Delta x)-f(a)$}
\rput[t](1,0.8){$\Delta x$}
\end{pspicture*}}
\end{center}
\end{frame}


%---------------------------------------------------------------------slide----
\begin{frame}
\frametitle{Tasa de variación instantánea}
En muchas ocasiones, es interesante estudiar la tasa de variación que experimenta una función, no en intervalo, sino en
un punto.

Conocer la tendencia de variación de una función en un instante puede ayudarnos a predecir valores en instantes
próximos.

\begin{definicion}[Tasa de variación instantánea y derivada]
Dada una función $y=f(x)$, se llama \emph{tasa de variación instantánea} de $f$ en un punto $a$, al límite de la tasa de
variación media de $f$ en el intervalo $[a,a+\Delta x]$, cuando $\Delta x$ tiende a 0, y lo notaremos
\[
\textrm{TVI}\;f (a)=\lim_{\Delta x\rightarrow 0} \textrm{TVM}\; f[a,a+\Delta x]=\lim_{\Delta x\rightarrow 0}\frac{\Delta y}{\Delta x}=\lim_{\Delta x\rightarrow 0}\frac{f(a+\Delta x)-f(a)}{\Delta x} 
\]
Cuando este límite existe, se dice que la función $f$ es derivable en el punto $a$, y al valor del mismo se le llama
derivada de $f$ en $a$, y se nota como
\[
f'(a) \mbox{ o bien } \frac{df}{dx}(a)
\]
\end{definicion}
\end{frame}


%---------------------------------------------------------------------slide----
\begin{frame}
\frametitle{Tasa de variación instantánea: Ejemplo}
Consideremos de nuevo la función $y=x^2$ que mide el área de un cuadrado de chapa metálica de lado $x$.

Si en un determinado instante el lado del cuadrado es $a$, y sometemos la chapa a un proceso de calentamiento que
aumenta el lado del cuadrado, ¿cuál es la tasa de variación instantánea del área del cuadrado en dicho instante?
\begin{align*}
\textrm{TVI}\;f(a)]&=\lim_{\Delta x\rightarrow 0}\frac{\Delta y}{\Delta x}=\lim_{\Delta x\rightarrow 0}\frac{f(a+\Delta x)-f(a)}{\Delta x} =\\
&=\lim_{\Delta x\rightarrow 0}\frac{2a\Delta x+\Delta x^2}{\Delta x}=\lim_{\Delta x\rightarrow 0} 2a+\Delta x= 2a. 
\end{align*}
Así pues,
\[
f'(a)=2a,
\]
lo que indica que la tendencia de crecimiento el área es del doble del valor del lado.  

El signo de $f'(a)$ indica la tendencia de crecimiento de $f$ en el punto $a$:
\begin{itemize}
\item[--]  $f'(a)>0$ indica que la tendencia es creciente.
\item[--]  $f'(a)<0$ indica que la tendencia es decreciente.
\end{itemize}
\end{frame}


%---------------------------------------------------------------------slide----
\begin{frame}
\frametitle{Interpretación geométrica de la tasa de variación instantánea}
La tasa de variación instantánea de $f$ en el punto $a$ es la pendiente de la recta \emph{tangente} a $f$ en el punto $(a,f(a))$.
\begin{center}
\scalebox{1}{\psset{xunit=2,algebraic}
\begin{pspicture*}(-1.5,-0.5)(3,3.5)
\psaxes[ticks=none,labels=none]{<->}(0,0)(-0.3,-0.5)(2,3.3)[$x$,-90][$y$,180]
\psplot[linecolor=blue]{0.2}{1.5}{x^3-x^2+x+0.5}
\footnotesize
\rput[r](1.4,3.2){$f(x)$}
\psxTick[ticksize=-3pt 0,labelsep=3pt](0.5){a}
\psxTick[ticksize=-3pt 0,labelsep=3pt](1.5){x}
\psline[linewidth=0.5pt,linestyle=dashed,linecolor=gray](0.5,0)(0.5,0.875)
\psline[linewidth=0.5pt,linestyle=dashed,linecolor=gray](1.5,0.875)(0,0.875)
\psline[linewidth=0.5pt,linestyle=dashed,linecolor=gray](1.5,0)(1.5,1.625)(0,1.625)
\psyTick[ticksize=-3pt 0,labelsep=3pt](0.875){f(a)}
\psyTick[ticksize=-3pt 0,labelsep=3pt](1.625){f(a)+f'(a)(x-a)}
\psplot[linecolor=red]{0.2}{1.8}{0.75*x+0.5}
\psline[arrows=|*-|*,linecolor=green](1.5,1.625)(1.5,0.875)
\psline[arrows=|*-|*,linecolor=green](0.5,0.875)(1.5,0.875)
\rput[l](1.6,1.25){$f'(a)(x-a)$}
\rput[t](1,0.8){$(x-a)$}
\end{pspicture*}}
\end{center}
\end{frame}



\subsection{Álgebra de derivadas}
%---------------------------------------------------------------------slide----
\begin{frame}
\frametitle{Propiedades de la derivada}
Si $y=c$, es una función constante, entonces $y'=0$.
Si $y=x$, es la función identidad, entonces  $y'=1$.

Si $u=f(x)$ y $v=g(x)$ son dos funciones diferenciables, entonces
\begin{itemize}
\item $(u+v)'=u'+v'$
\item $(u-v)'=u'-v'$
\item $(u\cdot v)'=u'\cdot v+ u\cdot v'$
\item $\left(\dfrac{u}{v}\right)'=\dfrac{u'\cdot v-u\cdot v'}{v^2}$
\end{itemize}
\end{frame}


\subsection{Derivada de una función compuesta: La regla de la cadena}

%---------------------------------------------------------------------slide----
\begin{frame}
\frametitle{Diferencial de una función compuesta}
\framesubtitle{La regla de la cadena}
Si $y=f\circ g$ es la composición de dos funciones $y=f(z)$ y $z=g(x)$, entonces
\[
(f\circ g)'(x)=f'(g(x))g'(x),
\]
o bien
\[
\frac{dy}{dx}=\frac{dy}{dz}\frac{dz}{dx}=f'(z)g'(x)=f'(g(x))g'(x).
\]
\end{frame}



\subsection{Derivada de la inversa de una función}
%---------------------------------------------------------------------slide----
\begin{frame}
\frametitle{Derivada de la función inversa}
Si $y=f(x)$ es una función y $x=f^{-1}(y)$ es su inversa, entonces
\[
\left(f^{-1}\right)'(y)=\frac{1}{f'(x)}=\frac{1}{f'(f^{-1}(y))}
\]
o bien
\[
\frac{dx}{dy}=\frac{1}{dy/dx}=\frac{1}{f'(x)}=\frac{1}{f'(f^{-1}(y))}
\]
\end{frame}



\subsection{Interpretación cinemática de la derivada}
% ---------------------------------------------------------------------slide----
\begin{frame}
\frametitle{Interpretación cinemática de la tasa de variación}
\framesubtitle{Movimiento rectilineo}
Supongase que la función $f(t)$ describe la posición de un objeto móvil sobre la recta real en el instante $t$.
Tomando como referencia el origen de coordenadas $O$ y el vector unitario $\mathbf{i}=(1)$, se puede representar la
posición $P$ del móvil en cada instante $t$ mediante un vector $\vec{OP}=x\mathbf{i}$ donde $x=f(t)$.
\begin{center}
\scalebox{1}{\psset{unit=1,algebraic}
\begin{pspicture*}(-5,-1.6)(5,1.5)
\psaxes[ticks=none,labels=none]{-}(0,0)(-0.5,0)(4,0)
\rput[b](2,1){Posición}
\psdot(3,0)
\psline[linecolor=blue]{->}(0,0)(3,0)
\psline[linecolor=red]{->}(0,0)(1,0)
\rput[b](0.5,0.1){$\mathbf{i}$}
\rput[b](3,0.1){$P$}
\rput[l](4.1,0){$\mathbb{R}$}
\psxTick[labelFontSize=\scriptstyle,ticksize=-3pt 0,labelsep=4pt](0){O}
\psxTick[labelFontSize=\scriptstyle,ticksize=-3pt 0,labelsep=4pt](1){1}
\psxTick[ticksize=-3pt 0,labelsep=4pt](3){}
\rput[l](2.9,-0.3){$x=f(t)$}
\psline(-4,0)(-2,0)
\rput[b](-3,1){Tiempo}
\psxTick[ticksize=-3pt 0,labelsep=4pt](-3){t}
\psbezier[linestyle=dashed]{->}(-2.5,-0.3)(-1,-1)(1,-1)(2.5,-0.3)
\rput[t](0,-1){$f$}
\end{pspicture*}}
\end{center}

\structure{\textbf{Observación}}
También tiene sentido pensar en $f$ como una función que mide otras magnitudes como por ejemplo la temperatura de un
cuerpo, la concentración de un gas o la cantidad de un compuesto en una reacción química en un instante $t$.
\end{frame}


% ---------------------------------------------------------------------slide----
\begin{frame}
\frametitle{Interpretación cinemática de la tasa de variación media}
En este contexto, si se toman los instantes $t=t_0$ y $t=t_0+\Delta t$, ambos del dominio $I$ de $f$, el vector
\[
\mathbf{v}_m=\frac{f(t_0+\Delta t)-f(t_0)}{\Delta t}
\]
que se conoce como \emph{velocidad media} de la trayectoria $f$ entre los instantes $t_0$ y $t_0+\Delta t$. 

\structure{\textbf{Ejemplo}}
Un vehículo realiza un viaje de Madrid a Barcelona.
Sea $f$ la función que da la posición el vehículo en cada instante.
Si el vehículo parte de Madrid (km 0) a las 8 y llega a Barcelona (km 600) a las 14 horas, entonces la velocidad media
del vehículo en el trayecto es 
\[ 
\mathbf{v}_m=\frac{f(14)-f(8)}{14-8}=\frac{600-0}{6} = 100 km/h. 
\]
\end{frame}


% ---------------------------------------------------------------------slide----
\begin{frame}
\frametitle{Interpretación cinemática de la derivada}
Siguiendo en este mismo contexto del movimiento rectilineo, la derivada de $f$ en el instante $t=t_0$ es el vector
\[
\mathbf{v}=f'(t_0)=\lim_{\Delta x\rightarrow 0}\frac{f(t_0+\Delta t)-f(t_0)}{\Delta t},
\]
que se conoce, siempre que exista el límite, como \emph{velocidad instantánea} o simplemente la \emph{velocidad} de la
trayectoria $f$ en el instante $t_0$.

Es decir, la derivada de la posición respecto del tiempo, es un campo de vectores que recibe el nombre de
\emph{velocidad a lo largo de la trayectoria $f$}.

\structure{\textbf{Ejemplo}}
Siguiendo con el ejemplo anterior, la velocidad instantánea del vehículo en un determinado instante sería lo que marca
el velocímetro en dicho instante.
\end{frame}


% ---------------------------------------------------------------------slide----
\begin{frame}
\frametitle{Generalización al movimiento curvilineo}
La derivada como velocidad a lo largo de una trayectoria en la recta real puede generalizarse a trayectorias en cualquier
espacio euclídeo $\mathbb{R}^n$.

Para el caso del plano real $\mathbb{R}^2$, si $f(t)$ describe la posición de un objeto móvil en el plano en el instante
$t$, tomando como referencia el origen de coordenadas $O$ y los vectores coordenados
$\{\mathbf{i}=(1,0),\mathbf{j}=(0,1)\}$, se puede representar la posición $P$ del móvil en cada instante $t$ mediante un
vector $\vec{OP}=x(t)\mathbf{i}+y(t)\mathbf{j}$ cuyas coordenadas
\[
\begin{cases}
x=x(t)\\
y=y(t)
\end{cases}
\quad
t\in I\subseteq \mathbb{R}
\]
se conocen como \emph{funciones coordenadas} de $f$ y se escribe $f(t)=(x(t),y(t))$.

\begin{center}
\scalebox{0.8}{\psset{unit=0.8,algebraic}
\begin{pspicture*}(-5,-0.5)(5,3.2)
\psaxes[ticks=none,labels=none]{<->}(0,0)(-0.5,-0.5)(3,3)[$x$,-90][$y$,180]
\rput[b](2.2,2.7){Posición}
\psline[linecolor=red]{->}(0,0)(1,0)
\psline[linecolor=red]{->}(0,0)(0,1)
\rput[b](0.8,0.1){$\mathbf{i}$}
\rput[l](0.1,0.8){$\mathbf{j}$}
\psdot(2,2)
\psline[linecolor=orange]{->}(0,0)(2,2)
\rput[l](2.1,1.8){$P=f(t)$}
\rput[l](3.5,2.7){$\mathbb{R}^2$}
\psxTick[labelFontSize=\scriptstyle,ticksize=-3pt 0,labelsep=4pt](1){1}
\psyTick[labelFontSize=\scriptstyle,ticksize=-3pt 0,labelsep=4pt](1){1}
\psline[linecolor=gray,linestyle=dashed](2,0)(2,2)(0,2)
\psxTick[ticksize=-3pt 0,labelsep=4pt](2){x(t)}
\psyTick[ticksize=-3pt 0,labelsep=4pt](2){y(t)}
\psline(-4,1)(-2,1)
\rput[b](-3,2.7){Tiempo}
\psline[linewidth=0.5pt](-3,1)(-3,0.85)
\rput[t](-3,0.8){$t$}
\psbezier[linestyle=dashed]{->}(-2.8,1.2)(-1,3)(0,3)(1.8,2.2)
\rput[t](-1,3){$f$}
\parametricplot[linecolor=blue]{-2}{2.5}{t+1.2146|sin(t)+1.2929}
\end{pspicture*}}
\end{center}  
\end{frame}


% ---------------------------------------------------------------------slide----
\begin{frame}
\frametitle{Velocidad en una trayectoria curvilinea en el plano}
En este contexto de una trayectoria $f(t)=(x(t),y(t))$ en el plano real $\mathbb{R}^2$, para un instante $t=t_0$, si existe el vector
\[
\mathbf{v} = \lim_{\Delta t\rightarrow 0} \frac{f(t_0+\Delta t)-f(t_0)}{\Delta t},
\] 
entonces $f$ es derivable en el instante $t=t_0$ y el vector $\mathbf{v}=f'(t)$ se conoce como \emph{velocidad} de $f$ en ese instante.

Como $f(t)=(x(t),y(t))$, 
\begin{align*}
f'(t)&=\lim_{\Delta t\rightarrow 0} \frac{f(t_0+\Delta t)-f(t_0)}{\Delta t} = \lim_{\Delta t\rightarrow 0} \frac{(x(t_0+\Delta t),y(t_0+\Delta t))-(x(t_0),y(t_0))}{\Delta t} =\\
&=  \lim_{\Delta t\rightarrow 0} \left(\frac{x(t_0+\Delta t)-x(t_0)}{\Delta t},\frac{y(t_0+\Delta t)-y(t_0)}{\Delta t}\right) =\\
&= \left(\lim_{\Delta t\rightarrow 0}\frac{x(t_0+\Delta t)-x(t_0)}{\Delta t},\lim_{\Delta t\rightarrow 0}\frac{y(t_0+\Delta t)-y(t_0)}{\Delta t}\right) = 
(x'(t),y'(t)). 
\end{align*} 
luego
\[
\mathbf{v} = x'(t)\mathbf{i}+y'(t)\mathbf{j}.
\]
\end{frame}


% ---------------------------------------------------------------------slide----
\begin{frame}
\frametitle{Velocidad en una trayectoria curvilinea en el plano}
\framesubtitle{Ejemplo}
Dada la trayectoria $f(t) = (\cos t,\sen t)$, $t\in \mathbb{R}$, cuya gráfica es la circunferencia de centro el origen
de coordenas y radio 1, sus funciones coordenadas son $x(t) = \cos t$, $y(t) = \sen t$, $t\in \mathbb{R}$, y su velocidad es 
\[
\mathbf{v}=f'(t)=(x'(t),y'(t))=(-\sen t, \cos t)
\]
En el instante $t=\pi/4$, el móvil estará en la posición $f(\pi/4) = (\cos(\pi/4),\sen(\pi/4)) =(\sqrt{2}/2,\sqrt{2}/2)$
y se moverá con una velocidad $\mathbf{v}=f'(\pi/4)=(-\sen(\pi/4),\cos(\pi/4))=(-\sqrt{2}/2,\sqrt{2}/2)$.
\begin{center}
\scalebox{0.8}{\psset{unit=1.2,algebraic}
\begin{pspicture*}(-2,-1.4)(2,1.5)
\psaxes[labelFontSize=\scriptstyle,ticksize=-3pt 0,labelsep=2pt]{<->}(0,0)(-1.4,-1.4)(1.4,1.4)[$x$,-90][$y$,180]
\pscircle[linecolor=blue](0,0){1}
\rput[l](-1,1){$f$}
\psline[linecolor=red]{->}(0.7071,0.7071)(0,1.4142)
\psdot(0.7071,0.7071)
\rput[l](0.8,0.8){$t=\pi/4$}
\rput[l](0.3,1.2){$\mathbf{v}$}
\end{pspicture*}}
\end{center}  
Obsérvese que el módulo del vector velocidad siempre será 1 ya que 
$|\mathbf{v}|=\sqrt{(-\sen t)^2+(\cos t)^2}=1$.
\end{frame}



\subsection{Recta tangente a una trayectoria}
% ---------------------------------------------------------------------slide----
\begin{frame}
\frametitle{Recta tangente a una trayectoria en el plano}
Los vectores paralelos a la velocidad $\mathbf{v}$ se denominan \emph{vectores tangentes} a la gráfica de la trayectoria
$f$ en el punto $P=f(t_0)$ y la recta que pasa por $P$ dirigida por $\mathbf{v}$ es la recta tangente a $f$ cuando
$t=t_0$.
\begin{definicion}[Recta tangente a una trayectoria]
Dada una trayectoria $f$ sobre el plano real $\mathbb{R}^2$, se llama \emph{recta tangente} a $f$ para $t=t_0$ a la
recta de ecuación
\[
l = f(t_0)+tf'(t_0) = (x(t_0),y(t_0))+t(x'(t_0),y'(t_0)) = (x(t_0)+tx'(t_0),y(t_0)+ty'(t_0)). 
\]
\end{definicion}
\structure{\texbf{Ejemplo}}
Se ha visto que para la trayectoria $f(t) = (\cos t,\sen t)$, $t\in \mathbb{R}$, cuya gráfica es la circunferencia de
centro el origen de coordenas y radio 1, en el instante $t=\pi/4$ la posición del móvil era
$f(\pi/4)=(\sqrt{2}/2,\sqrt{2}/2)$ y su velocidad $\mathbf{v}=(-\sqrt{2}/2,\sqrt{2}/2)$, de modo que la recta tangente a
$f$ en ese instante es
\[
l=f(\pi/2)+t\mathbf{v} = 
\left(\frac{\sqrt{2}}{2},\frac{\sqrt{2}}{2}\right)+t\left(\frac{-\sqrt{2}}{2},\frac{\sqrt{2}}{2}\right) = 
\left(\frac{\sqrt{2}}{2}-t\frac{\sqrt{2}}{2},\frac{\sqrt{2}}{2}+t\frac{\sqrt{2}}{2}\right).
\]
\end{frame}


% ---------------------------------------------------------------------slide----
\begin{frame}
\frametitle{Recta tangente a una trayectoria en el plano}
De la ecuación vectorial de la recta tangente a $f$ para $t=t_0$, se obtiene que sus funciones cartesianas son
\[
\begin{cases}
x=x(t_0)+tx'(t_0)\\
y=y(t_0)+ty'(t_0)
\end{cases}
\quad t\in \matbb{R},
\]
y despejando $t$ en ambas ecuaciones e igualando se llega a la ecuación cartesiana de la recta tangente
\[
\frac{x-x(t_0)}{x'(t_0)}=\frac{y-y(t_0)}{y'(t_0)},
\]
si $x'(t_0)\neq 0$ e $y'(t_0)\neq 0$, y de ahí a la ecuación en la forma punto-pendiente
\[
y-y(t_0)=\frac{y'(t_0)}{x'(t_0)}(x-x(t_0)).
\]
\structure{\texbf{Ejemplo}}
Partiendo de la ecuación vectorial de la tangente del ejemplo anterior
$l=\left(\frac{\sqrt{2}}{2}-t\frac{\sqrt{2}}{2},\frac{\sqrt{2}}{2}+t\frac{\sqrt{2}}{2}\right)$, su ecuación cartesiana
es
\[
\frac{x-\sqrt{2}/2}{-\sqrt{2}/2} = \frac{y-\sqrt{2}/2}{\sqrt{2}/2}\Rightarrow y-\sqrt{2}/2 =
\frac{-\sqrt{2}/2}{\sqrt{2}/2}(x-\sqrt{2}/2) \Rightarrow y=-x+\sqrt{2}.
\]
\end{frame}


% ---------------------------------------------------------------------slide----
\begin{frame}
\frametitle{Recta normal a una trayectoria en el plano}
Se ha visto que la recta tangente a una trayectoria $f$ cuando $t=t_0$ es la recta que pasa por el punto el punto
$P=f(t_0)$ dirigida por el vector velocidad $\mathbf{v}=f'(0)=(x'(t_0),y'(t_0))$. Si en lugar de tomar ese vector se
toma como vector director el vector $\mathbf{w}=(y'(t_0),-x'(t_0))$, que es ortogonal a $\mathbf{v}$, se obtiene otra
recta que se conoce como \emph{recta normal} a la trayectoria $f$ cuanto $t=t_0$.
\begin{definicion}[Recta normal a una trayectoria]
Dada una trayectoria $f$ sobre el plano real $\mathbb{R}^2$, se llama \emph{recta normal} a $f$ para $t=t_0$ a la recta de ecuación 
\[
l = (x(t_0),y(t_0))+t(y'(t_0),-x'(t_0)) = (x(t_0)+ty'(t_0),y(t_0)-tx'(t_0)). 
\]
\end{definicion}
Su ecuación cartesiana es
\[
\frac{x-x(t_0)}{y'(t_0)} = \frac{y-y(t_0)}{-x'(t_0)},
\]
y su ecuación en la forma punto pendiente
\[
y-y(t_0) = \frac{-x'(t_0)}{y'(t_0)}(x-x(t_0)).
\]
La recta normal es perpendicular a la recta tangente ya que sus vectores directores son ortogonales. 
\end{frame}


% ---------------------------------------------------------------------slide----
\begin{frame}
\frametitle{Recta normal a una trayectoria en el plano}
\framesubtitle{Ejemplo}
Siguiendo con el ejemplo de la trayectoria $f(t) = (\cos t,\sen t)$, $t\in \mathbb{R}$, la recta normal en el instante $t=\pi/4$ es
\begin{align*}
l&= (\cos(\pi/2),\sen(\pi/2))+t(\cos(\pi/2),\sen(\pi/2)) =\\
& \left(\frac{\sqrt{2}}{2},\frac{\sqrt{2}}{2}\right)+t\left(\frac{\sqrt{2}}{2},\frac{\sqrt{2}}{2}\right)
=\left(\frac{\sqrt{2}}{2}+t\frac{\sqrt{2}}{2},\frac{\sqrt{2}}{2}+t\frac{\sqrt{2}}{2}\right),
\end{align*}
y su ecuación cartesiana es
\[
\frac{x-\sqrt{2}/2}{\sqrt{2}/2} = \frac{y-\sqrt{2}/2}{\sqrt{2}/2}\Rightarrow y-\sqrt{2}/2 = \frac{\sqrt{2}/2}{\sqrt{2}/2}(x-\sqrt{2}/2) \Rightarrow y=x.
\]
\begin{center}
\scalebox{0.8}{\psset{unit=1.2,algebraic}
\begin{pspicture*}(-4,-1.4)(4,1.5)
\psaxes[labelFontSize=\scriptstyle,ticksize=-3pt 0,labelsep=2pt]{<->}(0,0)(-1.4,-1.4)(1.4,1.4)[$x$,-90][$y$,180]
\pscircle[linecolor=blue](0,0){1}
\rput[l](-1,1){$f$}
\psplot[linecolor=red]{-2}{2}{-x+2^(1/2)}
\psplot[linecolor=red]{-1}{2}{x}
\psdot(0.7071,0.7071)
\rput[l](0.9,0.7){$t=\pi/4$}
\rput[l](1.6,1.35){$y=x$}
\rput[l](2.1,-0.5){$y=-x+\sqrt{2}$}
\end{pspicture*}}
\end{center}  
\end{frame}


% ---------------------------------------------------------------------slide----
\begin{frame}
\frametitle{Rectas tangente y normal a una función}
Un caso particular de las recta tangente y normal a una trayectoria es son la recta tangente y normal a una función de una variable real. 
Si se tiene la función $y=f(x)$, $x\in I\subseteq \mathbb{R}$, una trayectoria que traza la gráfica de $f$ es 
\[
g(t) = (t,f(t))  \quad t\in I,
\]
y su velocidad es
\[
g'(t) = (1,f'(t)),
\]
de modo que la recta tangente para $t=x_0$ es
\[
\frac{x-x_0}{1} = \frac{y-f(x_0)}{f'(x_0)} \Rightarrow y-f(x_0) = f'(x_0)(x-x_0),
\]  
y la recta normal es 
\[
\frac{x-x_0}{f'(x_0)} = \frac{y-f(x_0)}{-1} \Rightarrow y-f(x_0) = \frac{-1}{f'(x_0)}(x-x_0),
\]
% \begin{center}
% \scalebox{1}{\psset{unit=0.9,algebraic}
\begin{pspicture*}(-2.5,-0.5)(4.5,3.5)
\psaxes[ticks=none,labels=none]{<->}(0,0)(-0.3,-0.5)(2,3.3)[$x$,-90][$y$,180]
\psplot[linecolor=blue]{0.2}{1.5}{x^3-x^2+x+0.5}
\footnotesize
\rput[r](1.4,3.3){$f(x)$}
\psxTick[ticksize=-3pt 0,labelsep=3pt](1){a}
\psline[linewidth=0.5pt,linestyle=dashed,linecolor=gray](1,0)(1,1.5)(0,1.5)
\psyTick[ticksize=-3pt 0,labelsep=3pt](1.5){f(a)}
\psplot[linecolor=red]{0.2}{1.8}{2*x-0.5}
\psplot[linecolor=red]{0.2}{1.8}{2-0.5*x}
\rput[l](1.6,2.5){$y=f(a)+f'(a)(x-a)$}
\rput[l](1.6,0.8){$y=f(a)-\dfrac{1}{f'(a)}(x-a)$}
\end{pspicture*}}
% \end{center}
\end{frame}


% ---------------------------------------------------------------------slide----
\begin{frame}
\frametitle{Rectas tangente y normal a una función}
\framesubtitle{Ejemplo}
Dada la función $y=f(x)=x^2$, la trayectoria que dibuja la gráfica de esta función es $g(t)=(t,t^2)$ y su velocidad es
$g'(t)=(1,2t)$, de modo que en el punto $(1,1)$, que se alcanza en el instante $t=1$, la recta tangente es
\[
\frac{x-1}{1} = \frac{y-1}{2} \Rightarrow y-1 = 2(x-1) \Rightarrow y = 2x-1,
\]  
y la recta normal es 
\[
\frac{x-1}{2} = \frac{y-1}{-1} \Rightarrow y-1 = \frac{-1}{2}(x-1) \Rightarrow y = \frac{-x}{2}+\frac{3}{2}).
\]
\begin{center}
\scalebox{0.8}{\psset{unit=1.2,algebraic}
\begin{pspicture*}(-4,-0.5)(4,3.2)
\psaxes[labelFontSize=\scriptstyle,ticksize=-3pt 0,labelsep=2pt]{<->}(0,0)(-2,-0.5)(2,3)[$x$,-90][$y$,180]
\psplot[linecolor=blue]{-2}{2}{x^2}
\rput[l](-1.6,3){$f$}
\psplot[linecolor=red]{-2}{2}{2*x-1}
\psplot[linecolor=red]{-2}{2}{-x/2+3/2}
\psdot(1,1)
\rput[l](1.2,1.1){$t=1$}
\rput[l](2.1,3){$y=2x-1$}
\rput[l](2.1,0.5){$y=\frac{-x}{2}+\frac{3}{2}$}
\end{pspicture*}}
\end{center} 
\end{frame}


% ---------------------------------------------------------------------slide----
\begin{frame}
\frametitle{Recta tangente a una trayectoria en el espacio}
El concepto de recta tangente a una trayectoria en el plano real puede extenderse fácilmente a trayectorias en el espacio real $\mathbb{R}^3$.

Si $f(t)=(x(t),y(t),z(t))$, $t\in I\subseteq \mathbb{R}$, es una trayectoria en el espacio real $\mathbb{R}^3$, entonces
el móvil que recorre esta trayectoria en el instante $t=t_0$, ocupará la posición $P=(x(t_0),y(t_0),z(t_0))$ y tendrá una velocidad $\mathbf{v}=f'(t)=(x'(t),y'(t),z'(t))$, de manera que la recta tangente a $f$ en ese instante será
\begin{align*}
l&=(x(t_0),y(t_0),z(t_0))+t(x'(t_0),y'(t_0),z'(t_0)) =\\
&= (x(t_0)+tx'(t_0),y(t_0)+ty'(t_0),z(t_0)+tz'(t_0)),
\end{align*}
cuyas ecuaciones cartesianas son 
\[
\frac{x-x(t_0)}{x'(t_0)}=\frac{y-y(t_0)}{y'(t_0)}=\frac{z-z(t_0)}{z'(t_0)},
\]
siempre que $x'(t_0)\neq 0$, $y'(t_0)\neq 0$ y $z'(t_0)\neq 0$.
\end{frame}


% ---------------------------------------------------------------------slide----
\begin{frame}
\frametitle{Recta tangente a una trayectoria en el espacio}
\framesubtitle{Ejemplo}
Dada la trayectoria del espacio $f(t)=(\cos t, \sen t, t)$, $t\in \mathbb{R}$, en el instante $t=\pi/2$, la trayectoria
pasará por el punto
\[
f(\pi/2)=(\cos(\pi/2),\sen(\pi/2),\pi/2)=(0,1,\pi/2),
\] con una velocidad 
\[
\mathbf{v}=f'(\pi/2)=(-\sen(\pi/2),\cos(\pi/2), 1)=(-1,0,1),
\] y la
tangente en ese punto es
\[
l=(0,1,\pi/2)+t(-1,0,1) = (-t,1,t+\pi/2)
\] 
\begin{center}
\scalebox{0.8}{\psset{Alpha=60,Beta=30}
\begin{pspicture}(-3,-2)(4,3.2)
\pstThreeDCoor[IIIDticks,xMax=2,yMax=2,zMax=3,linecolor=black]
\parametricplotThreeD[xPlotpoints=200, linewidth=1.5pt, linecolor=blue, plotstyle=curve, algebraic](0,3.14){cos(t) | sin(t) | t}
\parametricplotThreeD[xPlotpoints=10, linecolor=red, algebraic](-1.5,1.7){-t | 1 | t+1.57}
\pstThreeDDot[drawCoor=true](0,1,1.57)
\pstThreeDLine[arrows=->,linecolor=green](0,1,1.57)(-1,1,2.57)
\pstThreeDPut[pOrigin=l](0,1.1,1.57){$P=(0,1,\pi/2)$}
\pstThreeDPut[pOrigin=l](-1,1.1,2.57){$\mathbf{v}$}
\pstThreeDPut[pOrigin=l](-1.7,1.1,3.27){$l$}
\end{pspicture}
}
\end{center} 
\end{frame}


\subsection{Estudio del crecimiento de una función}
%---------------------------------------------------------------------slide----
\begin{frame}
\frametitle{Estudio del crecimiento de una función}
La principal aplicación de la derivada es el estudio del crecimiento de una función mediante el signo de la derivada.
\begin{teorema}
Si $f$ es una función cuya derivada existe en un intervalo $I$, entonces:
\begin{itemize}
\item Si $\forall x\in I\ f'(x)\geq 0$ entonces $f$ es creciente en el intervalo $I$.
\item Si $\forall x\in I\ f'(x)\leq 0$ entonces $f$ es decreciente en el intervalo $I$.
\end{itemize}
\end{teorema}
\structure{\textbf{Ejemplo}} 
La función $f(x)=x^3$ es creciente en todo $\mathbb{R}$ ya que $\forall x\in \mathbb{R}\
f'(x)\geq 0$. \vskip .5cm
\textbf{Observación}. \emph{Una función puede ser creciente o decreciente en un intervalo y no tener derivada.}
\end{frame}


%---------------------------------------------------------------------slide----
\begin{frame}
\frametitle{Estudio del crecimiento de una función}
\framesubtitle{Ejemplo}
Consideremos la función $f(x)=x^4-2x^2+1$. Su derivada $f'(x)=4x^3-4x$ está definida en todo $\mathbb{R}$ y es continua.
\begin{center}
\scalebox{1}{\psset{unit=0.95,algebraic}
\begin{pspicture*}(-5,-4.5)(5,2.5)
\psaxes[labelFontSize=\scriptstyle,ticksize=-3pt 0,labelsep=2pt]{<->}(0,0)(-2.5,-2.1)(2.5,2.1)
\psplot[linecolor=blue]{-1.5}{1.5}{x^4-2*x^2+1}
\psplot[linecolor=red]{-1.2}{1.2}{4*x^3-4*x}
\footnotesize
\rput[r](-1.5,2){\textcolor{blue}{$f(x)=x^4-2x^2+1$}}
\rput[r](-1.5,-1.5){\textcolor{red}{$f'(x)=4x^3-4x$}}
\normalsize
\uncover<2->{
\psaxes[labelFontSize=\scriptstyle,ticksize=-3pt 0,labelsep=2pt]{<->}(0,-3)(-2.5,-3)(2.5,-3)
\footnotesize
\rput[r](-2,-2.5){Crecimiento $f(x)$}
\rput[r](-2,-3.7){Signo $f'(x)$}
}
\uncover<3->{
\rput[c](-1,-3.7){\textcolor{red}{$0$}}
\psline[linecolor=gray,linestyle=dashed]{-}(-1,-3.5)(-1,2)
\rput[c](0,-3.7){\textcolor{red}{$0$}}
\psline[linecolor=gray,linestyle=dashed]{-}(0,-3.5)(0,2)
\rput[c](1,-3.7){\textcolor{red}{$0$}}
\psline[linecolor=gray,linestyle=dashed]{-}(1,-3.5)(1,2)
}
\uncover<4->{
\rput[c](-1.5,-3.7){\textcolor{red}{$-$}}
\rput[c](-1.5,-2.5){\textcolor{blue}{$\downarrow$}}
}
\uncover<5->{
\rput[c](-0.5,-3.7){\textcolor{red}{$+$}}
\rput[c](-0.5,-2.5){\textcolor{blue}{$\uparrow$}}
}
\uncover<6->{
\rput[c](0.5,-3.7){\textcolor{red}{$-$}}
\rput[c](0.5,-2.5){\textcolor{blue}{$\downarrow$}}
}
\uncover<7->{
\rput[c](1.5,-3.7){\textcolor{red}{$+$}}
\rput[c](1.5,-2.5){\textcolor{blue}{$\uparrow$}}
}
\end{pspicture*}}
\end{center}
\end{frame}


\subsection{Determinación de los extremos relativos de una función}
%---------------------------------------------------------------------slide----
\begin{frame}
\frametitle{Determinación de extremos relativos de una función}
Como consecuencia del resultado anterior, la derivada también sirve para determinar los extremos relativos de una función.
\begin{teorema}[Criterio de la primera derivada]
Sea $f$ es una función cuya derivada existe en un intervalo $I$, y sea $x_0\in I$ tal que $f'(x_0)=0$, entonces:
\begin{itemize}
\item Si existe un $\delta>0$ tal que $\forall x\in(x_0-\delta,x_0)\ f'(x)>0$ y $\forall x\in(x_0,x_0+\delta)\ f'(x)<0$ entonces $f$ tiene un \emph{máximo relativo} en $x_0$.
\item Si existe un $\delta>0$ tal que $\forall x\in(x_0-\delta,x_0)\ f'(x)<0$ y $\forall x\in(x_0,x_0+\delta)\ f'(x)>0$ entonces $f$ tiene un \emph{mínimo relativo} en $x_0$.
\item Si existe un $\delta>0$ tal que $\forall x\in(x_0-\delta,x_0)\ f'(x)>0$ y $\forall x\in(x_0,x_0+\delta)\ f'(x)>0$ entonces $f$ tiene un \emph{punto de inflexión creciente} en $x_0$.
\item Si existe un $\delta>0$ tal que $\forall x\in(x_0-\delta,x_0)\ f'(x)<0$ y $\forall x\in(x_0,x_0+\delta)\ f'(x)<0$ entonces $f$ tiene un \emph{punto de inflexión decreciente} en $x_0$.
\end{itemize}
\end{teorema}
Los puntos donde se anula la derivada de una función se denominan \emph{puntos críticos}.

%\textbf{Observación}. \emph{La anulación de la derivada es una condición necesaria pero no suficiente para la existencia de un extremo relativo en un punto.}

%\structure{\textbf{Ejemplo}} La función $f(x)=x^3$ tiene derivada $f'(x)=3x^2$ y por tanto tiene un punto crítico en $x=0$, pero no tiene un extremo relativo en el 0, sino un punto de inflexión.
\end{frame}


%---------------------------------------------------------------------slide----
\begin{frame}
\frametitle{Determinación de extremos relativos de una función}
\framesubtitle{Ejemplo}
Consideremos de nuevo la función $f(x)=x^4-2x^2+1$. 
Su derivada $f'(x)=4x^3-4x$ está definida en todo $\mathbb{R}$ y es continua.
\begin{center}
\scalebox{1}{\psset{unit=0.95,algebraic}
\begin{pspicture*}(-5,-4.5)(5,2.5)
\psaxes[labelFontSize=\scriptstyle,ticksize=-3pt 0,labelsep=2pt]{<->}(0,0)(-2.5,-2.1)(2.5,2.1)
\psplot[linecolor=blue]{-1.5}{1.5}{x^4-2*x^2+1}
\psplot[linecolor=red]{-1.2}{1.2}{4*x^3-4*x}
\footnotesize
\rput[r](-1.5,2){\textcolor{blue}{$f(x)=x^4-2x^2+1$}}
\rput[r](-1.5,-1.5){\textcolor{red}{$f'(x)=4x^3-4x$}}
\normalsize
\psaxes[labelFontSize=\scriptstyle,ticksize=-3pt 0,labelsep=2pt]{<->}(0,-3)(-2.5,-3)(2.5,-3)
\psline[linecolor=gray,linestyle=dashed]{<-}(-1,-4)(-1,2)
\psline[linecolor=gray,linestyle=dashed]{<-}(0,-4)(0,2)
\psline[linecolor=gray,linestyle=dashed]{<-}(1,-4)(1,2)
\footnotesize
\rput[r](-2,-2.5){Crecimiento $f(x)$}
\rput[r](-2,-3.7){Signo $f'(x)$}
\rput[c](-1.5,-2.5){\textcolor{blue}{$\downarrow$}}
\rput[c](-1.5,-3.7){\textcolor{red}{$-$}}
\rput[c](-1,-3.7){\textcolor{red}{$0$}}
\rput[c](-0.5,-2.5){\textcolor{blue}{$\uparrow$}}
\rput[c](-0.5,-3.7){\textcolor{red}{$+$}}
\rput[c](0,-3.7){\textcolor{red}{$0$}}
\rput[c](0.5,-2.5){\textcolor{blue}{$\downarrow$}}
\rput[c](0.5,-3.7){\textcolor{red}{$-$}}
\rput[c](1,-3.7){\textcolor{red}{$0$}}
\rput[c](1.5,-2.5){\textcolor{blue}{$\uparrow$}}
\rput[c](1.5,-3.7){\textcolor{red}{$+$}}
\uncover<2->{
\rput[r](-2,-4.3){Extremos $f(x)$}
\rput[c](-1,-4.3){\textcolor{blue}{Mín}}
\rput[c](0,-4.3){\textcolor{blue}{Máx}}
\rput[c](1,-4.3){\textcolor{blue}{Mín}}
}
\end{pspicture*}}
\end{center}
\end{frame}



\subsection{Estudio de la concavidad de una función}
%---------------------------------------------------------------------slide---
\begin{frame}
\frametitle{Estudio de la concavidad de una función}
La concavidad de una función puede estudiarse mediante el signo de la segunda derivada.
\begin{teorema}[Criterio de la segunda derivada]
Si $f$ es una función cuya segunda derivada existe en un intervalo $I$, entonces:
\begin{itemize}
\item Si $\forall x\in I\ f''(x)\geq 0$ entonces $f$ es cóncava en el intervalo $I$.
\item Si $\forall x\in I\ f''(x)\leq 0$ entonces $f$ es convexa en el intervalo $I$.
\end{itemize}
\end{teorema}

\structure{\textbf{Ejemplo}} La función $f(x)=x^2$ tiene segunda derivada $f''(x)=2>0$ y por tanto es cóncava en todo $\mathbb{R}$.
\vskip .5cm
\textbf{Observación}. \emph{Una función puede ser cóncava o convexa en un intervalo y no tener derivada.}
\end{frame}


%---------------------------------------------------------------------slide----
\begin{frame}
\frametitle{Estudio de la concavidad de una función}
\framesubtitle{Ejemplo}
Consideremos de nuevo la función $f(x)=x^4-2x^2+1$. Su segunda derivada $f''(x)=12x^2-4$ está definida en todo $\mathbb{R}$ y es continua.
\begin{center}
\scalebox{1}{\psset{unit=0.8,algebraic}
\begin{pspicture*}(-5,-6)(5,2.1)
\psaxes[labelFontSize=\scriptstyle,ticksize=-3pt 0,labelsep=2pt]{<->}(0,0)(-2.5,-4.1)(2.5,2.1)
\psplot[linecolor=blue]{-1.5}{1.5}{x^4-2*x^2+1}
\psplot[linecolor=red]{-1.2}{1.2}{4*x^3-4*x}
\psplot[linecolor=green]{-1.2}{1.2}{12*x^2-4}
\footnotesize
\rput[r](-1.5,1.9){\textcolor{blue}{$f(x)=x^4-2x^2+1$}}
\rput[r](-1.5,-1.5){\textcolor{red}{$f'(x)=4x^3-4x$}}
\rput[l](1,-2){\textcolor{green}{$f''(x)=12x^2-4$}}
\uncover<2->{
\psaxes[ticksize=-3pt 0,labelsep=2pt]{<->}(0,-5)(-2.5,-5)(2.5,-5)
\rput[r](-2,-4.5){Concavidad $f(x)$}
\rput[r](-2,-5.8){Signo $f''(x)$}
}
\uncover<3->{
\rput[c](-0.5773,-5.8){\textcolor{green}{$0$}}
\psline[linecolor=gray,linestyle=dashed](-0.5773,-5.5)(-0.5773,2)
\rput[c](0.5773,-5.8){\textcolor{green}{$0$}}
\psline[linecolor=gray,linestyle=dashed](0.5773,-5.5)(0.5773,2)
}
\uncover<4->{
\rput[c](-1.2,-5.8){\textcolor{green}{$+$}}
\rput[c](-1.2,-4.5){\textcolor{blue}{$\cup$}}
}
\uncover<5->{
\rput[c](0,-5.8){\textcolor{green}{$-$}}
\rput[c](0,-4.5){\textcolor{blue}{$\cap$}}
}
\uncover<6->{
\rput[c](1.2,-5.8){\textcolor{green}{$+$}}
\rput[c](1.2,-4.5){\textcolor{blue}{$\cup$}}
}
\end{pspicture*}}
\end{center}
\end{frame}


% 
% \subsection{El concepto de diferencial}
% %---------------------------------------------------------------------slide----
% \begin{frame}
% \frametitle{El concepto de diferencial}
% \begin{definicion}[Diferencial de una función en un punto]
% Dada una función $f$, se llama \emph{diferencial} de $f$ en un punto $a$, al la función
% \[
% \begin{array}{rccc}
% dy=df(a): & \mathbb{R} & \longrightarrow & \mathbb{R} \\
% & \Delta x & \longrightarrow & f'(a)\Delta x
% \end{array}
% \]
% \end{definicion}
% 
% Cuando $f$ es la función identidad $y=x$,  entonces $f'(a)=1$, y se cumple que
% \[ dx=dy=f'(a)\Delta x=\Delta x,\]
% de modo que también podemos definir el diferencial como 
% \[dy=df(a)=f'(a)dx.\]
% 
% De aquí se deduce otra forma de escribir la derivada de $f$ en $a$
% \[f'(a)=\frac{dy}{dx}=\frac{df(a)}{dx}.\]
% \end{frame}
% 
% 
% %---------------------------------------------------------------------slide----
% \begin{frame}
% \frametitle{Aproximación de una función mediante su diferencial}
% El diferencial de una función $f$ en un punto $a$, permite aproximar la variación de $f$ cerca de $a$. 
% 
% \begin{center}
% \scalebox{1}{\psset{xunit=2,yunit=1.2,algebraic}
\begin{pspicture*}(-1.2,-0.5)(4.3,4)
\psaxes[ticks=none,labels=none]{<->}(0,0)(-0.3,-0.5)(2,4)
\psplot[linecolor=blue]{0.2}{1.6}{x^3-x^2+x+0.5}
\rput[r](1.4,3.5){$f(x)$}
\psxTick[ticksize=-3pt 0,labelsep=3pt](0.5){a}
\psyTick[ticksize=-3pt 0,labelsep=3pt](0.875){f(a)}
\psline[linewidth=0.5pt,linestyle=dashed,linecolor=gray](0.5,0)(0.5,0.875)(0,0.875)
\uncover<2->{
\psxTick[ticksize=-3pt 0,labelsep=3pt](1.5){a+dx}
\psline[linewidth=0.5pt,linestyle=dashed,linecolor=gray](1.5,0)(1.5,3.125)(0,3.125)
\psyTick[ticksize=-3pt 0,labelsep=3pt](3.125){f(a+dx)}
\psline[arrows=|*-|*,linecolor=green](0.5,0.875)(1.5,0.875)
\rput[t](1,0.8){$dx$}
}
\uncover<3->{
\psline[linestyle=dotted,linecolor=gray](1.5,0.875)(2.6,0.875)
\psline[linestyle=dotted,linecolor=gray](1.5,3.125)(2.6,3.125)
\psline[arrows=|*-|*,linecolor=orange](2.6,0.875)(2.6,3.125)
\rput[l](2.7,1.9){$\Delta y=f(a+dx)-f(a)$}
}
\uncover<4->{
\psplot[linecolor=red]{0.2}{1.8}{0.75*x+0.5}
}
\uncover<5->{
\psline[linewidth=0.5pt,linestyle=dashed,linecolor=gray](1.5,1.625)(0,1.625)
\rput[r](-0.1,1.625){$f(a)+f'(a)dx$}
}
\uncover<6->{
\psline[arrows=|*-|*,linecolor=green](1.5,1.625)(1.5,0.875)
\rput[l](1.6,1.25){$dy=f'(a)dx$}
\rput[r](-0.4,2.4){\rotatebox{90}{$\approx$}}
}
\end{pspicture*}}
% \end{center}
% \end{frame}
% 
% 
% %---------------------------------------------------------------------slide----
% \begin{frame}
% \frametitle{Aproximación de una función mediante su diferencial: Ejemplo}
% Consideremos otra vez la función $y=x^2$ que mide el área de un cuadrado de chapa metálica de lado $x$.
% 
% Si el lado del cuadrado es $a$, y sometemos la chapa a un proceso de calentamiento que aumenta el lado del cuadrado, ¿cuál  será aproximadamente la variación que habrá experimentado el área, cuando el lado aumente una cantidad $dx$?
% \begin{columns}
% \begin{column}{0.3\textwidth} 
% \begin{align*}
% \Delta y &= f(a+dx)-f(a)=(a+dx)^2-a^2=\\
% &= a^2+2adx+dx^2-a^2=2adx+dx^2,\\
% \only<2->{dy &= f'(a)dx= 2adx.}
% \end{align*}
% \uncover<3->{Además, \[\lim_{dx\rightarrow 0}\Delta y-dy=\lim_{dx\rightarrow 0}dx^2=0.\]}
% \end{column}
% \begin{column}{0.3\textwidth}
% \begin{center}
% \scalebox{1}{\psset{unit=0.6}
\begin{pspicture*}(0,-0.5)(4,4)
\footnotesize
\psframe(0,0)(3,3)
\rput[t](1.5,-0.1){$a$}
\rput[t](1.5,1.7){$a^2$}
\psframe[fillstyle=solid,fillcolor=gray](0,3)(3,4)
\psframe[fillstyle=solid,fillcolor=gray](3,3)(4,4)
\psframe[fillstyle=solid,fillcolor=gray](3,0)(4,3)
\rput[t](3.5,-0.1){$dx$}
\rput[t](1.5,3.7){$adx$}
\rput[t](3.5,3.7){$dx^2$}
\rput[t](3.5,1.7){$adx$}
\uncover<2->{
\psframe[fillstyle=solid,fillcolor=orange](0,3)(3,4)
\psframe[fillstyle=solid,fillcolor=orange](3,0)(4,3)
\rput[t](1.5,3.7){$adx$}
\rput[t](3.5,1.7){$adx$}
}
\end{pspicture*}}
% \end{center}
% \end{column}
% \end{columns}
% \end{frame}


% %---------------------------------------------------------------------slide----
% \begin{frame}
% \frametitle{Propiedades del diferencial}
% Si $y=c$, es una función constante, entonces $dy=0$.
% Si $y=x$, es la función identidad, entonces  $dy=dx$.
% 
% Si $u=f(x)$ y $v=g(x)$ son dos funciones diferenciables, entonces
% \begin{itemize}
% \item $d(u+v)=d(u)+d(v)$
% \item $d(u-v)=d(u)-d(v)$
% \item $d(u\cdot v)=d(u)\cdot v+ u\cdot d(v)$
% \item $d\left(\dfrac{u}{v}\right)=\dfrac{du\cdot v-u\cdot dv}{v^2}$
% \end{itemize}
% \end{frame}





% \subsection{Derivada de una función implícita}
% %---------------------------------------------------------------------slide----
% \begin{frame}
% \frametitle{Derivada de una función implícita}
% Si $F(x,y)=0$ es una función implícita entonces 
% \[
% dF(x,y)=d0=0.
% \]
% 
% Si $F(x,y)=0$ es una función implícita en la que $y$ depende de $x$, entonces podemos calcular la derivada de $y$ a partir del diferencial
% \[
% \frac{dF(x,y)}{dx}=\frac{d0}{dx}=0.
% \]
% \structure{\textbf{Ejemplo}}. Consideremos la función implícita de la circunferencia de radio 1, $x^2+y^2=1$. Entonces su diferencial es
% \[
% d(x^2+y^2)=d1=0 \Leftrightarrow d(x^2)+d(y^2)=2x\;dx+2y\;dy=0.
% \]
% A partir de aquí podemos calcular fácilmente la derivada de $y$:
% \[
% \frac{d(x^2+y^2)}{dx}= \frac{2x\;dx+2y\;dy}{dx}=2x\frac{dx}{dx}+2y\frac{dy}{dx}= 2x+2y\frac{dy}{dx}=0 \Leftrightarrow \frac{dy}{dx}=\frac{-x}{y}.
% \]
% 
% \end{frame}
% 
% 
% \subsection{Derivada de una función paramétrica}
% %---------------------------------------------------------------------slide----
% \begin{frame}
% \frametitle{Derivada de una función parametrica}
% 
% Dada una función paramétrica
% \[
% \left\{%
% \begin{array}{l}
% x=f(t) \\
% y=g(t) 
% \end{array}%
% \right.  
% \]
% podemos calcular su derivada a partir de las derivadas de $f$ y $g$:
% \[\frac{dy}{dx}=\frac{g'(t)\,dt}{f'(t)\,dt}=\frac{g'(t)}{f'(t)}.\]
% 
% \structure{\textbf{Ejemplo}}. Consideremos la elipse
% \[
% \left\{%
% \begin{array}{l}
% x=2\sen t \\
% y=\cos t
% \end{array}%
% \right. 
% \]
% Entonces 
% \[
% \frac{dy}{dx}=\frac{-\sen t\; dt}{2 \cos t\; dt}=\frac{-1}{2}\tg t.
% \]
% \end{frame}
