%Version control information
%$HeadURL: https://ejercicioscalculo.googlecode.com/svn/trunk/compendio_ejercicios_calculo.tex $} {$LastChangedDate: 2008-07-09 16:26:20 +0200 (mi�, 09 jul 2008) $
%$LastChangedRevision: 6 $
%$LastChangedBy: asalber $

\section{Funciones Elementales}

%---------------------------------------------------------------------slide----
\begin{frame}
\frametitle{Funciones Elementales}
\setlength{\parskip}{0.3em}
\tableofcontents[sectionstyle=show/hide,hideothersubsections]
\end{frame}

\subsection{El concepto de función}

%---------------------------------------------------------------------slide----
\begin{frame}
\frametitle{¿Qué es una función?}
\begin{definicion}[Función de una variable]
Una \emph{función} $f$ de un conjunto $A$ en otro $B$ es una \emph{relación} que asocia cada elemento $a\in A$, con un \emph{único} elemento de $B$ que se denota $f(a)$, y se llama \emph{imagen} de $a$ mediante $f$.
\begin{align*}
f:\,&A\longrightarrow B\\
&a\longrightarrow f(a)
\end{align*}
\end{definicion}

\begin{center}
\scalebox{1}{\psset{unit=0.5}
\begin{pspicture*}(0,0)(10,4)
\psellipse(2,2)(1.5,2)
\psellipse(8,2)(1.5,2)
\rput(2.2,3){3}
\rput(1.8,2){4}
\rput(2.1,1){5}
\rput(7.6,1.7){\pspolygon*(0,0)(0.5,0.7)(1,0)}
\rput(7.5,2.7){\psframe*(0.6,0.6)}
\rput(7.6,0.7){\pspolygon*(0.2,0)(0,0.4)(0.5,0.6)(1,0.4)(0.8,0)}
\psline[arrows=->](2.6,3)(7.3,2)
\psline[arrows=->](2.6,2)(7.3,3)
\psline[arrows=->](2.6,1)(7.3,1)
\end{pspicture*}}
\end{center}
Cuando el conjunto origen y destino es el de los números reales $\mathbb{R}$, entonces se dice que  $f:\mathbb{R}\rightarrow \mathbb{R}$ es una \emph{función real de variable real}.
\end{frame}


%---------------------------------------------------------------------slide----
\begin{frame}
\frametitle{Formas de representar una función}
\begin{columns}
\begin{column}{.48\textwidth}
\textbf{Por extensión}
\begin{block}{Representación en forma de tabla}
\[
\begin{array}{|c|c|c|c|c|c|c|}
\hline
 x & -2 & -1 & 0 & 1 & 2 & \cdots \\
\hline
 y & 4  & 1  & 0 & 1 & 4 & \cdots \\
\hline
\end{array}
\]
\end{block}
\begin{block}{Representación gráfica}
\begin{center}
\scalebox{1}{\psset{unit=1,algebraic}
\begin{pspicture}(-2,-1)(2,4.5)
\psaxes[labelFontSize=\scriptstyle,ticksize=-3pt 0,labelsep=2pt]{<->}(0,0)(-2.5,-1)(2.5,4.5)
\psplot[linecolor=blue]{-2}{2}{x^2}
\psline[linecolor=red,linestyle=dotted](1.5,-0.1)(1.5,2.25)(-0.1,2.25)
\psxTick[ticksize=-3pt 0,labelsep=3pt](1.5){\color{red}x}
\psyTick[ticksize=-3pt 0,labelsep=3pt](2.25){\color{red}y}
\end{pspicture}}
\end{center}
\end{block}
\end{column}
\begin{column}{.48\textwidth}
\textbf{Por Intensión}
\begin{block}{Representación algebraica explícita}
\[y=x^2\]
\end{block}
\begin{block}{Representación algebraica implícita}
\[y-x^2=0\]
\end{block}
\begin{block}{Representación algebraica paramétrica}
\[  
\begin{cases}
y=t^2\\
x=t
\end{cases}
\]
\end{block}
\end{column}   
\end{columns}
\end{frame} 


%---------------------------------------------------------------------slide----
\begin{frame}
\frametitle{La función Identidad}
\begin{definicion}[Función Identidad]
Se llama \emph{función identidad}, a toda función $Id: A\rightarrow A$ que asocia cada elemento de $A$ con sigo mismo, es decir, 
\[Id(x)=x.\]
\end{definicion}
\begin{center}
\scalebox{1}{\psset{unit=0.8,algebraic}
\begin{pspicture*}(-3,-3)(3,3)
\psaxes[labelFontSize=\scriptstyle,ticksize=-3pt 0,labelsep=2pt]{<->}(0,0)(-3,-3)(3,3)
\psplot[linecolor=blue]{-2}{2}{x}
\rput[l](1,2.5){$f(x)=x$}
\end{pspicture*}}
\end{center}
\end{frame} 



\subsection{Dominio e imagen de una función}
%---------------------------------------------------------------------slide----
\begin{frame}
\frametitle{Dominio de una función}
\begin{definicion}[Dominio de una función]
El \emph{dominio} de una función $f$ es el conjunto de valores para los que la función está definida
\[
Dom(f)=\{x\in \mathbb{R}: f(x)\in \mathbb{R}\}
\]
\end{definicion}
\structure{\bfseries Ejemplo}
\begin{center}
\scalebox{1}{\psset{unit=0.8,algebraic}
\begin{pspicture*}(-5,-0.5)(5,4)
\psaxes[labelFontSize=\scriptstyle,ticksize=-3pt 0,labelsep=2pt]{<->}(0,0)(-3.3,-0.5)(3.3,4)
\psplot[linecolor=blue]{-3}{-1.00001}{1/sqrt(x^2-1)}
\psplot[linecolor=blue]{1.00001}{3}{1/sqrt(x^2-1)}
\rput[l](1.3,2){$f(x)=\dfrac{1}{\sqrt{x^2-1}}$}
\uncover<2->{
\psline[linecolor=red,linewidth=1pt,tbarsize=7pt 0]{<-)}(-3.3,0)(-1,0)
\psline[linecolor=red,linewidth=1pt,tbarsize=7pt 0]{(->}(1,0)(3.3,0)
}
\end{pspicture*}}\\
$\mbox{Dom}(f)=(-\infty,-1)\cup(1,\infty)$
\end{center}
\end{frame} 


%---------------------------------------------------------------------slide----
\begin{frame}
\frametitle{Imagen de una función}
\begin{definicion}[Imagen de una función]
La \emph{imagen} de una función $f$ es el conjunto de valores que la función puede tomar
\[
Img(f)=\{y\in \mathbb{R}: y=f(x) \mbox{ para algún } x\in\mathbb{R}\}
\]
\end{definicion}
\structure{\bfseries Ejemplo}
\begin{center}
\scalebox{1}{\psset{unit=0.7,algebraic}
\begin{pspicture*}(-5,-3)(5,3)
\psaxes[labelFontSize=\scriptstyle,ticksize=-3pt 0,labelsep=2pt]{<->}(0,0)(-3,-3)(3,3)
\psplot[linecolor=blue]{-2}{2}{x^2-2}
\normalsize
\rput[l](1,2.5){$f(x)=x^2-2$}
\uncover<2->{
\psline[linecolor=red,linewidth=1pt,tbarsize=7pt 0,arrows=[->](0,-2)(0,3)
}
\end{pspicture*}}\\
$\mbox{Img}(f)=(-2,\infty)$
\end{center}
\end{frame} 



\subsection{Composición e inversa de una función}
%---------------------------------------------------------------------slide----
\begin{frame}
\frametitle{Composición de funciones}
\begin{definicion}[Composición de funciones]
Dadas dos funciones $g:A\rightarrow B$ y $f:B\rightarrow C$, se define la \emph{función compuesta} $f\circ g$, (leído $g$ compuesto con $f$) como la función  
\begin{align*}
f\circ g:\,& A\longrightarrow C\\
& x\longrightarrow f(g(x))
\end{align*}
\end{definicion}
Para calcular la función compuesta $f\circ g(x)$, primero se aplica $g$ sobre $x$ y luego, se aplica $f$ sobre $g(x)$:
\[
x\stackrel{g}{\longrightarrow}g(x)\stackrel{f}{\longrightarrow}f(g(x))
\]
\structure{\textbf{Ejemplo}} Si $g(x)=\sqrt x$ y $f(x)=\sen x$, entonces 
\[
f\circ g(x)=f(g(x))=f(\sqrt x)=\sen \sqrt x.
\]
\begin{center}
\emph{¿Cuál es su dominio?}
\end{center}
\end{frame} 


%---------------------------------------------------------------------slide----
\begin{frame}
\frametitle{Inversa de una función}
\begin{definicion}[Función inversa]
Se llama \emph{función inversa} de $f:A\rightarrow B$ a la función $f^{-1}:B\rightarrow A$ que cumple
\[
f\circ f^{-1}=f^{-1}\circ f=Id(x)
\]
\end{definicion}

La función inversa de $f$ deshace o revierte el efecto de $f$. 
Es decir, si $f:A\rightarrow B$ asocia un elemento $x\in A$ con otro $y\in B$,
entonces $f^{-1}$ asocia el elemento $y$ con el $x$: 
\begin{center}
\scalebox{1}{\begin{pspicture*}(-5,-0.7)(5,0.7)
\rput[l](1,0){$y$}
\rput[r](-1,0){$x$}
\pscurve{->}(-0.8,0.1)(0,0.2)(0.8,0.1)
\rput[b](0,0.3){$f$}
\pscurve{<-}(-0.8,-0.1)(0,-0.2)(0.8,-0.1)
\rput[t](0,-0.3){$f^{-1}$}
\end{pspicture*}}
\end{center}

\alert{¡Ojo! } $f^{-1}\neq \dfrac{1}{f}$. Para que exista $f^{-1}$ se requiere
que $f$ sea \emph{inyectiva}.

\structure{\textbf{Ejemplo}}
\begin{itemize}
\item[--] La inversa de $f(x)=x^3$ es la función $f^{-1}(x)=\sqrt[3]{x}.$ 
\item[--] La inversa de la función $f(x)=\sen x$ es la función
$f^{-1}(x)=\arcsen x.$
\item[--] La función $x^2$ no tienen inversa pues no es inyectiva.
\end{itemize}
\end{frame} 



\subsection{Crecimiento de una función}
%---------------------------------------------------------------------slide----
\begin{frame}
\frametitle{Crecimiento de una función}
\begin{definicion}[Función creciente y decreciente]
Se dice que una función $f$ es \emph{creciente} en un intervalo $I$, si para todo $x_1,x_2\in I$, con $x_1<x_2$, se cumple $f(x_1)\leq f(x_2)$.

Se dice que una función $f$ es \emph{decreciente} en un intervalo $I$, si para todo $x_1,x_2\in I$, con $x_1<x_2$, se cumple $f(x_1)\geq f(x_2)$.
\end{definicion}

\begin{columns}
\begin{column}{0.4\textwidth}
\begin{center}
\scalebox{1}{\psset{unit=1.2,algebraic}
\begin{pspicture*}(-1,-0.5)(3,2.5)
\psaxes[ticks=none,labels=none]{<->}(0,0)(-0.5,-0.5)(2.5,2.5)
\psplot[linecolor=blue]{0.5}{1.7}{x^1.5}
\psline[linecolor=red,linestyle=dashed](1,0)(1,1)(0,1)
\psline[linecolor=red,linestyle=dashed](1.5,0)(1.5,1.8371)(0,1.8371)
\psxTick[ticksize=-3pt 0,labelsep=3pt](1){x_1}
\psxTick[ticksize=0,labelsep=3pt](1.25){<}
\psxTick[ticksize=-3pt 0,labelsep=3pt](1.5){x_2}
\psyTick[ticksize=-3pt 0,labelsep=3pt](1){f(x_1)}
\psyTick[ticksize=0,labelsep=10pt](1.415){\rotateleft{\leq}}
\psyTick[ticksize=-3pt 0,labelsep=3pt](1.8371){f(x_2)}
\end{pspicture*}}\\
\color{red}Función creciente
\end{center}
\end{column}
\begin{column}{0.4\textwidth}
\begin{center}
\scalebox{1}{\psset{unit=1.2,algebraic}
\begin{pspicture*}(-1,-0.5)(3,2.5)
\psaxes[ticks=none,labels=none]{<->}(0,0)(-0.5,-0.5)(2.5,2.5)
\psplot[linecolor=blue]{0.5}{1.7}{2.5-x^1.5}
\psline[linecolor=red,linestyle=dashed](0.8,0)(0.8,1.7845)(0,1.7845)
\psline[linecolor=red,linestyle=dashed](1.3,0)(1.3,1.0178)(0,1.0178)
\psxTick[ticksize=-3pt 0,labelsep=3pt](0.8){x_1}
\psxTick[ticksize=0,labelsep=3pt](1.05){<}
\psxTick[ticksize=-3pt 0,labelsep=3pt](1.3){x_2}
\psyTick[ticksize=-3pt 0,labelsep=3pt](1.7845){f(x_1)}
\psyTick[ticksize=0,labelsep=10pt](1.415){\rotateleft{\leq}}
\psyTick[ticksize=-3pt 0,labelsep=3pt](1.0178){f(x_2)}
\end{pspicture*}}\\
\color{red}Función decreciente
\end{center}
\end{column}
\end{columns}
\end{frame} 



\subsection{Extremos de una función}
%---------------------------------------------------------------------slide----
\begin{frame}
\frametitle{Extremos de una función}
\begin{definicion}[Máximo y mínimo relativo]
Se dice que una función $f$ tiene un máximo relativo en $x_0$, si existe un $\delta>0$ tal que para todo $x\in (x_0-\delta,x_0+\delta)$ se cumple $f(x_0)\geq f(x)$.

Se dice que una función $f$ tiene un mínimo relativo en $x_0$, si existe un $\delta>0$ tal que para todo $x\in (x_0-\delta,x_0+\delta)$ se cumple $f(x_0)\leq f(x)$.
\end{definicion}

\begin{columns}
\begin{column}{0.4\textwidth}
\begin{center}
\scalebox{1}{\psset{unit=1,algebraic}
\begin{pspicture*}(-1,-0.5)(3.5,2.5)
\psaxes[ticks=none,labels=none]{<->}(0,0)(-0.5,-0.5)(3.5,2.5)
\psplot[linecolor=blue]{0.5}{3.1}{2-(x-1.8)^2}
\psline[linecolor=red,linestyle=dashed](1.8,0)(1.8,2)(0,2)
\psline[linecolor=red,linestyle=dashed](1,0)(1,1.36)(0,1.36)
\footnotesize
\psxTick[ticksize=-3pt 0,labelsep=3pt](1.8){x_0}
\psxTick[ticksize=-3pt 0,labelsep=3pt](1){x}
\psyTick[ticksize=-3pt 0,labelsep=3pt](2){f(x_0)}
\psyTick[ticksize=-3pt 0,labelsep=3pt](1.36){f(x)}
\psyTick[ticksize=0,labelsep=10pt](1.72){\rotateleft{\leq}}
\rput(0.5,0){$($}
\psxTick[ticksize=-3pt 0,labelsep=3pt](0.5){x_0-\delta}
\rput(3.1,0){$)$}
\psxTick[ticksize=-3pt 0,labelsep=3pt](3.1){x_0+\delta}
\psdot[linecolor=green](1.8,2)
\end{pspicture*}}\\
\color{red}Máximo
\end{center}
\end{column}
\begin{column}{0.4\textwidth}
\begin{center}
\scalebox{1}{\psset{unit=1,algebraic}
\begin{pspicture*}(-1,-0.5)(3.5,2.5)
\psaxes[ticks=none,labels=none]{<->}(0,0)(-0.5,-0.5)(3.5,2.5)
\psplot[linecolor=blue]{0.5}{3.1}{(x-1.8)^2+0.5}
\psline[linecolor=red,linestyle=dashed](1.8,0)(1.8,0.5)(0,0.5)
\psline[linecolor=red,linestyle=dashed](1,0)(1,1.14)(0,1.14)
\footnotesize
\psxTick[ticksize=-3pt 0,labelsep=3pt](1.8){x_0}
\psxTick[ticksize=-3pt 0,labelsep=3pt](1){x}
\psyTick[ticksize=-3pt 0,labelsep=3pt](0.5){f(x_0)}
\psyTick[ticksize=-3pt 0,labelsep=3pt](1.14){f(x)}
\psyTick[ticksize=0,labelsep=8pt](0.8){\rotateleft{\leq}}
\rput(0.5,0){$($}
\psxTick[ticksize=-3pt 0,labelsep=3pt](0.5){x_0-\delta}
\rput(3.1,0){$)$}
\psxTick[ticksize=-3pt 0,labelsep=3pt](3.1){x_0+\delta}
\psdot[linecolor=green](1.8,0.5)
\end{pspicture*}}\\
\color{red}Mínimo
\end{center}
\end{column}
\end{columns}
\end{frame} 


\subsection{Concavidad de una función}

%---------------------------------------------------------------------slide----
\begin{frame}
\frametitle{Concavidad de una función}
\begin{definicion}[Función cóncava y convexa]
Se dice que una función $f$ es \emph{cóncava} en un intervalo $I$, si para todo $x_1,x_2\in I$, con $x_1<x_2$, se cumple que el segmento que une los puntos $(x_1,f(x_1))$ y $(x_2,f(x_2))$ queda por debajo de la gráfica de $f$.

Se dice que una función $f$ es \emph{convexa} en un intervalo $I$, si para todo $x_1,x_2\in I$, con $x_1<x_2$, se cumple que el segmento que une los puntos $(x_1,f(x_1))$ y $(x_2,f(x_2))$ queda por encima de la gráfica de $f$.

Al punto donde cambia la concavidad de una función se le llama \emph{punto de inflexión}.
\end{definicion}

\begin{columns}
\begin{column}{0.4\textwidth}
\begin{center}
\scalebox{1}{\psset{unit=1,algebraic}
\begin{pspicture*}(-1,-0.5)(3.5,2.5)
\psaxes[ticks=none,labels=none]{<->}(0,0)(-0.5,-0.5)(3.5,2.5)
\psplot[linecolor=blue]{0.5}{3.1}{(x-1.8)^2+0.5}
\psline[linecolor=red,linestyle=dashed](0.9,0)(0.9,1.31)(0,1.31)
\psline[linecolor=red,linestyle=dashed](2.9,0)(2.9,1.71)(0,1.71)
\psline[linecolor=green]{<->}(0.9,1.31)(2.9,1.71)
\psxTick[ticksize=-3pt 0,labelsep=3pt](0.9){x_1}
\psxTick[ticksize=-3pt 0,labelsep=3pt](2.9){x_2}
\psyTick[ticksize=-3pt 0,labelsep=3pt](1.31){f(x_1)}
\psyTick[ticksize=-3pt 0,labelsep=3pt](1.71){f(x_2)}
\end{pspicture*}}\\
\color{red}Función cóncava
\end{center}
\end{column}
\begin{column}{0.4\textwidth}
\begin{center}
\scalebox{1}{\psset{unit=1,algebraic}
\begin{pspicture*}(-1,-0.5)(3.5,2.5)
\psaxes[ticks=none,labels=none]{<->}(0,0)(-0.5,-0.5)(3.5,2.5)
\psplot[linecolor=blue]{0.5}{3.1}{2-(x-1.8)^2}
\psline[linecolor=red,linestyle=dashed](0.9,0)(0.9,1.19)(0,1.19)
\psline[linecolor=red,linestyle=dashed](2.9,0)(2.9,0.79)(0,0.79)
\psline[linecolor=green]{<->}(0.9,1.19)(2.9,0.79)
\psxTick[ticksize=-3pt 0,labelsep=3pt](0.9){x_1}
\psxTick[ticksize=-3pt 0,labelsep=3pt](2.9){x_2}
\psyTick[ticksize=-3pt 0,labelsep=3pt](1.19){f(x_1)}
\psyTick[ticksize=-3pt 0,labelsep=3pt](0.79){f(x_2)}
\end{pspicture*}}\\
\color{red}Función convexa
\end{center}
\end{column}
\end{columns}
\end{frame} 



\subsection{Funciones periódicas}
%---------------------------------------------------------------------slide----
\begin{frame}
\frametitle{Funciones periódicas}
\begin{definicion}[Función periódica y periodo]
Se dice que una función $f$ es \emph{periódica} si existe un valor $h>0$ tal que
\[f(x+h)=f(x)\]
para todo $x\in \textrm{Dom}(f)$.

Al menor valor de $h$ que verifica la igualdad anterior se le llama \emph{periodo} de $f$, y a la diferencia entre el máximo y el mínimo de la función se le llama \emph{amplitud} de $f$.
\end{definicion}

\begin{center}
\scalebox{1}{\psset{unit=1.5,algebraic}
\begin{pspicture*}(-4,-1.5)(3.5,1.5)
\psaxes[ticks=none,labels=none]{<->}(-1.9634,0)(-2.1,-1.3)(2.1,1.3)
\psplot[linecolor=blue,plotpoints=100]{-1.9634}{2}{cos(4*x)}
\psline[linecolor=red]{|-|}(-1.57075,1)(0,1)
\rput[b](-0.78,1.1){\color{red}Periodo}
\psline[linecolor=red]{|-|}(0,0)(0,1)
\rput[r](-0.1,0.5){\rotateleft{\color{red}Amplitud}}
\end{pspicture*}}
\end{center}
\end{frame} 



\subsection{Funciones polinómicas}
%---------------------------------------------------------------------slide----
\begin{frame}
\frametitle{Funciones polinómicas}
\begin{definicion}[Función polinómica]
Una \emph{función polinómica} es una función de la forma
\[
f(x)=a_0+a_1x+a_2x^2+\cdots+a_nx^n,
\]
donde $n$ es un entero no negativo que se llama \emph{grado del polinomio}, y $a_0,\ldots,a_n$ son constantes reales ($a_n\neq 0$) que se llaman \emph{coeficientes del polinomio}.
\end{definicion}
\begin{center}
\scalebox{1}{\psset{unit=0.6,algebraic}
\begin{pspicture*}(-8,-2)(8,4)
\psaxes[labelFontSize=\scriptstyle,ticksize=-3pt 0,labelsep=2pt]{<->}(0,0)(-3,-2)(3,4)
\psplot[linecolor=blue]{-3}{3}{2*x^2+x-1}
\psplot[linecolor=red]{-3}{3}{x^3-x^2-2*x+2}
\footnotesize
\rput[l](-5.5,2){$f(x)=2x^2+x-1$}
\rput[l](-6.5,-1.5){$g(x)=x^3-x^2-2x+2$}
\end{pspicture*}}
\end{center}
\end{frame} 

%---------------------------------------------------------------------slide----
\begin{frame}
\frametitle{Propiedades de las funciones polinómicas}
\begin{itemize}
\item Su dominio es $\mathbb{R}$.
\item Si el grado es impar, su imagen es $\mathbb{R}$.
\item La función identidad $Id(x)=x$ es un polinomio de grado 1.
\item Las funciones constantes $f(x)=c$ son polinomios de grado 0.
\item Un polinomio de grado $n$ tiene a lo sumo $n$ raíces (puntos donde $f(x)=0$). 
\end{itemize}
\end{frame} 



\subsection{Funciones racionales}
%---------------------------------------------------------------------slide----
\begin{frame}
\frametitle{Funciones racionales}
\begin{definicion}[Función racional]
Una \emph{función racional} es una función de la forma
\[
f(x)=\frac{p(x)}{q(x)}
\]
donde $p(x)$ y $q(x)$ son funcione polinómicas con $q(x)\neq 0$.
\end{definicion}
\begin{center}
\scalebox{1}{\psset{unit=0.55,algebraic}
\begin{pspicture*}(-4,-4)(4,4)
\psaxes[labelFontSize=\scriptstyle,ticksize=-3pt 0,labelsep=2pt]{<->}(0,0)(-4,-4)(4,4)
\psplot[linecolor=blue]{-3.5}{-1.0001}{(2*x+1)/(x^2-1)}
\psplot[linecolor=blue]{-0.9999}{0.9999}{(2*x+1)/(x^2-1)}
\psplot[linecolor=blue]{1.0001}{3.5}{(2*x+1)/(x^2-1)}
\footnotesize
\rput[l](-3.9,2){$f(x)=\dfrac{2x+1}{x^2-1}$}
\end{pspicture*}
\qquad
\begin{pspicture*}(-4,-4)(4,4)
\psaxes[labelFontSize=\scriptstyle,ticksize=-3pt 0,labelsep=2pt]{<->}(0,0)(-4,-4)(4,4)
\psplot[linecolor=blue]{-3.5}{-0.0001}{1/x}
\psplot[linecolor=blue]{0.0001}{3.5}{1/x}
\footnotesize
\rput[l](-3,2){$f(x)=\dfrac{1}{x}$}
\end{pspicture*}}
\end{center}
\end{frame} 


%---------------------------------------------------------------------slide----
\begin{frame}
\frametitle{Propiedades de las funciones racionales}
\begin{itemize}
\item Su dominio es $\mathbb{R}$ menos las raíces del polinomio del denominador. En estos puntos suele haber asíntotas verticales.
\item La tendencia en $\infty$ y $-\infty$ depende del grado del numerador y del denominador. 

Si $f(x)=\dfrac{a_0+\cdots +a_nx^n}{b_0+\cdots+b_mx^m}$, entonces
\begin{itemize}
  \item Si $n>m$ $\rightarrow$ $f(\pm\infty)=\pm\infty$.
  \item Si $n<m$ $\rightarrow$ $f(\pm\infty)=0$.
  \item Si $n=m$ $\rightarrow$ $f(\pm\infty)=\dfrac{a_n}{b_m}$.
\end{itemize}
\item Los polinomios son casos particulares de funciones racionales. 
\item Pueden descomponerse en producto de fracciones simples.
\end{itemize}
\end{frame} 



\subsection{Funciones potenciales}
%---------------------------------------------------------------------slide----
\begin{frame}
\frametitle{Funciones potenciales}
\begin{definicion}[Función potencial]
Una \emph{función potencial} es una función de la forma
\[
f(x)=x^r,
\]
donde $r$ es un número real.
\end{definicion}
\begin{center}
\scalebox{1}{\psset{unit=0.75,algebraic}
\begin{pspicture*}(-7,-3)(7,3)
\psaxes[labelFontSize=\scriptstyle,ticksize=-3pt 0,labelsep=2pt]{<->}(0,0)(-3,-3)(3,3)
\psplot[linecolor=blue]{0}{3}{x^0.5}
\psplot[linecolor=blue]{0}{3}{x^0.3333}
\psplot[linecolor=blue]{-3}{0}{-(-x)^0.333}
\psplot[linecolor=blue]{0}{3}{x^(5/3)}
\psplot[linecolor=blue]{-3}{0}{-(-x)^(5/3)}
\footnotesize
\rput[l](3.2,1.8){$f(x)=x^{1/2}=\sqrt x$}
\rput[l](3.2,1.3){$f(x)=x^{1/3}=\sqrt[3] x$}
\rput[l](2,2.5){$f(x)=x^{5/3}$}
\end{pspicture*}}
\end{center}
\end{frame} 


%---------------------------------------------------------------------slide----
\begin{frame}
\frametitle{Propiedades de las funciones potenciales}
\begin{itemize}
\item Si el exponente es un número racional $n/m$, entonces 
\[x^{n/m}=\sqrt[m]{x^n}.\]
Estas funciones se llaman \emph{irracionales}. En este caso, 
\begin{itemize}
\item si $m$ es impar el dominio es $\mathbb{R}$,
\item si $m$ es par el dominio es $\mathbb{R}^+$.
\end{itemize}
\item Todas las pasan por el punto $(1,1)$.
\item El crecimiento depende del exponente. Si $x>0$ entonces:
\begin{itemize}
\item Exponente positivo $\Rightarrow$ función creciente.
\item Exponente negativo $\Rightarrow$ función decreciente. 
\end{itemize}
Además, si $f(x)=x^r$ y $g(x)=x^s$, entonces:
\begin{itemize}
\item Si $r<s$ $\Rightarrow$ $f(x)>g(x)$ si $0<x<1$ y $f(x)<g(x)$ si $x>1$.
\item Si $r>s$ $\Rightarrow$ $f(x)<g(x)$ si $0<x<1$ y $f(x)>g(x)$ si $x>1$.
\end{itemize}

\item Los polinomios de la forma $f(x)=x^n$ son un caso particular de funciones potenciales. 
\end{itemize}
\end{frame} 



\subsection{Funciones exponenciales}
%---------------------------------------------------------------------slide----
\begin{frame}
\frametitle{Funciones exponenciales}
\begin{definicion}[Función exponencial]
Una \emph{función exponencial} de base $a$ es una función de la forma
\[
f(x)=a^x,
\]
donde $a$ es un valor real positivo distinto de 1.
\end{definicion}
\begin{center}
\scalebox{1}{\psset{unit=0.8,algebraic}
\begin{pspicture*}(-7,-1)(7,4)
\psaxes[labelFontSize=\scriptstyle,ticksize=-3pt 0,labelsep=2pt]{<->}(0,0)(-3,-1)(3,4)
\psplot[linecolor=blue]{-3}{3}{2^x}
\psplot[linecolor=blue]{-3}{3}{0.5^x}
\psplot[linecolor=blue]{-3}{3}{2.7183^x}
\footnotesize
\rput[l](1.5,2){$f(x)=2^x$}
\rput[r](-1.5,2){$f(x)=0.5^x$}
\rput[l](0,3.5){$f(x)=e^x$}
\end{pspicture*}}
\end{center}
\end{frame} 


%---------------------------------------------------------------------slide----
\begin{frame}
\frametitle{Propiedades de las funciones exponenciales}
\begin{itemize}
\item Su dominio es $\mathbb{R}$.
\item Su imagen es $\mathbb{R}^+$.
\item Todas las pasan por el punto $(0,1)$.
\item El crecimiento depende de la base. Si $f(x)=a^x$ entonces
\begin{itemize}
\item Si $0<a<1$ $\Rightarrow$ función decreciente.
\item Si $a>1$ $\Rightarrow$ función creciente. 
\end{itemize}
Además, si $f(x)=a^x$ y $g(x)=b^x$ con $a<b$, entonces
\begin{itemize}
\item Si $x<0$ $\Rightarrow$ $f(x)>g(x)$.
\item Si $x>0$ $\Rightarrow$ $f(x)<g(x)$.
\end{itemize}

\item No tiene sentido para $a=1$ por que sería una función constante.
\end{itemize}
\end{frame} 



\subsection{Funciones logarítmicas}
%---------------------------------------------------------------------slide----
\begin{frame}
\frametitle{Funciones logarítmicas}
\begin{definicion}[Función logarítmica]
Dada una función exponencial $f(x)=a^x$, se define la \emph{función logarítmica} de base $a$ como la función inversa de $f$, y se denota
\[
f^{-1}(x)=\log_a x,
\]
donde $a$ es un valor real positivo distinto de 1.
\end{definicion}
\begin{center}
\scalebox{1}{\psset{unit=0.7,algebraic}
\begin{pspicture*}(-3,-3)(5.5,3)
\psaxes[labelFontSize=\scriptstyle,ticksize=-3pt 0,labelsep=2pt]{<->}(0,0)(-2,-3)(4,3)
\psplot[linecolor=blue]{0.0001}{4}{ln(x)}
\psplot[linecolor=blue]{0.0001}{4}{ln(x)/ln(10)}
\psplot[linecolor=blue]{0.0001}{4}{ln(x)/ln(0.5)}
\footnotesize
\rput[l](2,1.5){$f(x)=\ln x$}
\rput[l](3.5,0.3){$f(x)=\log_{10}x$}
\rput[l](0.3,2.7){$f(x)=\log_{1/2}x$}
\end{pspicture*}}
\end{center}
\end{frame} 


%---------------------------------------------------------------------slide----
\begin{frame}
\frametitle{Propiedades de las funciones logarítmicas}
\begin{itemize}
\item Por ser la inversa de la función exponencial, sus gráficas son simétricas respecto a la bisectriz del primer y tercer cuadrantes. Por tanto:
\begin{itemize}
\item Su dominio es la imagen de la función exponencial, es decir $\mathbb{R}^+$.
\item Su imagen es el dominio de la función exponencial, es decir $\mathbb{R}$.
\end{itemize}
\item Todas pasan por el punto $(1,0)$.
\item El crecimiento depende de la base. Si $f(x)=\log_a x$ entonces
\begin{itemize}
\item Si $0<a<1$ $\Rightarrow$ función decreciente.
\item Si $a>1$ $\Rightarrow$ función creciente. 
\end{itemize}
Además, si $f(x)=\log_a x$ y $g(x)=\log_b x$ con $a<b$, entonces
\begin{itemize}
\item Si $0<x<1$ $\Rightarrow$ $f(x)<g(x)$.
\item Si $x>1$ $\Rightarrow$ $f(x)>g(x)$
\end{itemize}

\item No tiene sentido para $a=1$ por que sería una función constante.
\end{itemize}
\end{frame} 



\subsection{Funciones trigonométricas}
%---------------------------------------------------------------------slide----
\begin{frame}
\frametitle{Funciones trigonométricas}
Surgen en geometría al medir las relaciones entre los catetos de un triángulo rectángulo, que dependen del ángulo del cateto contiguo y la hipotenusa de dicho triángulo.
\begin{center}
\scalebox{1}{\psset{unit=0.8}
\begin{pspicture*}(0,-0.5)(3,2.5)
\scriptsize
\pspolygon(0,0)(2,0)(2,2)
\psarc(0,0){0.3}{0}{45}
\rput[t](1,-0.1){\color{blue}{$a$}}
\rput[l](2.1,1){\color{blue}$b$}
\rput[l](0.4,0.2){\color{red}$\alpha$}
\end{pspicture*}}
\end{center}
No obstante, esta no es la única definición posible, sino que también pueden definirse a partir de la función exponencial compleja.
\begin{center}
\begin{columns}
\begin{column}{0.4\textwidth}
\begin{itemize}
\item Seno
\item Coseno
\item Tangente
\end{itemize}
\end{column}
\begin{column}{0.4\textwidth}
\begin{itemize}
\item Arcoseno
\item Arcocoseno
\item Arcotangente
\end{itemize}
\end{column}
\end{columns}
\end{center}
\end{frame}


%---------------------------------------------------------------------slide----
\begin{frame}
\frametitle{Seno de un ángulo}
\begin{definicion}[Seno de un ángulo]
\begin{columns}
\begin{column}{0.6\textwidth}
Sea $\alpha$ cualquiera de los ángulos agudos de un triángulo rectángulo, se define el \emph{seno} de $\alpha$, y se nota $\sen \alpha$, como el cociente entre el cateto opuesto y la hipotenusa.
\end{column}
\begin{column}{0.2\textwidth}
\begin{center}
\scalebox{1}{\psset{unit=0.6}
\begin{pspicture*}(-0.2,-0.5)(2.5,2.3)
\scriptsize
\pspolygon(0,0)(2,0)(2,2)
\psarc(0,0){0.3}{0}{45}
\rput[t](0,-0.1){\color{blue}{$A$}}
\rput[t](2,-0.1){\color{blue}$B$}
\rput[l](2.1,2.1){\color{blue}$C$}
\rput[l](0.4,0.2){\color{red}$\alpha$}
\end{pspicture*}}\\
\scriptsize
$\sen \alpha= \dfrac{BC}{AC}$
\end{center}
\end{column}
\end{columns}
\end{definicion}
\bigskip
\begin{columns}
\begin{column}{0.6\textwidth}
La definición se extiende fácilmente a ángulos de circunferencia con vértice en el origen y uno de sus lados el eje $OX$, como el cociente entre la ordenada de cualquier punto del otro lado y su distancia al vértice.
\end{column}
\begin{column}{0.2\textwidth}
\begin{center}
\scalebox{1}{\psset{unit=0.6}
\begin{pspicture*}(-0.5,-0.5)(3,2.5)
\scriptsize
\psaxes[arrows=<->,ticks=none,labels=none](0,0)(-0.5,-0.5)(3,2.5)
\psline[linestyle=dashed](0,0)(2.5,2.5)
\psdot(2,2)
\psline[linecolor=blue](2,0)(2,2)
\psline[linecolor=blue](0,0)(2,2)
\psarc(0,0){0.3}{0}{45}
\rput[t](-0.2,-0.1){\color{blue}{$A$}}
\rput[t](2,-0.1){\color{blue}$B$}
\rput[l](2.1,2){\color{blue}$C$}
\rput[l](0.4,0.2){\color{red}$\alpha$}
\end{pspicture*}}\\
\scriptsize
$\sen \alpha= \dfrac{BC}{AC}$
\end{center}
\end{column}
\end{columns}
\end{frame} 


%---------------------------------------------------------------------slide----
\begin{frame}
\frametitle{Función seno}
\begin{definicion}[Función seno]
Se define la función \emph{seno},
\[f(x)=\sen x\]
como la función que asocia a cada ángulo $x$ (habitualmente medido en radianes) su seno.
\end{definicion}
\begin{center}
\scalebox{1}{\psset{unit=1,algebraic}
\begin{pspicture*}(-4,-1.5)(4,1.5)
\psaxes[labelFontSize=\scriptstyle,ticksize=-3pt
0,labelsep=2pt,trigLabels=true,trigLabelBase=2,dx=\psPiH,xunit=\psPi]{<->}(0,0)(-4,-1.5)(4,1.5)
\footnotesize
\psplot[linecolor=blue]{-4}{4}{sin(x)}
\rput[l](1,1.2){$f(x)=\sen x$}
\psline[linecolor=red]{|-|}(-3.1415,0)(3.1415,0)
\rput[t](0.5,-0.1){\color{red}$2\pi$}
\end{pspicture*}}
\end{center}
\end{frame} 


%---------------------------------------------------------------------slide----
\begin{frame}
\frametitle{Propiedades de la función seno}
\begin{itemize}
\item Su dominio es $\mathbb{R}$.
\item Su imagen es el intervalo $[-1,1]$.
\item Es periódica, con periodo $2\pi$ y amplitud $2$
\[\sen (x+2k\pi)= \sen x\quad \forall k\in \mathbb{Z}\]
\item Algunos valores para recordar:
\[
\begin{array}{llll}
\sen 0=0 & \sen \pi/6= 1/2 & \sen \pi/4=\sqrt{2}/2 & \sen \pi/3= \sqrt{3}/2\\
\sen \pi/2 =1 & \sen \pi = 0 & \sen 3\pi/2=-1 & \sen 2\pi=0
\end{array}
\]
\item Es una función impar: $\sen(-x)=-\sen x$.
\end{itemize}
\end{frame} 


%---------------------------------------------------------------------slide----
\begin{frame}
\frametitle{Coseno de un ángulo}
\begin{definicion}[Coseno de un ángulo]
\begin{columns}
\begin{column}{0.6\textwidth}
Sea $\alpha$ cualquiera de los ángulos agudos de un triángulo rectángulo, se define el \emph{coseno} de $\alpha$, y se nota $\cos \alpha$, como el cociente entre el cateto contiguo y la hipotenusa.
\end{column}
\begin{column}{0.2\textwidth}
\begin{center}
\scalebox{1}{\psset{unit=0.6}
\begin{pspicture*}(-0.2,-0.5)(2.5,2.3)
\scriptsize
\pspolygon(0,0)(2,0)(2,2)
\psarc(0,0){0.3}{0}{45}
\rput[t](0,-0.1){\color{blue}{$A$}}
\rput[t](2,-0.1){\color{blue}$B$}
\rput[l](2.1,2.1){\color{blue}$C$}
\rput[l](0.4,0.2){\color{red}$\alpha$}
\end{pspicture*}}\\
\scriptsize 
$\cos \alpha= \dfrac{AB}{AC}$
\end{center}
\end{column}
\end{columns}
\end{definicion}
\bigskip
\begin{columns}
\begin{column}{0.6\textwidth}
La definición se extiende fácilmente a ángulos de circunferencia con vértice en el origen y uno de sus lados el eje $OX$, como el cociente entre la abscisa de cualquier punto del otro lado y su distancia al vértice.
\end{column}
\begin{column}{0.2\textwidth}
\begin{center}
\scalebox{1}{\psset{unit=0.6}
\begin{pspicture*}(-0.5,-0.5)(3,2.5)
\scriptsize
\psaxes[arrows=<->,ticks=none,labels=none](0,0)(-0.5,-0.5)(3,2.5)
\psline[linestyle=dotted](0,0)(2.5,2.5)
\psdot(2,2)
\psline[linestyle=dotted](2,0)(2,2)
\psline[linecolor=blue](0,0)(2,2)
\psline[linecolor=blue](0,0)(2,0)
\psarc(0,0){0.3}{0}{45}
\rput[t](-0.2,-0.1){\color{blue}{$A$}}
\rput[t](2,-0.1){\color{blue}$B$}
\rput[l](2.1,2){\color{blue}$C$}
\rput[l](0.4,0.2){\color{red}$\alpha$}
\end{pspicture*}}\\
\scriptsize
$\cos \alpha= \dfrac{AB}{AC}$
\end{center}
\end{column}
\end{columns}
\end{frame} 


%---------------------------------------------------------------------slide----
\begin{frame}
\frametitle{Función coseno}
\begin{definicion}[Función coseno]
Se define la función \emph{coseno},
\[f(x)=\cos x\]
como la función que asocia a cada ángulo $x$ (habitualmente medido en radianes) su coseno.
\end{definicion}
\begin{center}
\scalebox{1}{\psset{unit=1,algebraic}
\begin{pspicture*}(-4,-1.5)(4,1.5)
\psaxes[labelFontSize=\scriptstyle,ticksize=-3pt
0,labelsep=2pt,trigLabels=true,trigLabelBase=2,dx=\psPiH,xunit=\psPi]{<->}(0,0)(-4,-1.5)(4,1.5)
\footnotesize
\psplot[linecolor=blue]{-4}{4}{cos(x)}
\rput[l](1,1.2){$f(x)=\cos x$}
\psline[linecolor=red]{|-|}(-3.1415,-1)(3.1415,-1)
\rput[b](0.5,-0.9){\color{red}$2\pi$}
\end{pspicture*}}
\end{center}
\end{frame} 

%---------------------------------------------------------------------slide----
\begin{frame}
\frametitle{Propiedades de la función coseno}
\begin{itemize}
\item Su dominio es $\mathbb{R}$.
\item Su imagen es el intervalo $[-1,1]$.
\item Es periódica, con periodo $2\pi$ y amplitud $2$
\[\cos (x+2k\pi)= \cos x\quad \forall k\in \mathbb{Z}\]
\item Algunos valores para recordar:
\[
\begin{array}{llll}
\cos 0=1 & \cos \pi/6= \sqrt{3}/2 & \cos \pi/4=\sqrt{2}/2 & \cos \pi/3= \sqrt{2}/2\\
\cos \pi/2 =0 & \cos \pi = -1 & \cos 3\pi/2=0 & \cos 2\pi=1
\end{array}
\]
\item Es una función par: $\cos(-x)=\cos x$.
\end{itemize}
\end{frame} 


%---------------------------------------------------------------------slide----
\begin{frame}
\frametitle{Tangente de un ángulo}
\begin{definicion}[Tangente de un ángulo]
\begin{columns}
\begin{column}{0.6\textwidth}
Sea $\alpha$ cualquiera de los ángulos agudos de un triángulo rectángulo, se define la \emph{tangente} de $\alpha$, y se nota $\tg \alpha$, como el cociente entre el cateto opuesto y el cateto contiguo.
\end{column}
\begin{column}{0.2\textwidth}
\begin{center}
\scalebox{1}{\psset{unit=0.6}
\begin{pspicture*}(-0.2,-0.5)(2.5,2.3)
\scriptsize
\pspolygon(0,0)(2,0)(2,2)
\psarc(0,0){0.3}{0}{45}
\rput[t](0,-0.1){\color{blue}{$A$}}
\rput[t](2,-0.1){\color{blue}$B$}
\rput[l](2.1,2.1){\color{blue}$C$}
\rput[l](0.4,0.2){\color{red}$\alpha$}
\end{pspicture*}}\\
\scriptsize
$\tg \alpha= \dfrac{BC}{AB}$
\end{center}
\end{column}
\end{columns}
\end{definicion}
\bigskip
\begin{columns}
\begin{column}{0.6\textwidth}
La definición se extiende fácilmente a ángulos de circunferencia con vértice en el origen y uno de sus lados el eje $OX$, como el cociente entre la ordenada y la abscisa de cualquier punto del otro lado.
\end{column}
\begin{column}{0.2\textwidth}
\begin{center}
\scalebox{1}{\psset{unit=0.6}
\begin{pspicture*}(-0.5,-0.5)(3,2.5)
\scriptsize
\psaxes[arrows=<->,ticks=none,labels=none](0,0)(-0.5,-0.5)(3,2.5)
\psline[linestyle=dotted](0,0)(2.5,2.5)
\psdot(2,2)
\psline[linestyle=dotted](2,0)(2,2)
\psline[linecolor=blue](2,0)(2,2)
\psline[linecolor=blue](0,0)(2,0)
\psarc(0,0){0.3}{0}{45}
\rput[t](-0.2,-0.1){\color{blue}{$A$}}
\rput[t](2,-0.1){\color{blue}$B$}
\rput[l](2.1,2){\color{blue}$C$}
\rput[l](0.4,0.2){\color{red}$\alpha$}
\end{pspicture*}}\\
\scriptsize
$\tg \alpha= \dfrac{BC}{AB}$
\end{center}
\end{column}
\end{columns}
\end{frame} 


%---------------------------------------------------------------------slide----
\begin{frame}
\frametitle{Función tangente}
\begin{definicion}[Función tangente]
Se define la función \emph{tangente},
\[f(x)=\tg x=\frac{\sen x}{\cos x}\]
como la función que asocia a cada ángulo $x$ (habitualmente medido en radianes) su tangente.
\end{definicion}
\begin{center}
\scalebox{1}{\psset{unit=0.7,algebraic}
\begin{pspicture*}(-4,-3)(4,3)
\psaxes[labelFontSize=\scriptstyle,ticksize=-3pt
0,labelsep=2pt,trigLabels=true,trigLabelBase=2,dx=\psPiH,xunit=\psPi]{<->}(0,0)(-4,-3)(4,3)
\psplot[linecolor=blue]{-4}{-1.5708}{tan(x)}
\psplot[linecolor=blue]{-1.5707}{1.5707}{tan(x)}
\psplot[linecolor=blue]{1.5708}{4}{tan(x)}
\footnotesize
\rput[l](1,1.2){$f(x)=\tg x$}
\psline[linecolor=red]{|-|}(-3.1415,0)(3.1415,0)
\rput[t](0.5,-0.1){\color{red}$2\pi$}
\end{pspicture*}}
\end{center}
\end{frame} 


%---------------------------------------------------------------------slide----
\begin{frame}
\frametitle{Propiedades de la función tangente}
\begin{itemize}
\item Su dominio es $\mathbb{R}$ menos las raíces del coseno, es decir $\mathbb{R}-\{2k\pi/2: k\in \mathbb{Z}\}$.
\item Su imagen es $\mathbb{R}$.
\item Es periódica, con periodo $2\pi$
\[\tg (x+2k\pi)= \tg x\quad \forall k\in \mathbb{Z}\]
\item Algunos valores para recordar:
\[
\begin{array}{llll}
\tg 0=0 & \tg \pi/6= 1/\sqrt{3} & \tg \pi/4=1 & \tg \pi/3= \sqrt{3}\\
\tg \pi/2 =+\infty & \tg \pi =0 & \tg 3\pi/2=-\infty & \tg 2\pi=0
\end{array}
\]
\end{itemize}
\end{frame} 


%---------------------------------------------------------------------slide----
\begin{frame}
\frametitle{Función arcoseno}
\begin{definicion}[Función arcoseno]
Se define la función \emph{arcoseno},
\[f(x)=\arcsen x\]
como la función inversa de la función seno.
\end{definicion}
\begin{center}
\scalebox{1}{\psset{unit=1,algebraic}
\begin{pspicture*}(-3.5,-2)(3.5,2)
\psplot[linecolor=blue]{-1}{1}{asin(x)}
\psaxes[labelFontSize=\scriptstyle,ticksize=-3pt 0,labelsep=2pt]{<->}(0,0)(-1.5,-2)(1.5,2)
\footnotesize
\rput[l](1,0.5){$f(x)=\arcsen x$}
\end{pspicture*}}
\end{center}
\end{frame} 



%---------------------------------------------------------------------slide----
\begin{frame}
\frametitle{Propiedades de la función arcoseno}
\begin{itemize}
\item Por ser la inversa de la función seno, sus gráficas son simétricas respecto a la bisectriz del primer y tercer cuadrantes. Por tanto:
\begin{itemize}
\item Su dominio es la imagen de la función seno, es decir $[-1,1]$.
\item Su imagen es el dominio restringido de la función seno, es decir $[-\pi/2,\pi/2]$.\footnote{Para que exista la inversa de la función seno, es necesario restringir su dominio a $[-\pi/2,\pi/2]$ para que sea inyectiva.}
\end{itemize}
\item Es creciente en todo el dominio.
\end{itemize}
\end{frame} 


%---------------------------------------------------------------------slide----
\begin{frame}
\frametitle{Función arcocoseno}
\begin{definicion}[Función arcocoseno]
Se define la función \emph{arcocoseno},
\[f(x)=\arccos x\]
como la función inversa de la función coseno.
\end{definicion}
\begin{center}
\scalebox{1}{\psset{unit=1,algebraic}
\begin{pspicture*}(-3.5,-0.5)(3.5,3.5)
\psaxes[labelFontSize=\scriptstyle,ticksize=-3pt 0,labelsep=2pt]{<->}(0,0)(-1.5,-0.5)(1.5,3.5)
\psplot[linecolor=blue]{-1}{1}{acos(x)}
\footnotesize
\rput[l](1,0.5){$f(x)=\arccos x$}
\end{pspicture*}}
\end{center}
\end{frame} 


%---------------------------------------------------------------------slide----
\begin{frame}
\frametitle{Propiedades de la función arcoseno}
\begin{itemize}
\item Por ser la inversa de la función coseno, sus gráficas son simétricas respecto a la bisectriz del primer y tercer cuadrantes. Por tanto:
\begin{itemize}
\item Su dominio es la imagen de la función coseno, es decir $[-1,1]$.
\item Su imagen es el dominio restringido de la función coseno, es decir $[0,\pi]$.\footnote{Para que exista la inversa de la función coseno, es necesario restringir su dominio a $[0,\pi]$ para que sea inyectiva.}
\end{itemize}
\item Es decreciente en todo el dominio.
\end{itemize}
\end{frame} 


%---------------------------------------------------------------------slide----
\begin{frame}
\frametitle{Función arcotangente}
\begin{definicion}[Función arcotangente]
Se define la función \emph{arcotangente},
\[f(x)=\arctg x\]
como la función inversa de la función tangente.
\end{definicion}
\begin{center}
\scalebox{1}{\psset{unit=1}
\begin{pspicture*}(-3,-2)(3,2)
\psaxes[labelFontSize=\scriptstyle,ticksize=-3pt 0,labelsep=2pt]{<->}(0,0)(-3,-2)(3,2)
\psplot[linecolor=blue]{0}{3}{x 1 atan 2 3.1415 mul mul 360 div}
\psplot[linecolor=blue]{-3}{0}{x neg 1 atan 2 3.1415 mul mul 360 div neg}
\footnotesize
\rput[l](0.5,1.5){$f(x)=\arctg x$}
\end{pspicture*}}
\end{center}
\end{frame} 


%---------------------------------------------------------------------slide----
\begin{frame}
\frametitle{Propiedades de la función arcotangente}
\begin{itemize}
\item Por ser la inversa de la función tangente, sus gráficas son simétricas respecto a la bisectriz del primer y tercer cuadrantes. Por tanto:
\begin{itemize}
\item Su dominio es la imagen de la función tangente, es decir $\mathbb{R}$.
\item Su imagen es el dominio restringido de la función tangente, es decir $(-\pi/2,\pi/2)$.\footnote{Para que exista la inversa de la función tangente, es necesario restringir su dominio a $(\pi/2,\pi/2)$ para que sea inyectiva.}
\end{itemize}
\item Es creciente en todo el dominio.
\end{itemize}
\end{frame} 


%---------------------------------------------------------------------slide----
\begin{frame}
\frametitle{Algunas relaciones trigonométricas}
\begin{itemize}
\item $\sen^2 x+\cos^2 x=1$
\item $\sen(x+y)=\sen x \cos y+\cos x \sen y$
\item $\cos(x+y)=\cos x \cos y-\sen x \sen y$
\item $\tg (x+y)= \dfrac{\tg x+\tg y}{1-\tg x \tg y}$
\item $\sen x + \sen y = 2 \sen \dfrac{x+y}{2}\cos\dfrac{x-y}{2}$
\item $\cos x + \cos y = 2 \cos \dfrac{x+y}{2}\cos\dfrac{x-y}{2}$
\item $\cos x - \cos y = -2 \sen \dfrac{x+y}{2}\sen\dfrac{x-y}{2}$
\end{itemize}
\end{frame} 