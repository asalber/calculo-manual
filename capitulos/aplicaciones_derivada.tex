%Version control information
%$HeadURL: https://ejercicioscalculo.googlecode.com/svn/trunk/compendio_ejercicios_calculo.tex $} {$LastChangedDate: 2008-07-09 16:26:20 +0200 (mi�, 09 jul 2008) $
%$LastChangedRevision: 6 $
%$LastChangedBy: asalber $


\section{Aplicaciones de la Derivada}
%---------------------------------------------------------------------slide----
\begin{frame}
\frametitle{Aplicaciones de la Derivada}
\tableofcontents[sectionstyle=show/hide,hideothersubsections]
\end{frame}



\subsection{Estudio del crecimiento de una función}
%---------------------------------------------------------------------slide----
\begin{frame}
\frametitle{Estudio del crecimiento de una función}
La principal aplicación de la derivada es el estudio del crecimiento de una función mediante el signo de la derivada.
\begin{teorema}
Si $f$ es una función cuya derivada existe en un intervalo $I$, entonces:
\begin{itemize}
\item Si $\forall x\in I\ f'(x)\geq 0$ entonces $f$ es creciente en el intervalo $I$.
\item Si $\forall x\in I\ f'(x)\leq 0$ entonces $f$ es decreciente en el intervalo $I$.
\end{itemize}
\end{teorema}
\structure{\textbf{Ejemplo}} La función $f(x)=x^3$ es creciente en todo $\mathbb{R}$ ya que $\forall x\in \mathbb{R}\
f'(x)\geq 0$. \vskip .5cm
\textbf{Observación}. \emph{Una función puede ser creciente o decreciente en un intervalo y no tener derivada.}
\end{frame}


%---------------------------------------------------------------------slide----
\begin{frame}
\frametitle{Estudio del crecimiento de una función}
\framesubtitle{Ejemplo}
Consideremos la función $f(x)=x^4-2x^2+1$. Su derivada $f'(x)=4x^3-4x$ está definida en todo $\mathbb{R}$ y es continua.
\begin{center}
\scalebox{1}{\psset{unit=0.95,algebraic}
\begin{pspicture*}(-5,-4.5)(5,2.5)
\psaxes[labelFontSize=\scriptstyle,ticksize=-3pt 0,labelsep=2pt]{<->}(0,0)(-2.5,-2.1)(2.5,2.1)
\psplot[linecolor=blue]{-1.5}{1.5}{x^4-2*x^2+1}
\psplot[linecolor=red]{-1.2}{1.2}{4*x^3-4*x}
\footnotesize
\rput[r](-1.5,2){\textcolor{blue}{$f(x)=x^4-2x^2+1$}}
\rput[r](-1.5,-1.5){\textcolor{red}{$f'(x)=4x^3-4x$}}
\normalsize
\uncover<2->{
\psaxes[labelFontSize=\scriptstyle,ticksize=-3pt 0,labelsep=2pt]{<->}(0,-3)(-2.5,-3)(2.5,-3)
\footnotesize
\rput[r](-2,-2.5){Crecimiento $f(x)$}
\rput[r](-2,-3.7){Signo $f'(x)$}
}
\uncover<3->{
\rput[c](-1,-3.7){\textcolor{red}{$0$}}
\psline[linecolor=gray,linestyle=dashed]{-}(-1,-3.5)(-1,2)
\rput[c](0,-3.7){\textcolor{red}{$0$}}
\psline[linecolor=gray,linestyle=dashed]{-}(0,-3.5)(0,2)
\rput[c](1,-3.7){\textcolor{red}{$0$}}
\psline[linecolor=gray,linestyle=dashed]{-}(1,-3.5)(1,2)
}
\uncover<4->{
\rput[c](-1.5,-3.7){\textcolor{red}{$-$}}
\rput[c](-1.5,-2.5){\textcolor{blue}{$\downarrow$}}
}
\uncover<5->{
\rput[c](-0.5,-3.7){\textcolor{red}{$+$}}
\rput[c](-0.5,-2.5){\textcolor{blue}{$\uparrow$}}
}
\uncover<6->{
\rput[c](0.5,-3.7){\textcolor{red}{$-$}}
\rput[c](0.5,-2.5){\textcolor{blue}{$\downarrow$}}
}
\uncover<7->{
\rput[c](1.5,-3.7){\textcolor{red}{$+$}}
\rput[c](1.5,-2.5){\textcolor{blue}{$\uparrow$}}
}
\end{pspicture*}}
\end{center}
\end{frame}


\subsection{Determinación de los extremos relativos de una función}
%---------------------------------------------------------------------slide----
\begin{frame}
\frametitle{Determinación de extremos relativos de una función}
Como consecuencia del resultado anterior, la derivada también sirve para determinar los extremos relativos de una función.
\begin{teorema}[Criterio de la primera derivada]
Sea $f$ es una función cuya derivada existe en un intervalo $I$, y sea $x_0\in I$ tal que $f'(x_0)=0$, entonces:
\begin{itemize}
\item Si existe un $\delta>0$ tal que $\forall x\in(x_0-\delta,x_0)\ f'(x)>0$ y $\forall x\in(x_0,x_0+\delta)\ f'(x)<0$ entonces $f$ tiene un \emph{máximo relativo} en $x_0$.
\item Si existe un $\delta>0$ tal que $\forall x\in(x_0-\delta,x_0)\ f'(x)<0$ y $\forall x\in(x_0,x_0+\delta)\ f'(x)>0$ entonces $f$ tiene un \emph{mínimo relativo} en $x_0$.
\item Si existe un $\delta>0$ tal que $\forall x\in(x_0-\delta,x_0)\ f'(x)>0$ y $\forall x\in(x_0,x_0+\delta)\ f'(x)>0$ entonces $f$ tiene un \emph{punto de inflexión creciente} en $x_0$.
\item Si existe un $\delta>0$ tal que $\forall x\in(x_0-\delta,x_0)\ f'(x)<0$ y $\forall x\in(x_0,x_0+\delta)\ f'(x)<0$ entonces $f$ tiene un \emph{punto de inflexión decreciente} en $x_0$.
\end{itemize}
\end{teorema}
Los puntos donde se anula la derivada de una función se denominan \emph{puntos críticos}.

%\textbf{Observación}. \emph{La anulación de la derivada es una condición necesaria pero no suficiente para la existencia de un extremo relativo en un punto.}

%\structure{\textbf{Ejemplo}} La función $f(x)=x^3$ tiene derivada $f'(x)=3x^2$ y por tanto tiene un punto crítico en $x=0$, pero no tiene un extremo relativo en el 0, sino un punto de inflexión.
\end{frame}


%---------------------------------------------------------------------slide----
\begin{frame}
\frametitle{Determinación de extremos relativos de una función}
\framesubtitle{Ejemplo}
Consideremos de nuevo la función $f(x)=x^4-2x^2+1$. Su derivada $f'(x)=4x^3-4x$ está definida en todo $\mathbb{R}$ y es continua.
\begin{center}
\scalebox{1}{\psset{unit=0.95,algebraic}
\begin{pspicture*}(-5,-4.5)(5,2.5)
\psaxes[labelFontSize=\scriptstyle,ticksize=-3pt 0,labelsep=2pt]{<->}(0,0)(-2.5,-2.1)(2.5,2.1)
\psplot[linecolor=blue]{-1.5}{1.5}{x^4-2*x^2+1}
\psplot[linecolor=red]{-1.2}{1.2}{4*x^3-4*x}
\footnotesize
\rput[r](-1.5,2){\textcolor{blue}{$f(x)=x^4-2x^2+1$}}
\rput[r](-1.5,-1.5){\textcolor{red}{$f'(x)=4x^3-4x$}}
\normalsize
\psaxes[labelFontSize=\scriptstyle,ticksize=-3pt 0,labelsep=2pt]{<->}(0,-3)(-2.5,-3)(2.5,-3)
\psline[linecolor=gray,linestyle=dashed]{<-}(-1,-4)(-1,2)
\psline[linecolor=gray,linestyle=dashed]{<-}(0,-4)(0,2)
\psline[linecolor=gray,linestyle=dashed]{<-}(1,-4)(1,2)
\footnotesize
\rput[r](-2,-2.5){Crecimiento $f(x)$}
\rput[r](-2,-3.7){Signo $f'(x)$}
\rput[c](-1.5,-2.5){\textcolor{blue}{$\downarrow$}}
\rput[c](-1.5,-3.7){\textcolor{red}{$-$}}
\rput[c](-1,-3.7){\textcolor{red}{$0$}}
\rput[c](-0.5,-2.5){\textcolor{blue}{$\uparrow$}}
\rput[c](-0.5,-3.7){\textcolor{red}{$+$}}
\rput[c](0,-3.7){\textcolor{red}{$0$}}
\rput[c](0.5,-2.5){\textcolor{blue}{$\downarrow$}}
\rput[c](0.5,-3.7){\textcolor{red}{$-$}}
\rput[c](1,-3.7){\textcolor{red}{$0$}}
\rput[c](1.5,-2.5){\textcolor{blue}{$\uparrow$}}
\rput[c](1.5,-3.7){\textcolor{red}{$+$}}
\uncover<2->{
\rput[r](-2,-4.3){Extremos $f(x)$}
\rput[c](-1,-4.3){\textcolor{blue}{Mín}}
\rput[c](0,-4.3){\textcolor{blue}{Máx}}
\rput[c](1,-4.3){\textcolor{blue}{Mín}}
}
\end{pspicture*}}
\end{center}
\end{frame}



\subsection{Estudio de la concavidad de una función}
%---------------------------------------------------------------------slide---
\begin{frame}
\frametitle{Estudio de la concavidad de una función}
La concavidad de una función puede estudiarse mediante el signo de la segunda derivada.
\begin{teorema}[Criterio de la segunda derivada]
Si $f$ es una función cuya segunda derivada existe en un intervalo $I$, entonces:
\begin{itemize}
\item Si $\forall x\in I\ f''(x)\geq 0$ entonces $f$ es cóncava en el intervalo $I$.
\item Si $\forall x\in I\ f''(x)\leq 0$ entonces $f$ es convexa en el intervalo $I$.
\end{itemize}
\end{teorema}

\structure{\textbf{Ejemplo}} La función $f(x)=x^2$ tiene segunda derivada $f''(x)=2>0$ y por tanto es cóncava en todo $\mathbb{R}$.
\vskip .5cm
\textbf{Observación}. \emph{Una función puede ser cóncava o convexa en un intervalo y no tener derivada.}
\end{frame}


%---------------------------------------------------------------------slide----
\begin{frame}
\frametitle{Estudio de la concavidad de una función}
\framesubtitle{Ejemplo}
Consideremos de nuevo la función $f(x)=x^4-2x^2+1$. Su segunda derivada $f''(x)=12x^2-4$ está definida en todo $\mathbb{R}$ y es continua.
\begin{center}
\scalebox{1}{\psset{unit=0.8,algebraic}
\begin{pspicture*}(-5,-6)(5,2.1)
\psaxes[labelFontSize=\scriptstyle,ticksize=-3pt 0,labelsep=2pt]{<->}(0,0)(-2.5,-4.1)(2.5,2.1)
\psplot[linecolor=blue]{-1.5}{1.5}{x^4-2*x^2+1}
\psplot[linecolor=red]{-1.2}{1.2}{4*x^3-4*x}
\psplot[linecolor=green]{-1.2}{1.2}{12*x^2-4}
\footnotesize
\rput[r](-1.5,1.9){\textcolor{blue}{$f(x)=x^4-2x^2+1$}}
\rput[r](-1.5,-1.5){\textcolor{red}{$f'(x)=4x^3-4x$}}
\rput[l](1,-2){\textcolor{green}{$f''(x)=12x^2-4$}}
\uncover<2->{
\psaxes[ticksize=-3pt 0,labelsep=2pt]{<->}(0,-5)(-2.5,-5)(2.5,-5)
\rput[r](-2,-4.5){Concavidad $f(x)$}
\rput[r](-2,-5.8){Signo $f''(x)$}
}
\uncover<3->{
\rput[c](-0.5773,-5.8){\textcolor{green}{$0$}}
\psline[linecolor=gray,linestyle=dashed](-0.5773,-5.5)(-0.5773,2)
\rput[c](0.5773,-5.8){\textcolor{green}{$0$}}
\psline[linecolor=gray,linestyle=dashed](0.5773,-5.5)(0.5773,2)
}
\uncover<4->{
\rput[c](-1.2,-5.8){\textcolor{green}{$+$}}
\rput[c](-1.2,-4.5){\textcolor{blue}{$\cup$}}
}
\uncover<5->{
\rput[c](0,-5.8){\textcolor{green}{$-$}}
\rput[c](0,-4.5){\textcolor{blue}{$\cap$}}
}
\uncover<6->{
\rput[c](1.2,-5.8){\textcolor{green}{$+$}}
\rput[c](1.2,-4.5){\textcolor{blue}{$\cup$}}
}
\end{pspicture*}}
\end{center}
\end{frame}



\subsection{Polinomios de Taylor}
%---------------------------------------------------------------------slide----
\begin{frame}
\frametitle{Aproximación de funciones mediante polinomios}
Las funciones trigonométricas, exponenciales, logarítmicas y sus composiciones son difíciles de manejar y calcular.

Por contra, los polinomios son funciones más sencillas. Dado un polinomio
\[
p(x) = a_0+a_1x+a_2x^2+\cdots+a_nx^n,
\]
el cálculo de $p(x)$ se reduce a realizar un número finito de productos y sumas. Esto convierte a los polinomios en funciones fáciles de calcular sobre todo para un ordenador.

Además, los polinomios son funciones con muy buenas propiedades ya que
\begin{itemize}
\item Son continuas en todo $\mathbb{R}$.
\item Son derivables (a cualquier orden) en $\mathbb{R}$.
\end{itemize}
\end{frame}


%---------------------------------------------------------------------slide----
\begin{frame}
\frametitle{Aproximación de una función mediante un polinomio}
\begin{block}{Objetivo}
Aproximar una función $f(x)$ mediante un polinomio $p(x)$ cerca de un punto $a$.
\end{block}

Lo importante es que cerca del punto $a$ la función y el polinomio se parezcan, aunque lejos del punto haya grandes diferencias.

Como mínimo exigiremos que la función coincida con el polinomio en el punto $a$, es decir, \[p(a)=f(a).\]
\end{frame}


%---------------------------------------------------------------------slide----
\begin{frame}
\frametitle{Aproximación mediante un polinomio de grado 0}
Un polinomio de grado 0 tiene ecuación
\[
p(x) = a_0,
\]
donde $a_0$ es una constante.

Como el polinomio debe valer lo que la función en el punto $a$, debe cumplir
\[p(a) = a_0 = f (a).\]

En consecuencia, el polinomio de grado 0 que mejor aproxima a $f$ en un entorno del punto $a$ es
\[p(x) = f (a).\]
\end{frame}


%---------------------------------------------------------------------slide----
\begin{frame}
\frametitle{Aproximación mediante un polinomio de grado 0}
\begin{center}
\scalebox{1}{\psset{unit=1.4,algebraic}
\begin{pspicture*}(-2.5,-0.5)(6.5,4)
\psaxes[ticks=none,labels=none]{<->}(0,0)(-0.5,-0.5)(4.5,4)
\psplot[linecolor=blue]{0.5}{3.6}{2.7183^(x-2.5)+0.5}
\rput[b](3.6,3.5){$f$}
\uncover<2->{
\psxTick[ticksize=-3pt 0,labelsep=3pt](2.5){a}
\psline[linestyle=dashed,linecolor=gray](2.5,0)(2.5,1.5)(0,1.5)
\psyTick[ticksize=-3pt 0,labelsep=3pt](1.5){f(a)}
}
\uncover<3->{
\psplot[linecolor=red]{0.5}{3.6}{1.5}
\rput[l](3.7,1.5){$p^0$}
}
\uncover<4->{
\psxTick[ticksize=-3pt 0,labelsep=3pt](3.5){x}
\psline[linestyle=dashed,linecolor=gray](3.5,0)(3.5,3.2183)(0,3.2183)
\psyTick[ticksize=-3pt 0,labelsep=3pt](3.2183){f(x)}
}
\uncover<5->{
\psline{<->}(3.5,1.5)(3.5,3.2183)
\rput[l](3.6,2.3){$f(x)-p^0(x)$}
}
\end{pspicture*}}
\end{center}
\end{frame}


%---------------------------------------------------------------------slide----
\begin{frame}
\frametitle{Aproximación mediante un polinomio de grado 1}
Un polinomio de grado 1 es una recta y tiene ecuación
\[p(x) = a_0+a_1x,\]
aunque también puede escribirse
\[p(x) = a_0+a_1(x-a).\]

De entre todos los polinomio de grado 1, el que mejor aproxima a $f$ en entorno del punto $a$ será el que cumpla las dos condiciones siguientes:
\begin{itemize}
\item[\structure{1-}] $p$ y $f$ valen lo mismo en $a$: $p(a) = f (a)$,
\item[\structure{2-}] $p$ y $f$ tienen la misma pendiente en $a$: $p'(a) = f '(a)$.
\end{itemize}
Esta última condición nos asegura que en un entorno de $a$, $p$ y $f$ tienen aproximadamente la misma tendencia de crecimiento, pero requiere que la función $f$ sea derivable en $a$.
\end{frame}


%---------------------------------------------------------------------slide----
\begin{frame}
\frametitle{La recta tangente: Mejor aproximación de grado 1}
Imponiendo las condiciones anteriores tenemos
\begin{itemize}
\item[\structure{1-}] $p(x)=a_0+a_1(x-a) \Rightarrow p(a)=a_0+a_1(a-a)=a_0=f(a)$,
\item[\structure{2-}] $p'(x)=a_1 \Rightarrow p'(a)=a_1=f'(a)$.
\end{itemize}

Así pues, el polinomio de grado 1 que mejor aproxima a $f$ en un entorno del punto $a$ es
\[p(x) = f(a)+f '(a)(x-a),\]
que resulta ser la recta tangente a $f$ en el punto $(a,f(a))$.

Recordemos, además, que si $f$ es derivable en $a$, entonces podemos aproximar la variación de $f$ en el intervalo $[a,x]$ mediante el diferencial de $f$ en $a$, es decir,
\[f(x)-f(a) \approx f'(a)(x-a)\Leftrightarrow f(x)=f(a)+f'(a)(x-a)\]
\end{frame}


%---------------------------------------------------------------------slide----
\begin{frame}
\frametitle{Aproximación mediante un polinomio de grado 1}
\begin{center}
\scalebox{1}{\psset{unit=1.4,algebraic}
\begin{pspicture*}(-2.5,-0.5)(6.5,4)
\psaxes[ticks=none,labels=none]{<->}(0,0)(-0.5,-0.5)(4.5,4)
\psplot[linecolor=blue]{0.5}{3.6}{2.7183^(x-2.5)+0.5}
\rput[b](3.6,3.5){$f$}
\psxTick[ticksize=-3pt 0,labelsep=3pt](2.5){x_0}
\psline[linestyle=dashed,linecolor=gray](2.5,0)(2.5,1.5)(0,1.5)
\psyTick[ticksize=-3pt 0,labelsep=3pt](1.5){f(x_0)}
\psplot[linecolor=red]{0.5}{3.6}{1.5}
\rput[l](3.7,1.5){$p^0$}
\uncover<2->{
\psplot[linecolor=green]{1.5}{3.6}{(x-1.5)+0.5}
\rput[l](3.7,2.5){$p^1$}
}
\uncover<3->{
\psxTick[ticksize=-3pt 0,labelsep=3pt](3.5){x}
\psline[linestyle=dashed,linecolor=gray](3.5,0)(3.5,3.2183)(0,3.2183)
\psyTick[ticksize=-3pt 0,labelsep=3pt](3.2183){f(x)}
\psline[linestyle=dashed,linecolor=gray](3.5,2.5)(0,2.5)
\psyTick[ticksize=-3pt 0,labelsep=3pt](2.5){p^1(x)}
}
\uncover<4->{
\psline{<->}(3.5,2.5)(3.5,3.2183)
\rput[l](3.6,2.9){$f(x)-p^1(x)$}
}
\end{pspicture*}}
\end{center}
\end{frame}


%---------------------------------------------------------------------slide----
\begin{frame}
\frametitle{Aproximación mediante un polinomio de grado 2}
Un polinomio de grado 2 es una parábola y tiene ecuación
\[p(x) = a_0+a_1x+a_2x^2,\]
aunque también puede escribirse
\[p(x) = a_0+a_1(x-a)+a_2(x-a)^2.\]

De entre todos los polinomio de grado 2, el que mejor aproxima a $f$ en entorno del punto $a$ será el que cumpla las tres condiciones siguientes:
\begin{itemize}
\item[\structure{1-}] $p$ y $f$ valen lo mismo en $a$: $p(a) = f (a)$,
\item[\structure{2-}] $p$ y $f$ tienen la misma pendiente en $a$: $p'(a) = f '(a)$.
\item[\structure{3-}] $p$ y $f$ tienen la misma concavidad en $a$: $p''(a)=f''(a)$.
\end{itemize}
Esta última condición requiere que la función $f$ sea dos veces derivable en $a$.
\end{frame}


%---------------------------------------------------------------------slide----
\begin{frame}
\frametitle{Mejor polinomio de grado 2}
Imponiendo las condiciones anteriores tenemos
\begin{itemize}
\item[\structure{1-}] $p(x)=a_0+a_1(x-a)+a_2(x-a)^2 \Rightarrow p(a)=a_0=f(a)$,
\item[\structure{2-}] $p'(x)=a_1+2a_2(x-a) \Rightarrow p'(a)=a_1+2a_2(a-a)=a_1=f'(a)$,
\item[\structure{3-}] $p''(x)=2a_2 \Rightarrow p''(a)=2a_2=f''(a) \Rightarrow a_2=\frac{f''(a)}{2}$.
\end{itemize}

Así pues, el polinomio de grado 2 que mejor aproxima a $f$ en un entorno del punto $a$ es
\[p(x) = f(a)+f '(a)(x-a)+\frac{f''(a)}{2}(x-a)^2.\]
\end{frame}


%---------------------------------------------------------------------slide----
\begin{frame}
\frametitle{Aproximación mediante un polinomio de grado 2}
\begin{center}
\scalebox{1}{\psset{unit=1.4,algebraic}
\begin{pspicture*}(-2.5,-0.5)(6.5,4)
\psaxes[ticks=none,labels=none]{<->}(0,0)(-0.5,-0.5)(4.5,4)
\psplot[linecolor=blue]{0.5}{3.6}{2.7183^(x-2.5)+0.5}
\rput[b](3.6,3.5){$f$}
\psxTick[ticksize=-3pt 0,labelsep=3pt](2.5){x_0}
\psline[linestyle=dashed,linecolor=gray](2.5,0)(2.5,1.5)(0,1.5)
\psyTick[ticksize=-3pt 0,labelsep=3pt](1.5){f(x_0)}
\psplot[linecolor=red]{0.5}{3.6}{1.5}
\rput[l](3.7,1.5){$p^0$}
\psplot[linecolor=green]{1.5}{3.6}{(x-1.5)+0.5}
\rput[l](3.7,2.5){$p^1$}
\uncover<2->{
\psplot[linecolor=orange]{0.5}{3.6}{-1+x+(x-2.5)^2/2}
\rput[l](3.7,2.9){$p^2$}
}
\uncover<3->{
\psxTick[ticksize=-3pt 0,labelsep=3pt](3.5){x}
\psline[linestyle=dashed,linecolor=gray](3.5,0)(3.5,3.2183)(0,3.2183)
\psyTick[ticksize=-3pt 0,labelsep=3pt](3.2183){f(x)}
\psline[linestyle=dashed,linecolor=gray](3.5,3)(0,3)
\psyTick[ticksize=-3pt 0,labelsep=3pt](3){p^2(x)}
}
\uncover<4->{
\psline{<->}(3.5,3)(3.5,3.2183)
\rput[l](3.7,3.2){$f(x)-p^2(x)$}
}
\end{pspicture*}}
\end{center}
\end{frame}


%---------------------------------------------------------------------slide----
\begin{frame}
\frametitle{Aproximación mediante un polinomio de grado $n$}
Un polinomio de grado $n$ es una parábola y tiene ecuación
\[p(x) = a_0+a_1x+a_2x^2+\cdots +a_nx^2,\]
aunque también puede escribirse
\[p(x) = a_0+a_1(x-a)+a_2(x-a)^2+\cdots +a_n(x-a)^n.\]

De entre todos los polinomio de grado $n$, el que mejor aproxima a $f$ en entorno del punto $a$ será el que cumpla las $n+1$ condiciones siguientes:
\begin{itemize}
\item[\structure{1-}] $p(a) = f (a)$,
\item[\structure{2-}] $p'(a) = f '(a)$,
\item[\structure{3-}] $p''(a)=f''(a)$,
\item[] $\cdots$
\item[\structure{n+1-}] $p^{(n}(a)=f^{(n}(a)$.
\end{itemize}
\alert{Obsérvese que para que se cumplan estas condiciones es necesario que $f$ sea $n$ veces derivable en el punto $a$.}
\end{frame}


%---------------------------------------------------------------------slide----
\begin{frame}
\frametitle{Cálculo de los coeficientes del polinomio de grado $n$}
Si calculamos las sucesivas derivadas de $p$ tenemos:
\begin{align*}
p(x) &= a_0+a_1(x-a)+a_2(x-a)^2+\cdots +a_n(x-a)^n,\\
p'(x)& = a_1+2a_2(x-a)+\cdots +na_n(x-a)^{n-1},\\
p''(x)& = 2a_2+\cdots +n(n-1)a_n(x-a)^{n-2},\\
\vdots\ \
\\
p^{(n}(x)&= n(n-1)(n-2)\cdots 1 a_n=n!a_n.
\end{align*}

Imponiendo las condiciones anteriores tenemos
\begin{itemize}
\item[\structure{1-}] $p(a) = a_0+a_1(a-a)+a_2(a-a)^2+\cdots +a_n(a-a)^n=a_0=f(a)$,
\item[\structure{2-}] $p'(a) = a_1+2a_2(a-a)+\cdots +na_n(a-a)^{n-1}=a_1=f'(a)$,
\item[\structure{3-}] $p''(a) = 2a_2+\cdots +n(n-1)a_n(a-a)^{n-2}=2a_2=f''(a)\Rightarrow a_2=f''(a)/2$,
\item[] $\cdots$
\item[\structure{n+1-}] $p^{(n}(a)=n!a_n=f^{(n}(a)=a_n=\frac{f^{(n}(a)}{n!}$.
\end{itemize}
\end{frame}


%---------------------------------------------------------------------slide----
\begin{frame}
\frametitle{Polinomio de Taylor de orden $n$}
\begin{definicion}[Polinomio de Taylor de orden $n$ para $f$ en el punto $a$]
Dada una función $f$, $n$ veces derivable en un punto $a$, se define el \emph{polinomio de Taylor} de orden $n$ para $f$ en el punto $a$ como
\begin{align*}
p_{f,a}^n(x)&=f(a)+f'(a)(x-a)+\frac{f''(a)}{2}(x-a)^2+\cdots +\frac{f^{(n}(a)}{n!}(x-a)^n = \\ &=\sum_{i=0}^{n}\frac{f^{(i}(a)}{i!}(x-a)^i.
\end{align*}
\end{definicion}

El polinomio de Taylor de orden $n$ para $f$ en el punto $a$ es el polinomio de orden $n$ que mejor aproxima a $f$ en un entorno del punto $a$, ya que es el único que cumple las $n+1$ condiciones anteriores.

Se pude demostrar, además, que
\[
\lim_{x\rightarrow a}\frac{f(x)-p_{f,a}^n(x)}{(x-a)^n}=0,
\]
y la diferencia entre $f$ y $p_{f,a}^n$ es un infinitésimo de orden superior a $n$.
\end{frame}


%---------------------------------------------------------------------slide----
\begin{frame}
\frametitle{Cálculo del polinomio de Taylor}
\framesubtitle{Ejemplo}
Vamos a aproximar la función $f(x)=\log x$ en un entorno del punto $1$ mediante un polinomio de grado $3$.

La ecuación del polinomio de Taylor de orden $3$ para $f$ en el punto $1$ es
\[
p_{f,1}^3(x)=f(1)+f'(1)(x-1)+\frac{f''(1)}{2}(x-1)^2+\frac{f'''(1)}{3!}(x-1)^3.
\]
Calculamos las tres primeras derivadas de $f$ en $1$:
\[
\begin{array}{lll}
f(x)=\log x & \quad & f(1)=\log 1 =0,\\
f'(x)=1/x & & f'(1)=1/1=1,\\
f''(x)=-1/x^2 & & f''(1)=-1/1^2=-1,\\
f'''(x)=2/x^3 & & f'''(1)=2/1^3=2.
\end{array}
\]
Sustituyendo en la ecuación del polinomio obtenemos
\[
p_{f,1}^3(x)=0+1(x-1)+\frac{-1}{2}(x-1)^2+\frac{2}{3!}(x-1)^3= \frac{2}{3}x^3-\frac{3}{2}x^2+3x-\frac{11}{6}.
\]
\end{frame}


%---------------------------------------------------------------------slide----
\begin{frame}
\frametitle{Polinomios de Taylor para la función logaritmo}
\begin{center}
\scalebox{1}{\psset{unit=1.5,algebraic}
\begin{pspicture*}(-1,-2)(4.5,2.1)
\psaxes[labelFontSize=\scriptstyle,ticksize=-3pt 0,labelsep=2pt]{<->}(0,0)(-0.5,-2)(3.5,2)
\footnotesize
\psplot[linecolor=blue]{0.0001}{3}{ln(x)}
\rput[l](3.1,1.1){$f(x)=\log(x)$}
\uncover<2->{
\psplot[linecolor=red]{0.001}{3}{x-1}
\rput[l](3.1,1.9){$p_{f,1}^1=-1+x$}
}
\uncover<3->{
\psplot[linecolor=green]{0.001}{3}{-1+x-(x-1)^2/2}
\rput[l](2.5,0.5){$p_{f,1}^2=-1+x-\frac{1}{2}(x-1)^2$}
}
\uncover<4->{
\psplot[linecolor=orange]{0.001}{3}{-1+x-(x-1)^2/2+2/6*(x-1)^3}
\rput[r](2.6,1.9){$p_{f,1}^3=-1+x-\frac{1}{2}(x-1)^2+\frac{1}{3}(x-1)^3$}
}
\end{pspicture*}}
\end{center}
\end{frame}


%---------------------------------------------------------------------slide----
\begin{frame}
\frametitle{Polinomio de Mc Laurin de orden $n$}
La ecuación del polinomio de Taylor se simplifica cuando el punto en torno al cual queremos aproximar es el $0$.
\begin{definicion}[Polinomio de Mc Laurin de orden $n$ para $f$]
Dada una función $f$, $n$ veces derivable en $0$, se define el \emph{polinomio de Mc Laurin} de orden $n$ para $f$ como
\begin{align*}
p_{f,0}^n(x)&=f(0)+f'(0)x+\frac{f''(0)}{2}x^2+\cdots +\frac{f^{(n}(0)}{n!}x^n = \\ &=\sum_{i=0}^{n}\frac{f^{(i}(0)}{i!}x^i.
\end{align*}
\end{definicion}
\end{frame}


%---------------------------------------------------------------------slide----
\begin{frame}
\frametitle{Cálculo del polinomio de Mc Laurin}
\framesubtitle{Ejemplo}
Vamos a aproximar la función $f(x)=\sen x$ en un entorno del punto $0$ mediante un polinomio de grado $3$.

La ecuación del polinomio de Mc Laurin de orden $3$ para $f$ es
\[
p_{f,0}^3(x)=f(0)+f'(0)x+\frac{f''(0)}{2}x^2+\frac{f'''(0)}{3!}x^3.
\]
Calculamos las tres primeras derivadas de $f$ en $0$:
\[
\begin{array}{lll}
f(x)=\sen x & \quad & f(0)=\sen 0 =0,\\
f'(x)=\cos x & & f'(0)=\cos 0=1,\\
f''(x)=-\sen x & & f''(0)=-\sen 0=0,\\
f'''(x)=-\cos x & & f'''(0)=-\cos 0=-1.
\end{array}
\]
Sustituyendo en la ecuación del polinomio obtenemos
\[
p_{f,0}^3(x)=0+1\cdot x+\frac{0}{2}x^2+\frac{-1}{3!}x^3= x-\frac{x^3}{6}.
\]
\end{frame}


%---------------------------------------------------------------------slide----
\begin{frame}
\frametitle{Polinomios de Mc Laurin para la función seno}
\begin{center}
\scalebox{1}{\psset{unit=1.5,algebraic}
\begin{pspicture*}(-3.5,-2)(4.5,2.1)
\psaxes[labelFontSize=\scriptstyle,ticksize=-3pt 0,labelsep=2pt]{<->}(0,0)(-3,-2)(3,2)
\footnotesize
\psplot[linecolor=blue]{-3}{3}{sin(x)}
\rput[l](3.1,0.1){$f(x)=\sen x$}
\uncover<2->{
\psplot[linecolor=red]{-3}{3}{x}
\rput[l](2.2,1.9){$p_{f,0}^1(x)=x$}
}
\uncover<3->{
\psplot[linecolor=green]{-3}{3}{x-x^3/6}
\rput[l](3.1,-1.6){$p_{f,0}^3(x)=x-\frac{1}{6}x^3$}
}
\uncover<4->{
\psplot[linecolor=orange]{-3}{3}{x-x^3/6+x^5/120}
\rput[l](2.4,0.9){$p_{f,0}^5(x)=x-\frac{1}{6}x^3+\frac{1}{120}x^5$}
}
\end{pspicture*}}
\end{center}
\end{frame}


%---------------------------------------------------------------------slide----
\begin{frame}
\frametitle{Polinomios de Mc Laurin de funciones elementales}
\[
\renewcommand{\arraystretch}{2.5}
\begin{array}{|c|c|}
\hline
f(x) & p_{f,0}^n(x) \\
\hline\hline
\sen x & \displaystyle x-\frac{x^3}{3!}+\frac{x^5}{5!}-\cdots +(-1)^k\frac{x^{2k-1}}{(2k-1)!} \mbox{ si $n=2k$ o $n=2k-1$}\\
\hline
\cos x &  \displaystyle 1-\frac{x^2}{2!}+\frac{x^4}{4!}-\cdots +(-1)^k\frac{x^{2k}}{(2k)!} \mbox{ si $n=2k$ o $n=2k+1$}\\
\hline
\arctg x &  \displaystyle x-\frac{x^3}{3}+\frac{x^5}{5}-\cdots +(-1)^k\frac{x^{2k-1}}{(2k-1)} \mbox{ si $n=2k$ o $n=2k-1$}\\
\hline
e^x & \displaystyle 1+x+\frac{x^2}{2!}+\frac{x^3}{3!}+\cdots + \frac{x^n}{n!}\\
\hline
\log(1+x) & \displaystyle x-\frac{x^2}{2}+\frac{x^3}{3}-\cdots +(-1)^{n-1}\frac{x^n}{n}\\
\hline
\end{array}
\]
\end{frame}



%---------------------------------------------------------------------slide----
\begin{frame}
\frametitle{Resto de Taylor}
Los polinomios de Taylor permiten calcular el valor aproximado de una función cerca de un punto, pero siempre se comete un error en dicha aproximación.
\begin{definicion}[Resto de Taylor]
Si $f$ es una función para la que existe el su polinomio de Taylor de orden $n$ en un punto $a$, $p_{f,a}^n$, entonces se define el \emph{resto de Taylor} de orden $n$ para $f$ en $a$ como
\[
r_{f,a}^n(x)=f(x)-p_{f,a}^n(x).
\]
\end{definicion}

El resto mide el error cometido al aproximar $f(x)$ mediante $p_{f,a}^n(x)$ y nos permite expresar la función $f$ como la suma de un polinomio de Taylor más su resto correspondiente:
\[
f(x)=p_{f,a}^n(x) + r_{f,a}^n(x).
\]
Esta última expresión se conoce como \emph{fórmula de Taylor} de orden $n$ para $f$ en el punto $a$.
\end{frame}


%---------------------------------------------------------------------slide----
\begin{frame}
\frametitle{Forma de Lagrange del resto}
\begin{teorema}[Lagrange]
Sea $f$ una función para la que las $n+1$ primeras derivadas están definidas en el intervalo $[a,x]$. Entonces existe un valor $t\in(a,x)$ tal que el resto de Taylor de orden $n$ para $f$ en el punto $a$ puede expresarse como
\[
r_{f,a}^n(x)=\frac{f^{(n+1}(t)}{(n+1)!}(x-a)^{n+1}.
\]
\end{teorema}

Esta expresión se conoce como \emph{forma de Lagrange del resto}, y realmente no nos permite calcular el resto, ya que para ello deberíamos conocer el valor de $t$.

Sin embargo, al saber que $t\in (a,x)$, podremos acotar el valor del resto y así el error cometido en cualquier aproximación.
\end{frame}


%---------------------------------------------------------------------slide----
\begin{frame}
\frametitle{Acotación del resto}
Una vez fijado el valor de $x$ donde queremos aproximar el valor de la función, el resto en la forma de Lagrange es una función que sólo depende de $t$. Por tanto, el problema de acotar el valor absoluto resto se reduce a encontrar el máximo de la expresión
\[
\left|
\frac{f^{(n+1}(t)}{(n+1)!}(x-a)^{n+1}
\right|
\]
en el intervalo $(a,x)$.

Dicho máximo será una cota del error, en valor absoluto, cometido al aproximar $f(x)$ mediante $p_{f,a}^n(x)$.

Se puede demostrar que al aumentar el grado de un polinomio de Taylor, el resto se hace cada vez menor. En
consecuencia, tomando un $n$ suficientemente grande, podemos conseguir una aproximación tan precisa como queramos. 
\end{frame}


%---------------------------------------------------------------------slide----
\begin{frame}
\frametitle{Acotación del resto}
\framesubtitle{Ejemplo}
Supongamos, por ejemplo, que queremos averiguar cuanto vale aproximadamente $\log 1.1$, y para ello utilizamos el polinomio de Mc Laurin de orden $3$ para la función $f(x)=log(1+x)$
\[
p_{f,0}^3(x)=x-\frac{x^2}{2}+\frac{x^3}{3}.
\]
Como $\log 1.1$ es el valor de $f(x)$ para $x=0.1$, un valor aproximado nos lo dará el polinomio anterior en dicho punto, es decir,
\[
p_{f,0}^3(0.1)=0.1-\frac{0.1^2}{2}+\frac{0.1^3}{3}=0.0953333.
\]
Así pues, $\log 1.1\approx 0.0953333$.

Para darnos una idea de la magnitud del error cometido en esta aproximación vamos a acotar el valor absoluto del resto.
\end{frame}


%---------------------------------------------------------------------slide----
\begin{frame}
\frametitle{Acotación del resto}
\framesubtitle{Ejemplo}
El resto en la forma de Lagrange correspondiente al polinomio anterior es
\[
r_{f,0}^3(x)=\frac{f^{iv}(t)}{4!}x^4=-\frac{1}{4(1+t)^4}x^4,\quad t\in(0,x).
\]
Sustituyendo en $x=0.1$ tenemos
\[
r_{f,0}^3(0.1)=-\frac{0.1^4}{4(1+t)^4}\quad t\in(0\,,\,0.1),
\]
que sólo depende de $t$.

Así pues, para acotar el resto, en valor absoluto, basta con calcular el máximo de la expresión
\[
\left|
-\frac{0.1^4}{4(1+t)^4}
\right|
\]
en el intervalo $[0\,,\,0.1]$.

Dicho máximo se encuentra en $t=0$, de modo que la cota obtenida es
\[
\left|r_{f,0}^3(0.1)\right|\leq
\left| -\frac{0.1^4}{4(1+0)^4} \right| =2.5\cdot 10^{-5}.
\]
\end{frame}


%---------------------------------------------------------------------slide----
\begin{frame}
\frametitle{Reducción del error}
Acabamos de ver que el polinomio de Mc Laurin de orden 3 nos permite aproximar $\log 1.1$ con un error menor de
$2.5\cdot 10^{-5}$, pero ¿hasta qué grado deberíamos llegar para cometer un error menor que $10^{-8}$? 

El resto en la forma de Lagrange correspondiente al polinomio de Mc Laurin de grado $n$ es
\[
r_{f,0}^n(x)=\frac{f^{(n+1}(t)}{(n+1)!}x^{n+1}=\frac{(-1)^n}{(n+1)(1+t)^{n+1}}x^{n+1},\quad t\in(0,x).
\]
Sustituyendo en $x=0.1$ tenemos
\[
r_{f,0}^n(0.1)=\frac{(-1)^n}{(n+1)(1+t)^{n+1}}0.1^{n+1},\quad t\in(0\,,\,0.1).
\]
\end{frame}


%---------------------------------------------------------------------slide----
\begin{frame}
\frametitle{Reducción del error}
Como antes, para acotar el resto, en valor absoluto, debemos buscar el máximo de la expresión
\[
\left|
\frac{(-1)^n}{(n+1)(1+t)^{n+1}}0.1^{n+1}
\right|
\]
en el intervalo $[0\,,\,0.1]$.

Dicho máximo se alcanza, al igual que antes, en $t=0$, de modo que obtenemos la cota
\[
\left|r_{f,0}^n(0.1)\right|\leq
\left|\frac{(-1)^n}{(n+1)(1+0)^{n+1}}0.1^{n+1} \right| =(n+1)^{-1} 10^{-(n+1)}.
\]
Así pues, para que el error sea menor que $10^{-8}$ debemos tomar el menor $n$ que satisfaga
\[
(n+1)^{-1} 10^{-(n+1)}\leq 10^{-8},
\]
y dicho valor es $n=7$.

Por tanto, $p_{f,0}^7(0.1)=0.095310181$ es una aproximación de $\log 1.1$ con un error menor que $10^{-8}$.
\end{frame} 