\section{Medida y Error}
%---------------------------------------------------------------------slide----
\mode<presentation>{
	\begin{frame}
		\frametitle{Medida y Error}
		\tableofcontents[sectionstyle=show/hide,hideothersubsections]
	\end{frame}
}


\subsection{Medida de una magnitud y error asociado}
%---------------------------------------------------------------------slide----
\begin{frame}
	\frametitle{Medidas}
	Cualquier ciencia experimental trata de conocer y comprender el mundo que estudia observándolo y realizando medidas
	experimentales.
	
	A partir de estas medidas se construirán y validarán los modelos matemáticos que expliquen los fenómenos naturales. En
	consecuencia, la bondad y calidad de estas medidas es fundamental a la hora de obtener modelos correctos.
	
	A la hora de realizar medidas experimentales de cualquier magnitud física, debe tenerse en cuenta que \emph{no existe
	ningún modo de medir una magnitud con infinita precisión} y por tanto, toda medida estará afectada de cierta
	\emph{imprecisión} o \emph{error}.
	
	\begin{block}{Medida de una magnitud}
		La medida de cualquier magnitud física $X$ debe expresarse indicando el mejor valor de la misma $x$ acompañado
		del error $\varepsilon$ de dicho valor:
		\[
			X = x\pm \varepsilon,
		\]
		donde $\varepsilon$ es una cota superior del error del valor $x$, es decir, $|X-x|\leq \varepsilon$.
	\end{block}
\end{frame}


%---------------------------------------------------------------------slide----
\begin{frame}
	\frametitle{Expresión de una medida y su error}
	A la hora de expresar una medida $X=x\pm \varepsilon$, deben tenerse en cuenta los siguientes criterios:
	\begin{itemize}
		\item El error $\varepsilon$ debe expresarse sólo con una o dos cifras significativas, donde la primera cifra
		      significativa de un número es la primera cifra distinta de cero comenzando por la izquierda, la segunda es la
		      siguiente, y así sucesivamente.
		      
		      Por tanto, \emph{el error debe redondearse por exceso en la primera o segunda cifra significativa}. 
		      
		\item El valor de la magnitud $x$ debe expresarse con tantas cifras significativas como indique su error y debe
		      redondearse, por exceso o por defecto, en la cifra significativa correspondiente a la última cifra considerada en el
		      error. 
	\end{itemize}
	
	\structure{\textbf{Ejemplo}} Si se mide la temperatura de un líquido y se obtiene un valor $186.27641$ K con un error
	$0.01638$ K, el resultado debe expresarse como
	\[
		T=186.276\pm 0.017 \mbox{ K}.
	\]
\end{frame}


%---------------------------------------------------------------------slide----
\begin{frame}
	\frametitle{Mejor valor de una magnitud}
	Siempre que sea posible las medidas de una magnitud deben repetirse para mejorar la precisión del resultado.
	
	Si se realizan $n$ mediciones de un mismo ente físico, todas ellas realizadas siguiendo un mismo procedimiento, se
	obtiene una muestra de medidas 
	\[
		x_1,x_2,\ldots, x_n.
	\]
	
	En principio, no hay ningún motivo para escoger uno u otro valor particular como mejor valor y por ello debe tomarse
	como mejor valor estimativo de la magnitud la \emph{media aritmética muestral} de los $n$ valores:
	\[
		\bar x = \frac{\sum_{i=1}^n x_i}{n}.
	\] 
\end{frame}



\subsection{Tipos de errores de una medida}
%---------------------------------------------------------------------slide----
\begin{frame}
	\frametitle{Tipos de errores asociados al mejor valor de una magnitud}
	Al medir una magnitud pueden aparecer distintos tipos de errores:
	\begin{itemize}
		\item Errores sistemáticos.
		\item Errores en el aparato de medida.
		\item Errores aleatorios o estadísticos.
	\end{itemize}
	El error total asociado al mejor valor de una magnitud será la suma de cada uno de estos tipos de errores. 
\end{frame}


%---------------------------------------------------------------------slide----
\begin{frame}
	\frametitle{Error sistemático}
	El \emph{error sistemático} $\varepsilon_{\textrm{sist}}$ es el debidos al método de medida utilizado. Suele
	deberse a la falta de calidad o calibrado del aparato de medida o a algún error o sesgo en la fórmula o diseño del
	experimento o a que la medición depende de la pericia del observador que mide.
	
	Afectan por igual a todas las medidas, haciendo que todas de ellas se desvíen en el mismo sentido, siendo por lo
	general difíles de dectectar y eliminar. Son, por lo tanto, los errores más indeseables y no existe ninguna expresión
	matemática que determine su valor, sino que deben ser estimados en cada caso según el método de medida utilizado.
	
	En el caso de errores debidos al calibrado del aparato, pueden eliminarse fácilmente haciendo una calibración del mismo
	antes de realizar las medidas. La calibración puede hacerse fácilmente midiendo una magnitud real de 0 y ajustando el
	aparato, si fuera necesario, para que efectivamente marque 0. 
\end{frame}


%---------------------------------------------------------------------slide----
\begin{frame}
	\frametitle{Error del aparato de medida }
	El \emph{error del aparato} $\Delta x$ es el debido al límite de precisión en la capacidad de medida del instrumento
	utilizado. 
	
	Este error, llamado también \emph{incertidumbre}, está siempre presente y es independiente del observador
	que realiza la medida.
	
	Cuando la escala del aparato no es continua, como ocurre en instrumentos digitales, la incertidumbre se toma igual a la
	mínima lectura que puede hacerse en dicha escala, y cuando es continua se toma igual a la mitad de la mínima lectura.
	
	\structure{\textbf{Ejemplo}} Si se mide el tiempo con un cronómetro digital con mínima unidad la décima de segundo,
	entonces la incertidumbre en las medidas será $\Delta t=0.1$ s, mientras que si se mide con un cronómetro analógico con
	movimiento continuo de la aguja, la incertidumbre será $\Delta t=0.05$ s.
\end{frame}


%---------------------------------------------------------------------slide----
\begin{frame}
	\frametitle{Error aleatorio}
	El \emph{error aleatorio} es el debido a las pequeñas variaciones incontrolables en las condiciones externas que se
	producen de unas medidas a otras, es decir, es el error debido al \emph{azar}. 
	
	Estos errores perturban de manera diferente a cada medida, explicando que se obtengan diferentes valores pese a que
	todas las observaciones se hagan siguiendo el mismo procedimiento o protocolo.
	
	En consecuencia, este error sólo tiene sentido cuando se realizan distintas medidas de una misma magnitud. 
\end{frame}


%---------------------------------------------------------------------slide----
\begin{frame}
	\frametitle{Cálculo del error aleatorio}
	Si se realizan $n$ mediciones, obteniendo los valores $x_1,\ldots,x_n$, entonces cada una de estas medidas puede
	expresarse como
	\[
		x_i = \mu + \varepsilon_i, 
	\]
	donde $\mu$ es la media poblacional de todas la medidas y $\varepsilon_i$ el error aleatorio asociado a cada medida.
	
	Como en el error aleatorio influyen multitud de factores debidos a azar, según el teorema central del límite puede
	afirmarse que $\varepsilon_i\sim N(0,\sigma)$, con $\sigma$ la desviación típica poblacional de todas las medidas, y que
	puede estimarse por medio de la
	\emph{cuasidesviación típica muestral}
	\[
		\hat{s}_x = \sqrt{\frac{\sum_{i=1}^n (x_i-\bar x)^2}{n-1}}, 
	\]
	
	Si, como hemos visto antes, se toma como mejor valor de la magnitud la media muestral $\bar x$, el error
	asociado será su desviación típica, que vale
	\[
		s_{\bar x} = \frac{\hat{s}_x}{\sqrt n} = \sqrt{\frac{\sum_{i=1}^n (x_i-\bar x)^2}{n(n-1)}}.
	\]
\end{frame}


%---------------------------------------------------------------------slide----
\begin{frame}
	\frametitle{Cálculo del error aleatorio}
	\framesubtitle{Ejemplo}
	Supongamos que al medir el periodo de oscilación de un péndulo con un cronómetro obtenemos las siguientes medidas:
	\begin{center}
		$1.21$s -- $1.19$s -- $1.22$s -- $1.18$s -- $1.19$s -- $1.20$s
	\end{center}
	
	Entonces el mejor valor del periodo del péndulo es 
	\[
		\bar x = \frac{1.21+1.19+1.22+1.18+1.19+1.20}{6} =  1.198333\mbox{s},
	\] 
	y el error debido al azar es
	\[
		s_{\bar x} = \sqrt{\frac{(1.21-1.198)^2+\cdots +(1.20-1.198)^2}{6\cdot 5}} = 0.006009\mbox{s}.
	\]
	Así pues, suponiendo que no haya error sistemático, la medida del periodo es, 
	\[
		1.198\pm 0.007\mbox{s}.
	\]
\end{frame}


%---------------------------------------------------------------------slide----
\begin{frame}
	\frametitle{Consideraciones sobre el error aleatorio}
	Puesto que el error aleatorio, si realmente depende del azar, debería seguir una distribución normal, entonces la media
	muestral $\bar x$ también seguirá una distribución normal
	\[
		\bar x\sim N(\mu,s_{\bar x}),
	\]
	de manera que, de acuerdo a la distribución normal se cumple
	\begin{itemize}
		\item $P(\bar x-s_{\bar x}\leq \mu\leq \bar x+s_{\bar x})=0.683$.
		\item $P(\bar x-2s_{\bar x}\leq \mu\leq \bar x+2s_{\bar x})=0.954$.
		\item $P(\bar x-3s_{\bar x}\leq \mu\leq \bar x+3s_{\bar x})=0.997$.
	\end{itemize}
	
	Por consiguiente, cuando se escribe $\bar x\pm s_{\bar x}$ no se está diciendo que el valor real está en el intervalo
	$(\bar x-s_{\bar x},\bar x+s_{\bar x})$, sino que existe aproximadamente un $68\%$ de probabilidad de que el valor real
	esté dentro del intervalo. 
	
	Otra consecuencia de esto mismo es que cuando una medida dista más de 3 veces la desviación típica de la media
	muestral, dicha medida debe descartarse ya que es muy probable que algo ha fallado en la medida. 
\end{frame}


%---------------------------------------------------------------------slide----
\begin{frame}
	\frametitle{Cálculo del error total}
	Según lo visto, el error total de una medida dependerá de si realizamos varias mediciones o sólo una.
	
	\begin{itemize}
		\item \structure{\textbf{Una medición}} Si sólo se realiza una medición, entonces el error aleatorio no tiene sentido y
		      el error total será la suma del error sistemático y el error del aparato, es decir,
		      \[
		      	\varepsilon = \varepsilon_{\textrm{sist}}+\Delta x.
		      \]
		\item \structure{\textbf{Varias mediciones}} Si se realizan varias mediciones, entonces el error total será la suma del
		      error sistemático, de la incertidumbre del aparato y del error aleatorio, es decir,
		      \[
		      	\varepsilon = \varepsilon_{\textrm{sist}}+\Delta x+s_{\bar x},
		      \]
		      aunque cuando se utilizan aparatos suficientemente precisos, la incertidumbre del aparto suele ser bastante menor que
		      el error aleatorio y puede despreciarse.
	\end{itemize} 
\end{frame}


%---------------------------------------------------------------------slide----
\begin{frame}
	\frametitle{Error sistemático vs error aleatorio}
	Utilizando el símil del tiro al blanco, tendríamos estas situaciones
	\begin{center}
		\scalebox{1}{\psset{unit=0.5}
\begin{pspicture*}(0,0)(18,14.5)
\footnotesize
\uncover<2->{
\pscircle[fillstyle=solid,fillcolor=red](4,12){2.5}
\pscircle[fillstyle=solid,fillcolor=white](4,12){2}
\pscircle[fillstyle=solid,fillcolor=red](4,12){1.5}
\pscircle[fillstyle=solid,fillcolor=white](4,12){1}
\pscircle[fillstyle=solid,fillcolor=red](4,12){0.5}
\psdots(4.1,12.1)(3.9,12.2)(3.8,12)(4,11.7)(4.2,11.8)(3.7,11.8)
\rput(4,9.1){Medidas sin error sistemático}
\rput(4,8.3){y error aleatorio pequeño}
}
\uncover<3->{
\pscircle[fillstyle=solid,fillcolor=red](14,12){2.5}
\pscircle[fillstyle=solid,fillcolor=white](14,12){2}
\pscircle[fillstyle=solid,fillcolor=red](14,12){1.5}
\pscircle[fillstyle=solid,fillcolor=white](14,12){1}
\pscircle[fillstyle=solid,fillcolor=red](14,12){0.5}
\psdots(15.1,12.1)(15.4,11)(13,11.9)(13.8,11.2)(14.5,13.2)(13,13.3)(14.2,12.1)
\rput(14,9.1){Medidas sin error sistemático}
\rput(14,8.3){y error aleatorio grande}
}
\uncover<4->{
\pscircle[fillstyle=solid,fillcolor=red](4,4){2.5}
\pscircle[fillstyle=solid,fillcolor=white](4,4){2}
\pscircle[fillstyle=solid,fillcolor=red](4,4){1.5}
\pscircle[fillstyle=solid,fillcolor=white](4,4){1}
\pscircle[fillstyle=solid,fillcolor=red](4,4){0.5}
\psdots(5.1,5.1)(4.9,5.2)(4.8,5)(5,4.7)(5.2,4.8)(3.7,11.8)(4.7,4.8)
\rput(4,1.1){Medidas con error sistemático}
\rput(4,0.3){y error aleatorio pequeño}
}
\uncover<5->{
\pscircle[fillstyle=solid,fillcolor=red](14,4){2.5}
\pscircle[fillstyle=solid,fillcolor=white](14,4){2}
\pscircle[fillstyle=solid,fillcolor=red](14,4){1.5}
\pscircle[fillstyle=solid,fillcolor=white](14,4){1}
\pscircle[fillstyle=solid,fillcolor=red](14,4){0.5}
\psdots(16.1,4.1)(16.4,3)(14,4.9)(14.8,4.2)(15.5,6.2)(14,6.3)(15.2,5.1)
\rput(14,1.1){Medidas con error sistemático}
\rput(14,0.3){y error aleatorio grande}
}
\end{pspicture*}}
	\end{center}
\end{frame}



\subsection{Error de las medidas indirectas}
%---------------------------------------------------------------------slide----
\begin{frame}
	\frametitle{Error de la medidas indirectas}
	Hay magnitudes para las que existe un aparato que permite medirlas directamente, como pueden ser la longitud, el
	tiempo, el volumen o la temperatura, mientras que otras deben ser medidas indirectamente a partir de una fórmula en la
	que intervienen medidas directas u otras indirectas calculadas previamente. Esto puede expresarse funcionalmente de la forma
	\[Y(X_1,\ldots,X_n),\]
	donde $Y$ es la magnitud medida indirectamente y $X_1,\ldots,X_n$ son las medias de las magnitudes de las que depende. 
	
	Puesto que las medidas directas tienen asociado un error, dicho error se propagará a las medidas indirectas según la
	fórmula que permita su cálculo.
	
	De acuerdo a la medición de las magnitudes de las que depende $Y$, puden darse cuatro casos:
	\begin{itemize}
		\item Todas las magnitudes se han medido una sola vez.
		\item Todas las magnitudes se han medido varias veces.
		\item Algunas magnitudes se han medido una sola vez y el resto varias veces.
		\item La magnitud $Y$ se ha medido indirectamente varias veces.
	\end{itemize}
\end{frame}


%---------------------------------------------------------------------slide----
\begin{frame}
	\frametitle{Cálculo del error de las medidas derivadas}
	\framesubtitle{Todas las magnitudes se han medido una sola vez}
	Supongamos que las magnitudes $X_1,\ldots,X_n$ se han medido una sóla vez obteniendo las medidas
	\[
		x_1\pm \varepsilon_1,\ldots, x_n\pm\varepsilon_n,
	\] 
	siendo $\varepsilon_i=\varepsilon_{\textrm{sist}_i}+\Delta x_i$.
	
	Entonces la expresión general del error total del valor obtenido para $Y$ es
	\[
		\varepsilon=\left|\frac{\partial Y(x_1,\ldots,x_n)}{\partial X_1}\right|\varepsilon_1+\cdots+\left|\frac{\partial
			Y(x_1,\ldots,x_n)}{\partial X_n}\right|\varepsilon_n.
	\]
\end{frame}


%---------------------------------------------------------------------slide----
\begin{frame}
	\frametitle{Cálculo del error de las medidas derivadas}
	\framesubtitle{Todas las magnitudes se han medido varias veces}
	Supongamos que las magnitudes $X_1,\ldots,X_n$ se han medido varias veces obteniendo las medidas
	\[
		\bar x_1\pm \varepsilon_1,\ldots, \bar x_n\pm\varepsilon_n,
	\] 
	siendo $\varepsilon_i=\varepsilon_{\textrm{sist}_i}+s_{\bar x_i}$.
	
	Entonces la expresión general del error total del valor obtenido para $Y$ es
	\[
		\varepsilon=\sqrt{\left(\frac{\partial Y(\bar x_1,\ldots,\bar x_n)}{\partial
			X_1}\varepsilon_1\right)^2+\cdots+\left(\frac{\partial Y(\bar x_1,\ldots,\bar x_n)}{\partial
			X_n}\varepsilon_n\right)^2}.
	\]
\end{frame}


%---------------------------------------------------------------------slide----
\begin{frame}
	\frametitle{Cálculo del error de las medidas derivadas}
	\framesubtitle{Algunas magnitudes se han medido una vez y otras varias veces}
	Supongamos que las magnitudes $X_1,\ldots,X_n$ se han medido algunas una sóla vez y otras varias veces obteniendo las
	medidas
	\[
		\bar x_1\pm \varepsilon_1,\ldots, \bar x_n\pm\varepsilon_n,
	\] 
	siendo $\varepsilon_i=\varepsilon_{\textrm{sist}_i}+\Delta x_i$ para
	las magnitudes que se han medido una sóla vez y $\varepsilon_i=\varepsilon_{\textrm{sist}_i}+s_{\bar x_i}$ para las que
	se han medido varias veces.
	
	Como ambos errores son conceptualmente diferences no deben utilizarse en un mismo cálculo. Lo que se hace es convertir
	las incertidumbres en desviaciones típicas, multiplicando por $2/3$. Esto es debido a que en el caso de las
	medidas aisladas se tiene la certeza absoluta de que la verdadera medida está en el intervalo
	$(x_i-\Delta_i,x_i+\Delta_i)$, mientras que para las medidas reiteradas se sabe que la verdadera medida estará en el
	intervalo $(\bar x_i-s_{\bar x_i},\bar x_i+s_{\bar x_i})$ con probabilidad $0.68$, que es más o menos $2/3$ del
	$100\%$.
	
	Una vez transformadas, el error total de $Y$ se calcula como antes
	\[
		\varepsilon=\sqrt{\left(\frac{\partial Y(\bar x_1,\ldots,\bar x_n)}{\partial
			X_1}\varepsilon_1\right)^2+\cdots+\left(\frac{\partial Y(\bar x_1,\ldots,\bar x_n)}{\partial
			X_n}\varepsilon_n\right)^2}.
	\]
\end{frame}


%---------------------------------------------------------------------slide----
\begin{frame}
	\frametitle{Cálculo del error de las medidas derivadas}
	\framesubtitle{Varias medidas indirectas de la magnitud}
	Supongamos que se ha medido varias veces de manera indirecta la magnitud $Y$ obteniendo las medidas
	\[
		\bar y_1\pm \varepsilon_1,\ldots, \bar y_n\pm\varepsilon_n,
	\] 
	donde $\varepsilon_i$ se ha calculado según el caso que corresponda de los anteriores.
	
	En este caso, si efectivamente todas son medidas de un mismo ente físico, deberían coincidir dentro del error según el
	criterio visto antes, es decir, considerando en intervalo $(y_i-3\varepsilon_i,y_i+3\varepsilon_i)$. En tal caso, se
	tomará como mejor valor posible la media aritmética de las medidas, ponderadas por el inverso del cuadrado del error de
	cada valor, es decir,
	\[
		\bar y =
		\frac{\frac{1}{\varepsilon_1^2}y_1+\cdots+\frac{1}{\varepsilon_n^2}y_n}{\frac{1}{\varepsilon_1^2}+\cdots+\frac{1}{\varepsilon_n^2}}
		= =\frac{\sum_{i=1}^n\frac{1}{\varepsilon_i^2}y_i}{\sum_{i=1}^n\frac{1}{\varepsilon_i^2}},
	\]
	y el error asociado a esta medida es
	\[
		\varepsilon = \frac{1}{\sqrt{\frac{1}{\varepsilon_1^2}+\cdots+\frac{1}{\varepsilon_n^2}}}
	\]
\end{frame}



