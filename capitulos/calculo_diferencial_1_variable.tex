\section{Cálculo diferencial en una variable}
%---------------------------------------------------------------------slide----
\mode<presentation>{
	\begin{frame}
		\frametitle{Cálculo diferencial en una variable}
		\tableofcontents[sectionstyle=show/hide,hideothersubsections]
	\end{frame}
}


\subsection{El concepto de derivada}
%---------------------------------------------------------------------slide----
\begin{frame}
	\frametitle{Tasa de variación media}
	\begin{definicion}[Incremento]
		Dada una función $y=f(x)$, se llama \emph{incremento} de $f$ en un intervalo $[a,b]$ a la diferencia entre el valor de $f$ en cada uno de los extremos del intervalo, y se nota
		\[\Delta y= f(b)-f(a).\]
	\end{definicion}
	
	Cuando $f$ es la función identidad $y=x$, se cumple que
	\[\Delta x=\Delta y= f(b)-f(a)=b-a,\]
	y por tanto, el incremento de $x$ en un intervalo es la amplitud del intervalo. Esto nos permite escribir el intervalo $[a,b]$ como $[a,a+\Delta x]$.
	
	\begin{definicion}[Tasa de variación media]
		Se llama \emph{tasa de variación media} de $f$ en el intervalo $[a,a+\Delta x]$, al cociente entre el incremento de $y$ y el incremento de $x$ en dicho intervalo, y se escribe
		\[
			\textrm{TVM}\;f[a,a+\Delta x]=\frac{\Delta y}{\Delta x}=\frac{f(a+\Delta x)-f(a)}{\Delta x}
		\]
	\end{definicion}
\end{frame}


%---------------------------------------------------------------------slide----
\begin{frame}
	\frametitle{Tasa de variación media: Ejemplo}
	Consideremos la función $y=x^2$ que mide el área de un cuadrado de chapa metálica de lado $x$.
	
	Si en un determinado instante el lado del cuadrado es $a$, y sometemos la chapa a un proceso de calentamiento que aumenta el lado del cuadrado una cantidad $\Delta x$, ¿en cuánto se incrementará el área del cuadrado?
	\begin{columns}
		\begin{column}{0.3\textwidth}
			\begin{align*}
				\Delta y & = f(a+\Delta x)-f(a)=(a+\Delta x)^2-a^2=               \\
				         & = a^2+2a\Delta x+\Delta x^2-a^2=2a\Delta x+\Delta x^2. 
			\end{align*}
		\end{column}
		\begin{column}{0.3\textwidth}
			\begin{center}
				\scalebox{1}{\psset{unit=0.6}
\begin{pspicture*}(0,-0.5)(4,4)
\scriptsize
\psframe(0,0)(3,3)
\rput[t](1.5,-0.1){$a$}
\rput[t](1.5,1.7){$a^2$}
\psframe[fillstyle=solid,fillcolor=orange](0,3)(3,4)
\psframe[fillstyle=solid,fillcolor=orange](3,3)(4,4)
\psframe[fillstyle=solid,fillcolor=orange](3,0)(4,3)
\rput[t](3.5,-0.1){$\Delta x$}
\rput[t](1.5,3.7){$a\Delta x$}
\rput[t](3.5,3.7){$\Delta x^2$}
\rput[t](3.5,1.7){$a\Delta x$}
\end{pspicture*}}
			\end{center}
		\end{column}
	\end{columns}
	¿Cuál será la tasa de variación media del área en el intervalo $[a,a+\Delta x]$?
	\[
		\textrm{TVM}\;f[a,a+\Delta x]=\frac{\Delta y}{\Delta x}=\frac{2a\Delta x+\Delta x^2}{\Delta x}=2a+\Delta x.
	\]
\end{frame}


%---------------------------------------------------------------------slide----
\begin{frame}
	\frametitle{Interpretación geométrica de la tasa de variación media}
	La tasa de variación media de $f$ en el intervalo $[a,a+\Delta x]$ es la pendiente de la recta \emph{secante} a $f$ en los puntos $(a,f(a))$ y $(a+\Delta x,f(a+\Delta x))$.
	\begin{center}
		\scalebox{1}{\psset{xunit=2,algebraic}
\begin{pspicture*}(-1,-0.5)(3,4)
\psaxes[ticksize=-3pt 0,labelsep=3pt,ticks=none]{<->}(0,0)(-0.3,-0.5)(2.5,3.8)[$x$,-90][$y$,180]
\psplot[linecolor=blue]{0.2}{1.8}{x^3-x^2+x+0.5}
\footnotesize
\rput[r](1.4,3.5){$f(x)$}
\psxTick[ticksize=-3pt 0,labelsep=3pt](0.5){a}
\psxTick[ticksize=-3pt 0,labelsep=3pt](1.5){a+\Delta x}
\psline[linewidth=0.5pt,linestyle=dashed,linecolor=gray](0.5,0)(0.5,0.875)
\psline[linewidth=0.5pt,linestyle=dashed,linecolor=gray](1.5,0.875)(0,0.875)
\psline[linewidth=0.5pt,linestyle=dashed,linecolor=gray](1.5,0)(1.5,3.125)(0,3.125)
\psyTick[ticksize=-3pt 0,labelsep=3pt](0.875){f(a)}
\psyTick[ticksize=-3pt 0,labelsep=3pt](3.125){f(a+\Delta x)}
\psplot[linecolor=red]{0.2}{1.8}{2.25*x-0.25}
\psline[arrows=|*-|*,linecolor=green](1.5,3.125)(1.5,0.875)
\psline[arrows=|*-|*,linecolor=green](0.5,0.875)(1.5,0.875)
\rput[l](1.6,2){$\Delta y=f(a+\Delta x)-f(a)$}
\rput[t](1,0.8){$\Delta x$}
\end{pspicture*}}
	\end{center}
\end{frame}


%---------------------------------------------------------------------slide----
\begin{frame}
	\frametitle{Tasa de variación instantánea}
	En muchas ocasiones, es interesante estudiar la tasa de variación que experimenta una función, no en intervalo, sino en
	un punto.
	
	Conocer la tendencia de variación de una función en un instante puede ayudarnos a predecir valores en instantes
	próximos.
	
	\begin{definicion}[Tasa de variación instantánea y derivada]
		Dada una función $y=f(x)$, se llama \emph{tasa de variación instantánea} de $f$ en un punto $a$, al límite de la tasa de
		variación media de $f$ en el intervalo $[a,a+\Delta x]$, cuando $\Delta x$ tiende a 0, y lo notaremos
		\[
			\textrm{TVI}\;f (a)=\lim_{\Delta x\rightarrow 0} \textrm{TVM}\; f[a,a+\Delta x]=\lim_{\Delta x\rightarrow 0}\frac{\Delta y}{\Delta x}=\lim_{\Delta x\rightarrow 0}\frac{f(a+\Delta x)-f(a)}{\Delta x}
		\]
		Cuando este límite existe, se dice que la función $f$ es derivable en el punto $a$, y al valor del mismo se le llama
		derivada de $f$ en $a$, y se nota como
		\[
			f'(a) \mbox{ o bien } \frac{df}{dx}(a)
		\]
	\end{definicion}
\end{frame}


%---------------------------------------------------------------------slide----
\begin{frame}
	\frametitle{Tasa de variación instantánea: Ejemplo}
	Consideremos de nuevo la función $y=x^2$ que mide el área de un cuadrado de chapa metálica de lado $x$.
	
	Si en un determinado instante el lado del cuadrado es $a$, y sometemos la chapa a un proceso de calentamiento que
	aumenta el lado del cuadrado, ¿cuál es la tasa de variación instantánea del área del cuadrado en dicho instante?
	\begin{align*}
		\textrm{TVI}\;f(a)] & =\lim_{\Delta x\rightarrow 0}\frac{\Delta y}{\Delta x}=\lim_{\Delta x\rightarrow 0}\frac{f(a+\Delta x)-f(a)}{\Delta x} = \\
		                    & =\lim_{\Delta x\rightarrow 0}\frac{2a\Delta x+\Delta x^2}{\Delta x}=\lim_{\Delta x\rightarrow 0} 2a+\Delta x= 2a.        
	\end{align*}
	Así pues,
	\[
		f'(a)=2a,
	\]
	lo que indica que la tendencia de crecimiento el área es del doble del valor del lado.
	
	El signo de $f'(a)$ indica la tendencia de crecimiento de $f$ en el punto $a$:
	\begin{itemize}
		\item[--]  $f'(a)>0$ indica que la tendencia es creciente.
		\item[--]  $f'(a)<0$ indica que la tendencia es decreciente.
	\end{itemize}
\end{frame}


%---------------------------------------------------------------------slide----
\begin{frame}
	\frametitle{Interpretación geométrica de la tasa de variación instantánea}
	La tasa de variación instantánea de $f$ en el punto $a$ es la pendiente de la recta \emph{tangente} a $f$ en el punto $(a,f(a))$.
	\begin{center}
		\scalebox{1}{\psset{xunit=2,algebraic}
\begin{pspicture*}(-1.5,-0.5)(3,3.5)
\psaxes[ticks=none,labels=none]{<->}(0,0)(-0.3,-0.5)(2,3.3)[$x$,-90][$y$,180]
\psplot[linecolor=blue]{0.2}{1.5}{x^3-x^2+x+0.5}
\footnotesize
\rput[r](1.4,3.2){$f(x)$}
\psxTick[ticksize=-3pt 0,labelsep=3pt](0.5){a}
\psxTick[ticksize=-3pt 0,labelsep=3pt](1.5){x}
\psline[linewidth=0.5pt,linestyle=dashed,linecolor=gray](0.5,0)(0.5,0.875)
\psline[linewidth=0.5pt,linestyle=dashed,linecolor=gray](1.5,0.875)(0,0.875)
\psline[linewidth=0.5pt,linestyle=dashed,linecolor=gray](1.5,0)(1.5,1.625)(0,1.625)
\psyTick[ticksize=-3pt 0,labelsep=3pt](0.875){f(a)}
\psyTick[ticksize=-3pt 0,labelsep=3pt](1.625){f(a)+f'(a)(x-a)}
\psplot[linecolor=red]{0.2}{1.8}{0.75*x+0.5}
\psline[arrows=|*-|*,linecolor=green](1.5,1.625)(1.5,0.875)
\psline[arrows=|*-|*,linecolor=green](0.5,0.875)(1.5,0.875)
\rput[l](1.6,1.25){$f'(a)(x-a)$}
\rput[t](1,0.8){$(x-a)$}
\end{pspicture*}}
	\end{center}
\end{frame}


% ---------------------------------------------------------------------slide----
\begin{frame}
	\frametitle{Interpretación cinemática de la tasa de variación}
	\framesubtitle{Movimiento rectilineo}
	Supongase que la función $f(t)$ describe la posición de un objeto móvil sobre la recta real en el instante $t$.
	Tomando como referencia el origen de coordenadas $O$ y el vector unitario $\mathbf{i}=(1)$, se puede representar la
	posición $P$ del móvil en cada instante $t$ mediante un vector $\vec{OP}=x\mathbf{i}$ donde $x=f(t)$.
	\begin{center}
		\scalebox{1}{\psset{unit=1,algebraic}
\begin{pspicture*}(-5,-1.6)(5,1.5)
\psaxes[ticks=none,labels=none]{-}(0,0)(-0.5,0)(4,0)
\rput[b](2,1){Posición}
\psdot(3,0)
\psline[linecolor=blue]{->}(0,0)(3,0)
\psline[linecolor=red]{->}(0,0)(1,0)
\rput[b](0.5,0.1){$\mathbf{i}$}
\rput[b](3,0.1){$P$}
\rput[l](4.1,0){$\mathbb{R}$}
\psxTick[labelFontSize=\scriptstyle,ticksize=-3pt 0,labelsep=4pt](0){O}
\psxTick[labelFontSize=\scriptstyle,ticksize=-3pt 0,labelsep=4pt](1){1}
\psxTick[ticksize=-3pt 0,labelsep=4pt](3){}
\rput[l](2.9,-0.3){$x=f(t)$}
\psline(-4,0)(-2,0)
\rput[b](-3,1){Tiempo}
\psxTick[ticksize=-3pt 0,labelsep=4pt](-3){t}
\psbezier[linestyle=dashed]{->}(-2.5,-0.3)(-1,-1)(1,-1)(2.5,-0.3)
\rput[t](0,-1){$f$}
\end{pspicture*}}
	\end{center}
	
	\structure{\textbf{Observación}}
	También tiene sentido pensar en $f$ como una función que mide otras magnitudes como por ejemplo la temperatura de un
	cuerpo, la concentración de un gas o la cantidad de un compuesto en una reacción química en un instante $t$.
\end{frame}


% ---------------------------------------------------------------------slide----
\begin{frame}
	\frametitle{Interpretación cinemática de la tasa de variación media}
	En este contexto, si se toman los instantes $t=a$ y $t=a+\Delta t$, ambos del dominio $I$ de $f$, el vector
	\[
		\mathbf{v}_m=\frac{f(a+\Delta t)-f(a)}{\Delta t}
	\]
	que se conoce como \emph{velocidad media} de la trayectoria $f$ entre los instantes $a$ y $a+\Delta t$.
	
	\structure{\textbf{Ejemplo}}
	Un vehículo realiza un viaje de Madrid a Barcelona.
	Sea $f$ la función que da la posición el vehículo en cada instante.
	Si el vehículo parte de Madrid (km 0) a las 8 y llega a Barcelona (km 600) a las 14 horas, entonces la velocidad media
	del vehículo en el trayecto es
	\[
		\mathbf{v}_m=\frac{f(14)-f(8)}{14-8}=\frac{600-0}{6} = 100 km/h.
	\]
\end{frame}


% ---------------------------------------------------------------------slide----
\begin{frame}
	\frametitle{Interpretación cinemática de la derivada}
	Siguiendo en este mismo contexto del movimiento rectilineo, la derivada de $f$ en el instante $t=t_0$ es el vector
	\[
		\mathbf{v}=f'(t_0)=\lim_{\Delta t\rightarrow 0}\frac{f(t_0+\Delta t)-f(t_0)}{\Delta t},
	\]
	que se conoce, siempre que exista el límite, como \emph{velocidad instantánea} o simplemente la \emph{velocidad} de la
	trayectoria $f$ en el instante $t_0$.
	
	Es decir, la derivada de la posición respecto del tiempo, es un campo de vectores que recibe el nombre de
	\emph{velocidad a lo largo de la trayectoria $f$}.
	
	\structure{\textbf{Ejemplo}}
	Siguiendo con el ejemplo anterior, lo que marca el velocímetro en un determinado instante sería el módulo del vector
	velocidad  en ese instante.
\end{frame}



\subsection{Álgebra de derivadas}

%---------------------------------------------------------------------slide----
\begin{frame}
	\frametitle{Propiedades de la derivada}
	Si $y=c$, es una función constante, entonces $y'=0$.
	
	Si $y=x$, es la función identidad, entonces  $y'=1$.
	
	Si $u=f(x)$ y $v=g(x)$ son dos funciones diferenciables, entonces
	\begin{itemize}
		\item $(u+v)'=u'+v'$
		\item $(u-v)'=u'-v'$
		\item $(u\cdot v)'=u'\cdot v+ u\cdot v'$
		\item $\left(\dfrac{u}{v}\right)'=\dfrac{u'\cdot v-u\cdot v'}{v^2}$
	\end{itemize}
\end{frame}


\subsection{Derivada de una función compuesta: La regla de la cadena}

%---------------------------------------------------------------------slide----
\begin{frame}
	\frametitle{Diferencial de una función compuesta}
	\framesubtitle{La regla de la cadena}
	\begin{teorema}[Regla de la cadena] Si
		$y=f\circ g$ es la composición de dos funciones $y=f(z)$ y $z=g(x)$, entonces
		\[
			(f\circ g)'(x)=f'(g(x))g'(x),
		\]
	\end{teorema}
	
	Resulta sencillo demostrarlo con la notación diferencial
	\[
		\frac{dy}{dx}=\frac{dy}{dz}\frac{dz}{dx}=f'(z)g'(x)=f'(g(x))g'(x).
	\]
	
	\structure{\textbf{Ejemplo}} Si $f(z)=\sen z$ y $g(x)=x^2$, entonces $f\circ g(x)=\sen(x^2)$ y, aplicando la regla de la cadena, su derivada
	vale
	\[
		(f\circ g)'(x)=f'(g(x))g'(x) = \cos g(x) 2x = \cos(x^2)2x.
	\]
	Por otro lado, $g\circ f(z)= (\sin z)^2$ y, de nuevo aplicando la regla de la cadena, su derivada vale
	
	\[
		(g\circ f)'(z)=g'(f(z))f'(z) = 2f(z)\cos z = 2\sen z\cos z.
	\]
\end{frame}



\subsection{Derivada de la inversa de una función}
%---------------------------------------------------------------------slide----
\begin{frame}
	\frametitle{Derivada de la función inversa}
	\begin{teorema}[Derivada de la inversa]
		Si $y=f(x)$ es una función y $x=f^{-1}(y)$ es su inversa, entonces
		\[
			\left(f^{-1}\right)'(y)=\frac{1}{f'(x)}=\frac{1}{f'(f^{-1}(y))}
		\]
	\end{teorema}
	
	También resulta sencillo de demostrar con la notación diferencial
	\[
		\frac{dx}{dy}=\frac{1}{dy/dx}=\frac{1}{f'(x)}=\frac{1}{f'(f^{-1}(y))}
	\]
	
	\structure{\textbf{Ejemplo}} La inversa de la función exponencial $y=f(x)=e^x$ es el logaritmo neperiano $x=f^{-1}(y)=\ln y$, de modo que
	para calcular la derivada del logaritmo podemos utilizar el teorema de la derivada de la inversa y se tiene
	\[
		\left(f^{-1}\right)'(y)=\frac{1}{f'(x)}=\frac{1}{e^x}=\frac{1}{e^{\ln y}}=\frac{1}{y}.
	\]
\end{frame}



\subsection{Aproximación de funciones mediante polinomios}

%---------------------------------------------------------------------slide----
\begin{frame}
	\frametitle{Aproximación de una función mediante un polinomio}
	Una apliación muy útil de la derivada es la aproximación de funciones mediante polinomios.
	
	Los polinomios son funciones sencillas de calcular (mediante sumas y productos), que tienen muy buenas propiedades:
	\begin{itemize}
		\item Están definidos en todos los números reales.
		\item Son funciones continuas.
		\item Son derivables hasta cualquier orden y sus derivadas son continuas.
	\end{itemize}
	
	\begin{block}{Objetivo}
		Aproximar una función $f(x)$ mediante un polinomio $p(x)$ cerca de un valor $x=x_0$.
	\end{block}
	
\end{frame}


%---------------------------------------------------------------------slide----
\begin{frame}
	\frametitle{Aproximación mediante un polinomio de grado 0}
	Un polinomio de grado 0 tiene ecuación
	\[
		p(x) = c_0,
	\]
	donde $c_0$ es una constante.
	
	Como el polinomio debe valer lo que la función en el punto $x_0$, debe cumplir
	\[p(x_0) = c_0 = f(x_0).\]
	
	En consecuencia, el polinomio de grado 0 que mejor aproxima a $f$ en un entorno de $x_0$ es
	\[p(x) = f(x_0).\]
\end{frame}


%---------------------------------------------------------------------slide----
\begin{frame}
	\frametitle{Aproximación mediante un polinomio de grado 0}
	\begin{center}
		\scalebox{1}{\psset{unit=1.4,algebraic}
\begin{pspicture*}(-2.5,-0.5)(6.5,4)
\psaxes[ticks=none,labels=none]{<->}(0,0)(-0.5,-0.5)(4.5,4)
\psplot[linecolor=blue]{0.5}{3.6}{2.7183^(x-2.5)+0.5}
\rput[b](3.6,3.5){$f$}
\uncover<2->{
\psxTick[ticksize=-3pt 0,labelsep=3pt](2.5){a}
\psline[linestyle=dashed,linecolor=gray](2.5,0)(2.5,1.5)(0,1.5)
\psyTick[ticksize=-3pt 0,labelsep=3pt](1.5){f(a)}
}
\uncover<3->{
\psplot[linecolor=red]{0.5}{3.6}{1.5}
\rput[l](3.7,1.5){$p^0$}
}
\uncover<4->{
\psxTick[ticksize=-3pt 0,labelsep=3pt](3.5){x}
\psline[linestyle=dashed,linecolor=gray](3.5,0)(3.5,3.2183)(0,3.2183)
\psyTick[ticksize=-3pt 0,labelsep=3pt](3.2183){f(x)}
}
\uncover<5->{
\psline{<->}(3.5,1.5)(3.5,3.2183)
\rput[l](3.6,2.3){$f(x)-p^0(x)$}
}
\end{pspicture*}}
	\end{center}
\end{frame}


%---------------------------------------------------------------------slide----
\begin{frame}
	\frametitle{Aproximación mediante un polinomio de grado 1}
	Un polinomio de grado 1 es una recta y tiene ecuación
	\[
		p(x) = c_0+c_1x,
	\]
	aunque también puede escribirse
	\[
		p(x) = c_0+c_1(x-x_0).
	\]
	
	De entre todos los polinomios de grado 1, el que mejor aproxima a $f$ en entorno de $x_0$ será el que cumpla las dos condiciones siguientes:
	\begin{itemize}
		\item[\structure{1-}] $p$ y $f$ valen lo mismo en $x_0$: $p(x_0) = f(x_0)$,
		\item[\structure{2-}] $p$ y $f$ tienen la misma tasa de crecimiento en $a$: $p'(x_0) = f'(x_0)$.
	\end{itemize}
	Esta última condición nos asegura que en un entorno de $x_0$, $p$ y $f$ tienen aproximadamente la misma tendencia de crecimiento, pero requiere que la función $f$ sea derivable en $x_0$.
\end{frame}


%---------------------------------------------------------------------slide----
\begin{frame}
	\frametitle{La recta tangente: Mejor aproximación de grado 1}
	Imponiendo las condiciones anteriores tenemos
	\begin{itemize}
		\item[\structure{1-}] $p(x)=c_0+c_1(x-x_0) \Rightarrow p(x_0)=c_0+c_1(x_0-x_0)=c_0=f(x_0)$,
		\item[\structure{2-}] $p'(x)=c_1 \Rightarrow p'(x_0)=c_1=f'(x_0)$.
	\end{itemize}
	
	Así pues, el polinomio de grado 1 que mejor aproxima a $f$ en un entorno de $x_0$ es
	\[
		p(x) = f(x_0)+f '(x_0)(x-x_0),
	\]
	que resulta ser la recta tangente a $f$ en el punto $(x_0,f(x_0))$.
\end{frame}


%---------------------------------------------------------------------slide----
\begin{frame}
	\frametitle{Aproximación mediante un polinomio de grado 1}
	\begin{center}
		\scalebox{1}{\psset{unit=1.4,algebraic}
\begin{pspicture*}(-2.5,-0.5)(6.5,4)
\psaxes[ticks=none,labels=none]{<->}(0,0)(-0.5,-0.5)(4.5,4)
\psplot[linecolor=blue]{0.5}{3.6}{2.7183^(x-2.5)+0.5}
\rput[b](3.6,3.5){$f$}
\psxTick[ticksize=-3pt 0,labelsep=3pt](2.5){x_0}
\psline[linestyle=dashed,linecolor=gray](2.5,0)(2.5,1.5)(0,1.5)
\psyTick[ticksize=-3pt 0,labelsep=3pt](1.5){f(x_0)}
\psplot[linecolor=red]{0.5}{3.6}{1.5}
\rput[l](3.7,1.5){$p^0$}
\uncover<2->{
\psplot[linecolor=green]{1.5}{3.6}{(x-1.5)+0.5}
\rput[l](3.7,2.5){$p^1$}
}
\uncover<3->{
\psxTick[ticksize=-3pt 0,labelsep=3pt](3.5){x}
\psline[linestyle=dashed,linecolor=gray](3.5,0)(3.5,3.2183)(0,3.2183)
\psyTick[ticksize=-3pt 0,labelsep=3pt](3.2183){f(x)}
\psline[linestyle=dashed,linecolor=gray](3.5,2.5)(0,2.5)
\psyTick[ticksize=-3pt 0,labelsep=3pt](2.5){p^1(x)}
}
\uncover<4->{
\psline{<->}(3.5,2.5)(3.5,3.2183)
\rput[l](3.6,2.9){$f(x)-p^1(x)$}
}
\end{pspicture*}}
	\end{center}
\end{frame}


%---------------------------------------------------------------------slide----
\begin{frame}
	\frametitle{Aproximación mediante un polinomio de grado 2}
	Un polinomio de grado 2 es una parábola y tiene ecuación
	\[
		p(x) = c_0+c_1x+c_2x^2,
	\]
	aunque también puede escribirse
	\[
		p(x) = c_0+c_1(x-x_0)+c_2(x-x_0)^2.
	\]
	
	De entre todos los polinomio de grado 2, el que mejor aproxima a $f$ en entorno de $x_0$ será el que cumpla las tres condiciones siguientes:
	\begin{itemize}
		\item[\structure{1-}] $p$ y $f$ valen lo mismo en $x_0$: $p(x_0) = f(x_0)$,
		\item[\structure{2-}] $p$ y $f$ tienen la misma tasa de crecimiento en $x_0$: $p'(x_0) = f'(x_0)$.
		\item[\structure{3-}] $p$ y $f$ tienen la misma curvatura en $x_0$: $p''(x_0)=f''(x_0)$.
	\end{itemize}
	Esta última condición requiere que la función $f$ sea dos veces derivable en $x_0$.
\end{frame}


%---------------------------------------------------------------------slide----
\begin{frame}
	\frametitle{Mejor polinomio de grado 2}
	Imponiendo las condiciones anteriores tenemos
	\begin{itemize}
		\item[\structure{1-}] $p(x)=c_0+c_1(x-x_0)+c_2(x-x_0)^2 \Rightarrow p(x_0)=c_0=f(x_0)$,
		\item[\structure{2-}] $p'(x)=c_1+2c_2(x-x_0) \Rightarrow p'(x_0)=c_1+2c_2(x_0-x_0)=c_1=f'(x_0)$,
		\item[\structure{3-}] $p''(x)=2c_2 \Rightarrow p''(x_0)=2c_2=f''(x_0) \Rightarrow c_2=\frac{f''(x_0)}{2}$.
	\end{itemize}
	
	Así pues, el polinomio de grado 2 que mejor aproxima a $f$ en un entorno de $x_0$ es
	\[
		p(x) = f(x_0)+f'(x_0)(x-x_0)+\frac{f''(x_0)}{2}(x-x_0)^2.
	\]
\end{frame}


%---------------------------------------------------------------------slide----
\begin{frame}
	\frametitle{Aproximación mediante un polinomio de grado 2}
	\begin{center}
		\scalebox{1}{\psset{unit=1.4,algebraic}
\begin{pspicture*}(-2.5,-0.5)(6.5,4)
\psaxes[ticks=none,labels=none]{<->}(0,0)(-0.5,-0.5)(4.5,4)
\psplot[linecolor=blue]{0.5}{3.6}{2.7183^(x-2.5)+0.5}
\rput[b](3.6,3.5){$f$}
\psxTick[ticksize=-3pt 0,labelsep=3pt](2.5){x_0}
\psline[linestyle=dashed,linecolor=gray](2.5,0)(2.5,1.5)(0,1.5)
\psyTick[ticksize=-3pt 0,labelsep=3pt](1.5){f(x_0)}
\psplot[linecolor=red]{0.5}{3.6}{1.5}
\rput[l](3.7,1.5){$p^0$}
\psplot[linecolor=green]{1.5}{3.6}{(x-1.5)+0.5}
\rput[l](3.7,2.5){$p^1$}
\uncover<2->{
\psplot[linecolor=orange]{0.5}{3.6}{-1+x+(x-2.5)^2/2}
\rput[l](3.7,2.9){$p^2$}
}
\uncover<3->{
\psxTick[ticksize=-3pt 0,labelsep=3pt](3.5){x}
\psline[linestyle=dashed,linecolor=gray](3.5,0)(3.5,3.2183)(0,3.2183)
\psyTick[ticksize=-3pt 0,labelsep=3pt](3.2183){f(x)}
\psline[linestyle=dashed,linecolor=gray](3.5,3)(0,3)
\psyTick[ticksize=-3pt 0,labelsep=3pt](3){p^2(x)}
}
\uncover<4->{
\psline{<->}(3.5,3)(3.5,3.2183)
\rput[l](3.7,3.2){$f(x)-p^2(x)$}
}
\end{pspicture*}}
	\end{center}
\end{frame}


%---------------------------------------------------------------------slide----
\begin{frame}
	\frametitle{Aproximación mediante un polinomio de grado $n$}
	Un polinomio de grado $n$ tiene ecuación
	\[p(x) = c_0+c_1x+c_2x^2+\cdots +c_nx^n,\]
	aunque también puede escribirse
	\[p(x) = c_0+c_1(x-x_0)+c_2(x-x_0)^2+\cdots +c_n(x-x_0)^n.\]
	
	De entre todos los polinomio de grado $n$, el que mejor aproxima a $f$ en entorno de $x_0$ será el que cumpla las $n+1$ condiciones siguientes:
	\begin{itemize}
		\item[\structure{1-}] $p(x_0) = f(x_0)$,
		\item[\structure{2-}] $p'(x_0) = f'(x_0)$,
		\item[\structure{3-}] $p''(x_0)=f''(x_0)$,
		\item[] $\cdots$
		\item[\structure{n+1-}] $p^{(n}(x_0)=f^{(n}(x_0)$.
	\end{itemize}
	\alert{Obsérvese que para que se cumplan estas condiciones es necesario que $f$ sea $n$ veces derivable en $x_0$.}
\end{frame}


%---------------------------------------------------------------------slide----
\begin{frame}
	\frametitle{Cálculo de los coeficientes del polinomio de grado $n$}
	Las sucesivas derivadas de $p$ valen
	\begin{align*}
		p(x)      & = c_0+c_1(x-x_0)+c_2(x-x_0)^2+\cdots +c_n(x-x_0)^n, \\
		p'(x)     & = c_1+2c_2(x-x_0)+\cdots +nc_n(x-x_0)^{n-1},        \\
		p''(x)    & = 2c_2+\cdots +n(n-1)c_n(x-x_0)^{n-2},              \\
		\vdots\ \
		\\
		p^{(n}(x) & = n(n-1)(n-2)\cdots 1 c_n=n!c_n.                    
	\end{align*}
	
	Imponiendo las condiciones anteriores se tiene
	\begin{itemize}
		\item[\structure{1-}] $p(x_0) = c_0+c_1(x_0-x_0)+c_2(x_0-x_0)^2+\cdots +c_n(x_0-x_0)^n=c_0=f(x_0)$,
		\item[\structure{2-}] $p'(x_0) = c_1+2c_2(x_0-x_0)+\cdots +nc_n(x_0-x_0)^{n-1}=c_1=f'(x_0)$,
		\item[\structure{3-}] $p''(x_0) = 2c_2+\cdots +n(n-1)c_n(x_0-x_0)^{n-2}=2c_2=f''(x_0)\Rightarrow c_2=f''(x_0)/2$,
		\item[] $\cdots$
		\item[\structure{n+1-}] $p^{(n}(x_0)=n!c_n=f^{(n}(x_0)=c_n=\frac{f^{(n}(x_0)}{n!}$.
	\end{itemize}
\end{frame}


%---------------------------------------------------------------------slide----
\begin{frame}
	\frametitle{Polinomio de Taylor de orden $n$}
	\begin{definicion}[Polinomio de Taylor de orden $n$ para $f$ en el punto $a$]
		Dada una función $f$, $n$ veces derivable en $x=x_0$, se define el \emph{polinomio de Taylor} de orden $n$ para $f$ en $x_0$ como
		\begin{align*}
			p_{f,x_0}^n(x) & =f(x_0)+f'(x_0)(x-x_0)+\frac{f''(x_0)}{2}(x-x_0)^2+\cdots +\frac{f^{(n}(x_0)}{n!}(x-x_0)^n = \\ &=\sum_{i=0}^{n}\frac{f^{(i}(x_0)}{i!}(x-x_0)^i,
		\end{align*}
		o bien, escribiendo $x=x_0+h$
		\[
			p_f^n(x_0+h)=f(x_0)+f'(x_0)h+\frac{f''(x_0)}{2}h^2+\cdots +\frac{f^{(n}(x_0)}{n!}h^n =\sum_{i=0}^{n}\frac{f^{(i}(x_0)}{i!}h^i,
		\]
	\end{definicion}
	
	El polinomio de Taylor de orden $n$ para $f$ en $x_0$ es el polinomio de orden $n$ que mejor aproxima a $f$ alrededor de $x_0$, ya que es el único que cumple las $n+1$ condiciones anteriores.
\end{frame}


%---------------------------------------------------------------------slide----
\begin{frame}
	\frametitle{Cálculo del polinomio de Taylor}
	\framesubtitle{Ejemplo}
	Vamos a aproximar la función $f(x)=\log x$ en un entorno del punto $1$ mediante un polinomio de grado $3$.
	
	La ecuación del polinomio de Taylor de orden $3$ para $f$ en el punto $1$ es
	\[
		p_{f,1}^3(x)=f(1)+f'(1)(x-1)+\frac{f''(1)}{2}(x-1)^2+\frac{f'''(1)}{3!}(x-1)^3.
	\]
	Calculamos las tres primeras derivadas de $f$ en $1$:
	\[
		\begin{array}{lll}
			f(x)=\log x   & \quad & f(1)=\log 1 =0,   \\
			f'(x)=1/x     &       & f'(1)=1/1=1,      \\
			f''(x)=-1/x^2 &       & f''(1)=-1/1^2=-1, \\
			f'''(x)=2/x^3 &       & f'''(1)=2/1^3=2.  
		\end{array}
	\]
	Sustituyendo en la ecuación del polinomio se tiene
	\[
		p_{f,1}^3(x)=0+1(x-1)+\frac{-1}{2}(x-1)^2+\frac{2}{3!}(x-1)^3= \frac{2}{3}x^3-\frac{3}{2}x^2+3x-\frac{11}{6}.
	\]
\end{frame}


%---------------------------------------------------------------------slide----
\begin{frame}
	\frametitle{Polinomios de Taylor para la función logaritmo}
	\begin{center}
		\scalebox{1}{\psset{unit=1.5,algebraic}
\begin{pspicture*}(-1,-2)(4.5,2.1)
\psaxes[labelFontSize=\scriptstyle,ticksize=-3pt 0,labelsep=2pt]{<->}(0,0)(-0.5,-2)(3.5,2)
\footnotesize
\psplot[linecolor=blue]{0.0001}{3}{ln(x)}
\rput[l](3.1,1.1){$f(x)=\log(x)$}
\uncover<2->{
\psplot[linecolor=red]{0.001}{3}{x-1}
\rput[l](3.1,1.9){$p_{f,1}^1=-1+x$}
}
\uncover<3->{
\psplot[linecolor=green]{0.001}{3}{-1+x-(x-1)^2/2}
\rput[l](2.5,0.5){$p_{f,1}^2=-1+x-\frac{1}{2}(x-1)^2$}
}
\uncover<4->{
\psplot[linecolor=orange]{0.001}{3}{-1+x-(x-1)^2/2+2/6*(x-1)^3}
\rput[r](2.6,1.9){$p_{f,1}^3=-1+x-\frac{1}{2}(x-1)^2+\frac{1}{3}(x-1)^3$}
}
\end{pspicture*}}
	\end{center}
\end{frame}


%---------------------------------------------------------------------slide----
\begin{frame}
	\frametitle{Polinomio de Maclaurin de orden $n$}
	La ecuación del polinomio de Taylor se simplifica cuando el punto en torno al cual queremos aproximar es el $0$.
	\begin{definicion}[Polinomio de Maclaurin de orden $n$ para $f$]
		Dada una función $f$, $n$ veces derivable en $0$, se define el \emph{polinomio de Maclaurin} de orden $n$ para $f$ como
		\begin{align*}
			p_{f,0}^n(x) & =f(0)+f'(0)x+\frac{f''(0)}{2}x^2+\cdots +\frac{f^{(n}(0)}{n!}x^n = \\ &=\sum_{i=0}^{n}\frac{f^{(i}(0)}{i!}x^i.
		\end{align*}
	\end{definicion}
\end{frame}


%---------------------------------------------------------------------slide----
\begin{frame}
	\frametitle{Cálculo del polinomio de Maclaurin}
	\framesubtitle{Ejemplo}
	Vamos a aproximar la función $f(x)=\sen x$ en un entorno del punto $0$ mediante un polinomio de grado $3$.
	
	La ecuación del polinomio de Maclaurin de orden $3$ para $f$ es
	\[
		p_{f,0}^3(x)=f(0)+f'(0)x+\frac{f''(0)}{2}x^2+\frac{f'''(0)}{3!}x^3.
	\]
	Calculamos las tres primeras derivadas de $f$ en $0$:
	\[
		\begin{array}{lll}
			f(x)=\sen x     & \quad & f(0)=\sen 0 =0,     \\
			f'(x)=\cos x    &       & f'(0)=\cos 0=1,     \\
			f''(x)=-\sen x  &       & f''(0)=-\sen 0=0,   \\
			f'''(x)=-\cos x &       & f'''(0)=-\cos 0=-1. 
		\end{array}
	\]
	Sustituyendo en la ecuación del polinomio obtenemos
	\[
		p_{f,0}^3(x)=0+1\cdot x+\frac{0}{2}x^2+\frac{-1}{3!}x^3= x-\frac{x^3}{6}.
	\]
\end{frame}


%---------------------------------------------------------------------slide----
\begin{frame}
	\frametitle{Polinomios de Maclaurin para la función seno}
	\begin{center}
		\scalebox{1}{\psset{unit=1.5,algebraic}
\begin{pspicture*}(-3.5,-2)(4.5,2.1)
\psaxes[labelFontSize=\scriptstyle,ticksize=-3pt 0,labelsep=2pt]{<->}(0,0)(-3,-2)(3,2)
\footnotesize
\psplot[linecolor=blue]{-3}{3}{sin(x)}
\rput[l](3.1,0.1){$f(x)=\sen x$}
\uncover<2->{
\psplot[linecolor=red]{-3}{3}{x}
\rput[l](2.2,1.9){$p_{f,0}^1(x)=x$}
}
\uncover<3->{
\psplot[linecolor=green]{-3}{3}{x-x^3/6}
\rput[l](3.1,-1.6){$p_{f,0}^3(x)=x-\frac{1}{6}x^3$}
}
\uncover<4->{
\psplot[linecolor=orange]{-3}{3}{x-x^3/6+x^5/120}
\rput[l](2.4,0.9){$p_{f,0}^5(x)=x-\frac{1}{6}x^3+\frac{1}{120}x^5$}
}
\end{pspicture*}}
	\end{center}
\end{frame}


%---------------------------------------------------------------------slide----
\begin{frame}
	\frametitle{Polinomios de Maclaurin de funciones elementales}
	\[
		\renewcommand{\arraystretch}{2.5}
		\begin{array}{|c|c|}
			\hline
			f(x)      & p_{f,0}^n(x)                                                                                                      \\
			\hline\hline
			\sen x    & \displaystyle x-\frac{x^3}{3!}+\frac{x^5}{5!}-\cdots +(-1)^k\frac{x^{2k-1}}{(2k-1)!} \mbox{ si $n=2k$ o $n=2k-1$} \\
			\hline
			\cos x    & \displaystyle 1-\frac{x^2}{2!}+\frac{x^4}{4!}-\cdots +(-1)^k\frac{x^{2k}}{(2k)!} \mbox{ si $n=2k$ o $n=2k+1$}     \\
			\hline
			\arctg x  & \displaystyle x-\frac{x^3}{3}+\frac{x^5}{5}-\cdots +(-1)^k\frac{x^{2k-1}}{(2k-1)} \mbox{ si $n=2k$ o $n=2k-1$}    \\
			\hline
			e^x       & \displaystyle 1+x+\frac{x^2}{2!}+\frac{x^3}{3!}+\cdots + \frac{x^n}{n!}                                           \\
			\hline
			\log(1+x) & \displaystyle x-\frac{x^2}{2}+\frac{x^3}{3}-\cdots +(-1)^{n-1}\frac{x^n}{n}                                       \\
			\hline
		\end{array}
	\]
\end{frame}


%---------------------------------------------------------------------slide----
\begin{frame}
	\frametitle{Resto de Taylor}
	Los polinomios de Taylor permiten calcular el valor aproximado de una función cerca de un valor $x_0$, pero siempre se comete un error en dicha aproximación.
	\begin{definicion}[Resto de Taylor]
		Si $f$ es una función para la que existe el su polinomio de Taylor de orden $n$ en $x_0$, $p_{f,x_0}^n$, entonces se define el \emph{resto de Taylor} de orden $n$ para $f$ en $x_0$ como
		\[
			r_{f,x_0}^n(x)=f(x)-p_{f,x_0}^n(x).
		\]
	\end{definicion}
	
	El resto mide el error cometido al aproximar $f(x)$ mediante $p_{f,x_0}^n(x)$ y permite expresar la función $f$ como la suma de un polinomio de Taylor más su resto correspondiente:
	\[
		f(x)=p_{f,x_0}^n(x) + r_{f,x_0}^n(x).
	\]
	Esta expresión se conoce como \emph{fórmula de Taylor} de orden $n$ para $f$ en $x_0$. Se pude demostrar, además, que
	\[
		\lim_{h\rightarrow 0}\frac{r_{f,x_0}^n(x_0+h)}{h^n}=0,
	\]
	lo cual indica que el resto $r_{f,x_0}^n(x_0+h)$ es mucho menor que $h^n$.
\end{frame}



\subsection{Estudio de funciones: Crecimiento, extremos y concavidad}

%---------------------------------------------------------------------slide----
\begin{frame}
	\frametitle{Estudio del crecimiento de una función}
	La principal aplicación de la derivada es el estudio del crecimiento de una función mediante el signo de la derivada.
	\begin{teorema}
		Si $f$ es una función cuya derivada existe en un intervalo $I$, entonces:
		\begin{itemize}
			\item Si $\forall x\in I\ f'(x)\geq 0$ entonces $f$ es creciente en el intervalo $I$.
			\item Si $\forall x\in I\ f'(x)\leq 0$ entonces $f$ es decreciente en el intervalo $I$.
		\end{itemize}
	\end{teorema}
	\structure{\textbf{Ejemplo}}
	La función $f(x)=x^3$ es creciente en todo $\mathbb{R}$ ya que $\forall x\in \mathbb{R}\
	f'(x)\geq 0$. \vskip .5cm
	\textbf{Observación}. \emph{Una función puede ser creciente o decreciente en un intervalo y no tener derivada.}
\end{frame}


%---------------------------------------------------------------------slide----
\begin{frame}
	\frametitle{Estudio del crecimiento de una función}
	\framesubtitle{Ejemplo}
	Consideremos la función $f(x)=x^4-2x^2+1$. Su derivada $f'(x)=4x^3-4x$ está definida en todo $\mathbb{R}$ y es continua.
	\begin{center}
		\scalebox{1}{\psset{unit=0.95,algebraic}
\begin{pspicture*}(-5,-4.5)(5,2.5)
\psaxes[labelFontSize=\scriptstyle,ticksize=-3pt 0,labelsep=2pt]{<->}(0,0)(-2.5,-2.1)(2.5,2.1)
\psplot[linecolor=blue]{-1.5}{1.5}{x^4-2*x^2+1}
\psplot[linecolor=red]{-1.2}{1.2}{4*x^3-4*x}
\footnotesize
\rput[r](-1.5,2){\textcolor{blue}{$f(x)=x^4-2x^2+1$}}
\rput[r](-1.5,-1.5){\textcolor{red}{$f'(x)=4x^3-4x$}}
\normalsize
\uncover<2->{
\psaxes[labelFontSize=\scriptstyle,ticksize=-3pt 0,labelsep=2pt]{<->}(0,-3)(-2.5,-3)(2.5,-3)
\footnotesize
\rput[r](-2,-2.5){Crecimiento $f(x)$}
\rput[r](-2,-3.7){Signo $f'(x)$}
}
\uncover<3->{
\rput[c](-1,-3.7){\textcolor{red}{$0$}}
\psline[linecolor=gray,linestyle=dashed]{-}(-1,-3.5)(-1,2)
\rput[c](0,-3.7){\textcolor{red}{$0$}}
\psline[linecolor=gray,linestyle=dashed]{-}(0,-3.5)(0,2)
\rput[c](1,-3.7){\textcolor{red}{$0$}}
\psline[linecolor=gray,linestyle=dashed]{-}(1,-3.5)(1,2)
}
\uncover<4->{
\rput[c](-1.5,-3.7){\textcolor{red}{$-$}}
\rput[c](-1.5,-2.5){\textcolor{blue}{$\downarrow$}}
}
\uncover<5->{
\rput[c](-0.5,-3.7){\textcolor{red}{$+$}}
\rput[c](-0.5,-2.5){\textcolor{blue}{$\uparrow$}}
}
\uncover<6->{
\rput[c](0.5,-3.7){\textcolor{red}{$-$}}
\rput[c](0.5,-2.5){\textcolor{blue}{$\downarrow$}}
}
\uncover<7->{
\rput[c](1.5,-3.7){\textcolor{red}{$+$}}
\rput[c](1.5,-2.5){\textcolor{blue}{$\uparrow$}}
}
\end{pspicture*}}
	\end{center}
\end{frame}


%---------------------------------------------------------------------slide----
\begin{frame}
	\frametitle{Determinación de extremos relativos de una función}
	Como consecuencia del resultado anterior, la derivada también sirve para determinar los extremos relativos de una función.
	\begin{teorema}[Criterio de la primera derivada]
		Sea $f$ es una función cuya derivada existe en un intervalo $I$, y sea $x_0\in I$ tal que $f'(x_0)=0$, entonces:
		\begin{itemize}
			\item Si existe un $\delta>0$ tal que $\forall x\in(x_0-\delta,x_0)\ f'(x)>0$ y $\forall x\in(x_0,x_0+\delta)\ f'(x)<0$ entonces $f$ tiene un \emph{máximo relativo} en $x_0$.
			\item Si existe un $\delta>0$ tal que $\forall x\in(x_0-\delta,x_0)\ f'(x)<0$ y $\forall x\in(x_0,x_0+\delta)\ f'(x)>0$ entonces $f$ tiene un \emph{mínimo relativo} en $x_0$.
			\item Si existe un $\delta>0$ tal que $\forall x\in(x_0-\delta,x_0)\ f'(x)>0$ y $\forall x\in(x_0,x_0+\delta)\ f'(x)>0$ entonces $f$ tiene un \emph{punto de inflexión creciente} en $x_0$.
			\item Si existe un $\delta>0$ tal que $\forall x\in(x_0-\delta,x_0)\ f'(x)<0$ y $\forall x\in(x_0,x_0+\delta)\ f'(x)<0$ entonces $f$ tiene un \emph{punto de inflexión decreciente} en $x_0$.
		\end{itemize}
	\end{teorema}
	Los puntos donde se anula la derivada de una función se denominan \emph{puntos críticos}.
	
	%\textbf{Observación}. \emph{La anulación de la derivada es una condición necesaria pero no suficiente para la existencia de un extremo relativo en un punto.}
	
	%\structure{\textbf{Ejemplo}} La función $f(x)=x^3$ tiene derivada $f'(x)=3x^2$ y por tanto tiene un punto crítico en $x=0$, pero no tiene un extremo relativo en el 0, sino un punto de inflexión.
\end{frame}


%---------------------------------------------------------------------slide----
\begin{frame}
	\frametitle{Determinación de extremos relativos de una función}
	\framesubtitle{Ejemplo}
	Consideremos de nuevo la función $f(x)=x^4-2x^2+1$.
	Su derivada $f'(x)=4x^3-4x$ está definida en todo $\mathbb{R}$ y es continua.
	\begin{center}
		\scalebox{1}{\psset{unit=0.95,algebraic}
\begin{pspicture*}(-5,-4.5)(5,2.5)
\psaxes[labelFontSize=\scriptstyle,ticksize=-3pt 0,labelsep=2pt]{<->}(0,0)(-2.5,-2.1)(2.5,2.1)
\psplot[linecolor=blue]{-1.5}{1.5}{x^4-2*x^2+1}
\psplot[linecolor=red]{-1.2}{1.2}{4*x^3-4*x}
\footnotesize
\rput[r](-1.5,2){\textcolor{blue}{$f(x)=x^4-2x^2+1$}}
\rput[r](-1.5,-1.5){\textcolor{red}{$f'(x)=4x^3-4x$}}
\normalsize
\psaxes[labelFontSize=\scriptstyle,ticksize=-3pt 0,labelsep=2pt]{<->}(0,-3)(-2.5,-3)(2.5,-3)
\psline[linecolor=gray,linestyle=dashed]{<-}(-1,-4)(-1,2)
\psline[linecolor=gray,linestyle=dashed]{<-}(0,-4)(0,2)
\psline[linecolor=gray,linestyle=dashed]{<-}(1,-4)(1,2)
\footnotesize
\rput[r](-2,-2.5){Crecimiento $f(x)$}
\rput[r](-2,-3.7){Signo $f'(x)$}
\rput[c](-1.5,-2.5){\textcolor{blue}{$\downarrow$}}
\rput[c](-1.5,-3.7){\textcolor{red}{$-$}}
\rput[c](-1,-3.7){\textcolor{red}{$0$}}
\rput[c](-0.5,-2.5){\textcolor{blue}{$\uparrow$}}
\rput[c](-0.5,-3.7){\textcolor{red}{$+$}}
\rput[c](0,-3.7){\textcolor{red}{$0$}}
\rput[c](0.5,-2.5){\textcolor{blue}{$\downarrow$}}
\rput[c](0.5,-3.7){\textcolor{red}{$-$}}
\rput[c](1,-3.7){\textcolor{red}{$0$}}
\rput[c](1.5,-2.5){\textcolor{blue}{$\uparrow$}}
\rput[c](1.5,-3.7){\textcolor{red}{$+$}}
\uncover<2->{
\rput[r](-2,-4.3){Extremos $f(x)$}
\rput[c](-1,-4.3){\textcolor{blue}{Mín}}
\rput[c](0,-4.3){\textcolor{blue}{Máx}}
\rput[c](1,-4.3){\textcolor{blue}{Mín}}
}
\end{pspicture*}}
	\end{center}
\end{frame}


%---------------------------------------------------------------------slide---
\begin{frame}
	\frametitle{Estudio de la concavidad de una función}
	La concavidad de una función puede estudiarse mediante el signo de la segunda derivada.
	\begin{teorema}[Criterio de la segunda derivada]
		Si $f$ es una función cuya segunda derivada existe en un intervalo $I$, entonces:
		\begin{itemize}
			\item Si $\forall x\in I\ f''(x)\geq 0$ entonces $f$ es cóncava en el intervalo $I$.
			\item Si $\forall x\in I\ f''(x)\leq 0$ entonces $f$ es convexa en el intervalo $I$.
		\end{itemize}
	\end{teorema}
	
	\structure{\textbf{Ejemplo}} La función $f(x)=x^2$ tiene segunda derivada $f''(x)=2>0$ y por tanto es cóncava en todo $\mathbb{R}$.
	\vskip .5cm
	\textbf{Observación}. \emph{Una función puede ser cóncava o convexa en un intervalo y no tener derivada.}
\end{frame}


%---------------------------------------------------------------------slide----
\begin{frame}
	\frametitle{Estudio de la concavidad de una función}
	\framesubtitle{Ejemplo}
	Consideremos de nuevo la función $f(x)=x^4-2x^2+1$. Su segunda derivada $f''(x)=12x^2-4$ está definida en todo $\mathbb{R}$ y es continua.
	\begin{center}
		\scalebox{1}{\psset{unit=0.8,algebraic}
\begin{pspicture*}(-5,-6)(5,2.1)
\psaxes[labelFontSize=\scriptstyle,ticksize=-3pt 0,labelsep=2pt]{<->}(0,0)(-2.5,-4.1)(2.5,2.1)
\psplot[linecolor=blue]{-1.5}{1.5}{x^4-2*x^2+1}
\psplot[linecolor=red]{-1.2}{1.2}{4*x^3-4*x}
\psplot[linecolor=green]{-1.2}{1.2}{12*x^2-4}
\footnotesize
\rput[r](-1.5,1.9){\textcolor{blue}{$f(x)=x^4-2x^2+1$}}
\rput[r](-1.5,-1.5){\textcolor{red}{$f'(x)=4x^3-4x$}}
\rput[l](1,-2){\textcolor{green}{$f''(x)=12x^2-4$}}
\uncover<2->{
\psaxes[ticksize=-3pt 0,labelsep=2pt]{<->}(0,-5)(-2.5,-5)(2.5,-5)
\rput[r](-2,-4.5){Concavidad $f(x)$}
\rput[r](-2,-5.8){Signo $f''(x)$}
}
\uncover<3->{
\rput[c](-0.5773,-5.8){\textcolor{green}{$0$}}
\psline[linecolor=gray,linestyle=dashed](-0.5773,-5.5)(-0.5773,2)
\rput[c](0.5773,-5.8){\textcolor{green}{$0$}}
\psline[linecolor=gray,linestyle=dashed](0.5773,-5.5)(0.5773,2)
}
\uncover<4->{
\rput[c](-1.2,-5.8){\textcolor{green}{$+$}}
\rput[c](-1.2,-4.5){\textcolor{blue}{$\cup$}}
}
\uncover<5->{
\rput[c](0,-5.8){\textcolor{green}{$-$}}
\rput[c](0,-4.5){\textcolor{blue}{$\cap$}}
}
\uncover<6->{
\rput[c](1.2,-5.8){\textcolor{green}{$+$}}
\rput[c](1.2,-4.5){\textcolor{blue}{$\cup$}}
}
\end{pspicture*}}
	\end{center}
\end{frame}



%
% \subsection{El concepto de diferencial}
% %---------------------------------------------------------------------slide----
% \begin{frame}
% \frametitle{El concepto de diferencial}
% \begin{definicion}[Diferencial de una función en un punto]
% Dada una función $f$, se llama \emph{diferencial} de $f$ en un punto $a$, al la función
% \[
% \begin{array}{rccc}
% dy=df(a): & \mathbb{R} & \longrightarrow & \mathbb{R} \\
% & \Delta x & \longrightarrow & f'(a)\Delta x
% \end{array}
% \]
% \end{definicion}
%
% Cuando $f$ es la función identidad $y=x$,  entonces $f'(a)=1$, y se cumple que
% \[ dx=dy=f'(a)\Delta x=\Delta x,\]
% de modo que también podemos definir el diferencial como
% \[dy=df(a)=f'(a)dx.\]
%
% De aquí se deduce otra forma de escribir la derivada de $f$ en $a$
% \[f'(a)=\frac{dy}{dx}=\frac{df(a)}{dx}.\]
% \end{frame}
%
%
% %---------------------------------------------------------------------slide----
% \begin{frame}
% \frametitle{Aproximación de una función mediante su diferencial}
% El diferencial de una función $f$ en un punto $a$, permite aproximar la variación de $f$ cerca de $a$.
%
% \begin{center}
% \scalebox{1}{\psset{xunit=2,yunit=1.2,algebraic}
\begin{pspicture*}(-1.2,-0.5)(4.3,4)
\psaxes[ticks=none,labels=none]{<->}(0,0)(-0.3,-0.5)(2,4)
\psplot[linecolor=blue]{0.2}{1.6}{x^3-x^2+x+0.5}
\rput[r](1.4,3.5){$f(x)$}
\psxTick[ticksize=-3pt 0,labelsep=3pt](0.5){a}
\psyTick[ticksize=-3pt 0,labelsep=3pt](0.875){f(a)}
\psline[linewidth=0.5pt,linestyle=dashed,linecolor=gray](0.5,0)(0.5,0.875)(0,0.875)
\uncover<2->{
\psxTick[ticksize=-3pt 0,labelsep=3pt](1.5){a+dx}
\psline[linewidth=0.5pt,linestyle=dashed,linecolor=gray](1.5,0)(1.5,3.125)(0,3.125)
\psyTick[ticksize=-3pt 0,labelsep=3pt](3.125){f(a+dx)}
\psline[arrows=|*-|*,linecolor=green](0.5,0.875)(1.5,0.875)
\rput[t](1,0.8){$dx$}
}
\uncover<3->{
\psline[linestyle=dotted,linecolor=gray](1.5,0.875)(2.6,0.875)
\psline[linestyle=dotted,linecolor=gray](1.5,3.125)(2.6,3.125)
\psline[arrows=|*-|*,linecolor=orange](2.6,0.875)(2.6,3.125)
\rput[l](2.7,1.9){$\Delta y=f(a+dx)-f(a)$}
}
\uncover<4->{
\psplot[linecolor=red]{0.2}{1.8}{0.75*x+0.5}
}
\uncover<5->{
\psline[linewidth=0.5pt,linestyle=dashed,linecolor=gray](1.5,1.625)(0,1.625)
\rput[r](-0.1,1.625){$f(a)+f'(a)dx$}
}
\uncover<6->{
\psline[arrows=|*-|*,linecolor=green](1.5,1.625)(1.5,0.875)
\rput[l](1.6,1.25){$dy=f'(a)dx$}
\rput[r](-0.4,2.4){\rotatebox{90}{$\approx$}}
}
\end{pspicture*}}
% \end{center}
% \end{frame}
%
%
% %---------------------------------------------------------------------slide----
% \begin{frame}
% \frametitle{Aproximación de una función mediante su diferencial: Ejemplo}
% Consideremos otra vez la función $y=x^2$ que mide el área de un cuadrado de chapa metálica de lado $x$.
%
% Si el lado del cuadrado es $a$, y sometemos la chapa a un proceso de calentamiento que aumenta el lado del cuadrado, ¿cuál  será aproximadamente la variación que habrá experimentado el área, cuando el lado aumente una cantidad $dx$?
% \begin{columns}
% \begin{column}{0.3\textwidth}
% \begin{align*}
% \Delta y &= f(a+dx)-f(a)=(a+dx)^2-a^2=\\
% &= a^2+2adx+dx^2-a^2=2adx+dx^2,\\
% \only<2->{dy &= f'(a)dx= 2adx.}
% \end{align*}
% \uncover<3->{Además, \[\lim_{dx\rightarrow 0}\Delta y-dy=\lim_{dx\rightarrow 0}dx^2=0.\]}
% \end{column}
% \begin{column}{0.3\textwidth}
% \begin{center}
% \scalebox{1}{\psset{unit=0.6}
\begin{pspicture*}(0,-0.5)(4,4)
\footnotesize
\psframe(0,0)(3,3)
\rput[t](1.5,-0.1){$a$}
\rput[t](1.5,1.7){$a^2$}
\psframe[fillstyle=solid,fillcolor=gray](0,3)(3,4)
\psframe[fillstyle=solid,fillcolor=gray](3,3)(4,4)
\psframe[fillstyle=solid,fillcolor=gray](3,0)(4,3)
\rput[t](3.5,-0.1){$dx$}
\rput[t](1.5,3.7){$adx$}
\rput[t](3.5,3.7){$dx^2$}
\rput[t](3.5,1.7){$adx$}
\uncover<2->{
\psframe[fillstyle=solid,fillcolor=orange](0,3)(3,4)
\psframe[fillstyle=solid,fillcolor=orange](3,0)(4,3)
\rput[t](1.5,3.7){$adx$}
\rput[t](3.5,1.7){$adx$}
}
\end{pspicture*}}
% \end{center}
% \end{column}
% \end{columns}
% \end{frame}


% %---------------------------------------------------------------------slide----
% \begin{frame}
% \frametitle{Propiedades del diferencial}
% Si $y=c$, es una función constante, entonces $dy=0$.
% Si $y=x$, es la función identidad, entonces  $dy=dx$.
%
% Si $u=f(x)$ y $v=g(x)$ son dos funciones diferenciables, entonces
% \begin{itemize}
% \item $d(u+v)=d(u)+d(v)$
% \item $d(u-v)=d(u)-d(v)$
% \item $d(u\cdot v)=d(u)\cdot v+ u\cdot d(v)$
% \item $d\left(\dfrac{u}{v}\right)=\dfrac{du\cdot v-u\cdot dv}{v^2}$
% \end{itemize}
% \end{frame}





% \subsection{Derivada de una función implícita}
% %---------------------------------------------------------------------slide----
% \begin{frame}
% \frametitle{Derivada de una función implícita}
% Si $F(x,y)=0$ es una función implícita entonces
% \[
% dF(x,y)=d0=0.
% \]
%
% Si $F(x,y)=0$ es una función implícita en la que $y$ depende de $x$, entonces podemos calcular la derivada de $y$ a partir del diferencial
% \[
% \frac{dF(x,y)}{dx}=\frac{d0}{dx}=0.
% \]
% \structure{\textbf{Ejemplo}}. Consideremos la función implícita de la circunferencia de radio 1, $x^2+y^2=1$. Entonces su diferencial es
% \[
% d(x^2+y^2)=d1=0 \Leftrightarrow d(x^2)+d(y^2)=2x\;dx+2y\;dy=0.
% \]
% A partir de aquí podemos calcular fácilmente la derivada de $y$:
% \[
% \frac{d(x^2+y^2)}{dx}= \frac{2x\;dx+2y\;dy}{dx}=2x\frac{dx}{dx}+2y\frac{dy}{dx}= 2x+2y\frac{dy}{dx}=0 \Leftrightarrow \frac{dy}{dx}=\frac{-x}{y}.
% \]
%
% \end{frame}
%
%
% \subsection{Derivada de una función paramétrica}
% %---------------------------------------------------------------------slide----
% \begin{frame}
% \frametitle{Derivada de una función parametrica}
%
% Dada una función paramétrica
% \[
% \left\{%
% \begin{array}{l}
% x=f(t) \\
% y=g(t)
% \end{array}%
% \right.
% \]
% podemos calcular su derivada a partir de las derivadas de $f$ y $g$:
% \[\frac{dy}{dx}=\frac{g'(t)\,dt}{f'(t)\,dt}=\frac{g'(t)}{f'(t)}.\]
%
% \structure{\textbf{Ejemplo}}. Consideremos la elipse
% \[
% \left\{%
% \begin{array}{l}
% x=2\sen t \\
% y=\cos t
% \end{array}%
% \right.
% \]
% Entonces
% \[
% \frac{dy}{dx}=\frac{-\sen t\; dt}{2 \cos t\; dt}=\frac{-1}{2}\tg t.
% \]
% \end{frame}
