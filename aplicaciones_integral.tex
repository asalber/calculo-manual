%Version control information
%$HeadURL: https://ejercicioscalculo.googlecode.com/svn/trunk/compendio_ejercicios_calculo.tex $} {$LastChangedDate: 2008-07-09 16:26:20 +0200 (mi�, 09 jul 2008) $
%$LastChangedRevision: 6 $
%$LastChangedBy: asalber $


\section{Aplicaciones de la Integral}
%---------------------------------------------------------------------slide----
\begin{frame}
\frametitle{Aplicaciones de la Integral}
\tableofcontents[sectionstyle=show/hide,hideothersubsections]
\end{frame}



\subsection{Integral definida}
%---------------------------------------------------------------------slide----
\begin{frame}
\frametitle{Integral definida}
\begin{definicion}[Integral definida]
Sea $f(x)$ una función cuya primitiva es $F(x)$ en un intervalo $[a,b]$. Se define la integral definida de $f(x)$ en el intervalo $[a,b]$ como
\[
\int_a^b f(x)\,dx = \left[F(x)\right]_a^b=F(b)-F(a)
\]
\end{definicion}
Ejemplo. Dada la función $f(x)=x^2$, se tiene
\[
\int_1^2 x^2\,dx = \left[\frac{x^3}{3}\right]_1^2 = \frac{2^3}{3}-\frac{1^3}{3} = \frac{7}{3}.
\]
\end{frame}


%---------------------------------------------------------------------slide----
\begin{frame}
\frametitle{Propiedades de la integral definida}
Dadas dos funciones $f$ y $g$ integrables en $[a,b]$ y $k \in \mathbb{R}$ se cumplen las siguientes propiedades
\begin{itemize}
\item $\int_{a}^{b}(f(x)+g(x))\,dx=\int_{a}^{b}f(x)\,dx+\int_{a}^{b}g(x)\,dx$ (linealidad)
\item $\int_{a}^{b}{kf(x)}\,dx=k\int_{a}^{b}{f(x)}\,dx$ (linealidad)
\item $\int_{a}^{b}{f(x)\,dx} \leq \int_{a}^{b}{g(x)\,dx}$ si $f(x)\leq g(x)\ \forall x \in [a,b]$ (monotonía)
\item $\int_{a}^{b}{f(x)\,dx} = \int_{a}^{c}{f(x)\,dx}+\int_{c}^{b}{f(x)\,dx}$ para cualquier $c\in(a,b)$ (aditividad)
\item $\int_a^b f(x)\,dx = -\int_b^a f(x)\,dx$
\end{itemize}
\end{frame}



\subsection{Cálculo de áreas}
%---------------------------------------------------------------------slide----
\begin{frame}
\frametitle{Cálculo de áreas}
\framesubtitle{Área delimitada por una función positiva y el eje de abscisas}
Si $f$ es una función integrable en un intervalo $[a,b]$ y $f(x)\geq 0\ \forall x\in[a,b]$, entonces la integral definida
\[\int_a^b f(x)\,dx\]
mide le área que queda entre la función $f$ y el eje de abscisas en el intervalo $[a,b]$.
\begin{center}
\scalebox{1}{\psset{yunit=0.8,algebraic}
\begin{pspicture*}(-0.5,-0.5)(4.5,4.5)
\psaxes[ticks=none,labels=none]{<->}(0,0)(-0.5,-0.5)(4.5,4.5)
\pscustom[linecolor=blue]{%
\psplot{1}{3}{x^3-6*x^2+11*x-3}
\gsave
\psline(3,0)(1,0)
\fill[fillstyle=solid,fillcolor=yellow]
\grestore
\psplot{3}{3.2}{x^3-6*x^2+11*x-3}}
\psplot[linecolor=blue]{0.6}{1}{x^3-6*x^2+11*x-3}
\psline[linecolor=gray](1,-0.1)(1,3)
\psline[linecolor=gray](3,-0.1)(3,3)
\psxTick[ticksize=-3pt 0,labelsep=3pt](1){a}
\psxTick[ticksize=-3pt 0,labelsep=3pt](3){b}
\rput[c](2,1.5){$\displaystyle \int_a^b f(x)\,dx$}
\rput[l](3.5,3.5){$f(x)$}
\end{pspicture*}}
\end{center}
\end{frame}


%---------------------------------------------------------------------slide----
\begin{frame}
\frametitle{Cálculo de áreas}
\framesubtitle{Área delimitada por una función negativa y el eje de abscisas}
Si $f$ es una función integrable en un intervalo $[a,b]$ y $f(x)\leq 0\ \forall x\in[a,b]$, entonces el área que queda entre la función $f$ y el eje de abscisas en el intervalo $[a,b]$ es
\[
-\int_a^b f(x)\,dx.
\]
\begin{center}
\scalebox{1}{\psset{yunit=0.8,algebraic}
\begin{pspicture*}(-0.5,-4.5)(4.5,0.5)
\psaxes[ticks=none,labels=none]{<->}(0,0)(-0.5,-4.5)(4.5,0.5)
\pscustom[linecolor=blue]{%
\psplot{1}{3}{-x^3+6*x^2-11*x+3}
\gsave
\psline(3,0)(1,0)
\fill[fillstyle=solid,fillcolor=color1light]
\grestore
\psplot{3}{3.2}{-x^3+6*x^2-11*x+3}}
\psplot[linecolor=blue]{0.6}{1}{-x^3+6*x^2-11*x+3}
\psline[linecolor=gray](1,0.1)(1,-3)
\psline[linecolor=gray](3,0.1)(3,-3)
\psxTick[ticksize=3pt 0,labelsep=-14pt](1){a}
\psxTick[ticksize=3pt 0,labelsep=-14pt](3){b}
\rput[c](2,-1.5){$-\displaystyle \int_a^b f(x)\,dx$}
\rput[l](3.5,-3.5){$f(x)$}
\end{pspicture*}}
\end{center}
\end{frame}


%---------------------------------------------------------------------slide----
\begin{frame}
\frametitle{Cálculo de áreas}
\framesubtitle{Área delimitada por una función y el eje de abscisas}
Si $f$ cambia de signo a lo largo del intervalo $[a,b]$ entonces se divide el intervalo de integración en intervalos donde $f$ tenga el mismo signo, se calcula cada área por separado y se suman.
\begin{center}
\scalebox{1}{\psset{yunit=0.8,xunit=1.5,algebraic}
\begin{pspicture*}(-0.5,-2.5)(4,2.5)
\psaxes[ticks=none,labels=none]{<->}(0,0)(-0.5,-2.5)(4,2.5)
\pscustom[linecolor=blue]{%
\psplot{1}{3}{x^3-6*x^2+9*x-2}
\gsave
\psline(3,0)(1,0)
\fill[fillstyle=solid,fillcolor=color1light]
\grestore
\psplot{3}{3.2}{x^3-6*x^2+9*x-2}}
\psplot[linecolor=blue]{0.6}{1}{x^3-6*x^2+9*x-2}
\psline[linecolor=gray](1,-0.1)(1,2)
\psline[linecolor=gray](3,0.1)(3,-2)
\psline[linecolor=gray](2,0.1)(2,-0.1)
\psxTick[ticksize=-3pt 0,labelsep=3pt](1){a}
\psxTick[ticksize=3pt 0,labelsep=-14pt](3){b}
\psxTick[ticksize=-3pt 0,labelsep=3pt](2){c}
\rput[l](1,0.5){\small$\int_a^c f(x)\,dx$}
\rput[r](3,-0.5){\small$-\int_c^b f(x)\,dx$}
\rput[l](3.5,-1.5){$f(x)$}
\end{pspicture*}}
\end{center}
\[\int_a^c f(x)\,dx -\int_c^b f(x)\,dx\]
\end{frame}


%---------------------------------------------------------------------slide----
\begin{frame}
\frametitle{Cálculo de áreas}
\framesubtitle{Área delimitada por dos funciones}
Si $f$ y $g$ son dos funciones integrables en el intervalo $[a,b]$ y se
verifica que $g(x)\leq f(x)$ $\forall x\in[a,b]$, entonces el área de la región
plana limitada por las curvas $y=f(x)$, $y=g(x)$, y las rectas $x=a$ y $x=b$
viene dada por
\[
\int_{a}^{b}{(f(x)- g(x))\,dx}.
\]

\begin{center}
\scalebox{1}{\psset{yunit=0.9,xunit=1.3,algebraic}
\begin{pspicture*}(-0.5,-0.5)(4.5,4.5)
\psaxes[ticks=none,labels=none]{<->}(0,0)(-0.5,-0.5)(4.5,4.5)
\pscustom[linecolor=blue]{%
\psplot{1}{3}{x^3-6*x^2+11*x-3}
\gsave
\psline(3,0)
\psplot{3}{1}{-x^3+6*x^2-11*x+7}
\fill[fillstyle=solid,fillcolor=yellow]
\grestore
\psplot{3}{3.2}{x^3-6*x^2+11*x-3}}
\psplot[linecolor=blue]{0.6}{1}{x^3-6*x^2+11*x-3}
\psplot[linecolor=blue]{0.6}{3.2}{-x^3+6*x^2-11*x+7}
\psline[linecolor=gray](1,-0.1)(1,3)
\psline[linecolor=gray](3,-0.1)(3,3)
\psxTick[ticksize=-3pt 0,labelsep=3pt](1){a}
\psxTick[ticksize=-3pt 0,labelsep=3pt](3){b}
\rput[c](2,2){$\displaystyle \int_a^b f(x)-g(x)\,dx$}
\rput[l](3.5,3.5){$f(x)$}
\rput[l](3.5,0.5){$g(x)$}
\end{pspicture*}}
\end{center}
\end{frame}
