%Version control information
%$HeadURL: https://ejercicioscalculo.googlecode.com/svn/trunk/compendio_ejercicios_calculo.tex $} {$LastChangedDate: 2008-07-09 16:26:20 +0200 (mi�, 09 jul 2008) $
%$LastChangedRevision: 6 $
%$LastChangedBy: asalber $

\section{Límites y Continuidad}

%---------------------------------------------------------------------slide----
\begin{frame}
\frametitle{Límites y continuidad}
\tableofcontents[sectionstyle=show/hide,hideothersubsections]
\end{frame}



\subsection{El concepto de límite}
%---------------------------------------------------------------------slide----
\begin{frame}
\frametitle{Aproximación al concepto de límite}
El concepto de límite está ligado al de tendencia.

Decimos que $x$ \emph{tiende} a un valor $a$, y lo escribimos $x\rightarrow a$, si se pueden tomar valores de $x$ tan próximos a $a$ como se quiera, pero sin llegar a valer $a$.

Si la aproximación es por defecto (con valores menores que $a$) se dice que $x$ tiende a $a$ por la izquierda, y se escribe $x\rightarrow a^-$, y si es por exceso (con valores mayores que $a$) se dice que $x$ tiende a $a$ por la derecha, y se escribe $x\rightarrow a^+$.

Cuando la variable $x$ de una función $f$ tiende a un valor $a$, cabe preguntarse si sus imágenes mediante $f$ tienden a otro valor concreto:
\begin{center}
\alert{¿A donde se aproxima $f(x)$ cuando $x$ se aproxima a $a$?}
\end{center}

Si $f(x)$ tiende a un valor $l$ cuando $x$ tiende a $a$, se dice que $l$ es el \emph{límite} de $f(x)$ cuando $x\rightarrow a$, y se escribe
\[\lim_{x\rightarrow a}f(x)=l.\]
\end{frame}



%---------------------------------------------------------------------slide----
\begin{frame}
\frametitle{Límites laterales}

 Si $f(x)$ tiende a $l$ cuando $x$ tiende a $a$ por la izquierda, entonces se dice que $l$ es el \emph{límite por la izquierda} de $f(x)$ cuando $x\rightarrow a^-$, y se escribe
\[\lim_{x\rightarrow a^-}f(x)=l.\]

 Si $f(x)$ tiende a $l$ cuando $x$ se aproxima a $a$ por exceso, entonces se dice que $l$ es el \emph{límite por la derecha} de $f(x)$ cuando $x\rightarrow a^-$, y se escribe
\[\lim_{x\rightarrow a^+}f(x)=l.\]

\begin{center}
\alert{Para que exista el límite deben existir los límites laterales y ser iguales.}
\end{center}
\[
\footnotesize
\begin{array}{c}
\underbrace{\begin{array}{ccc}
\textrm{Aproximación por defecto} & \qquad & \textrm{Aproximación por exceso}\\
\begin{array}{|l|l|}
\hline
\multicolumn{1}{|c|}{x}      & \multicolumn{1}{c|}{f(x)=x^2}   \\
\hline\hline
 1.9    & 3.61       \\
\hline
 1.99   & 3.9601     \\
\hline
 1.999  & 3.996001   \\
\hline
 1.9999 & 3.99960001 \\
\hline
\end{array}
& &
\begin{array}{|l|l|}
\hline
\multicolumn{1}{|c|}{x}      & \multicolumn{1}{c|}{f(x)=x^2}   \\
\hline\hline
 2.1    & 4.41       \\
\hline
 2.01   & 4.0401    \\
\hline
 2.001  & 4.004001   \\
\hline
 2.0001 & 4.00040001 \\
\hline
\end{array}\\
\Downarrow & & \Downarrow\\
\lim_{x\rightarrow 2^-}x^2=4
& &
\lim_{x\rightarrow 2^+}x^2=4
\end{array}}\\
\Downarrow\\
\lim_{x\rightarrow 2}x^2=4
\end{array}
\]
\end{frame}


%---------------------------------------------------------------------slide----
\begin{frame}
\frametitle{Límites que no existen (I)}
Si la función no está definida entorno a un punto, entonces no existe el límite en dicho punto

\structure{\textbf{Ejemplo}} Consideremos la función $f(x)=\dfrac{1}{\sqrt{x^2-1}}$ y veamos que pasa cuando $x\rightarrow 0$:
\begin{columns}
\begin{column}{0.6\textwidth}
\[
\footnotesize
\begin{array}{c}
\underbrace{\begin{array}{ccc}
\textrm{Por la izquierda} & \qquad & \textrm{Por la derecha }\\
\begin{array}{|l|l|}
\hline
\multicolumn{1}{|c|}{x}      & \multicolumn{1}{c|}{f(x)}   \\
\hline\hline
 -0.1   & \textrm{No exite}      \\
\hline
 -0.01   & \textrm{No existe}     \\
\hline
 -0.001  & \textrm{No existe}   \\
\hline
\end{array}
& &
\begin{array}{|l|l|}
\hline
\multicolumn{1}{|c|}{x}      & \multicolumn{1}{c|}{f(x)}   \\
\hline\hline
 0.1    &  \textrm{No existe}      \\
\hline
 0.01   & \textrm{No existe}    \\
\hline
 0.001  & \textrm{No existe}   \\
\hline
\end{array}\\
\Downarrow & & \Downarrow\\
\displaystyle \textrm{No existe } \lim_{x\rightarrow 0^-}\frac{1}{\sqrt{x^2-1}}
& &
\displaystyle \textrm{No existe } \lim_{x\rightarrow 0^+}\frac{1}{\sqrt{x^2-1}}
\end{array}}\\
\Downarrow\\
\displaystyle \textrm{No existe }\lim_{x\rightarrow 0}\frac{1}{\sqrt{x^2-1}}
\end{array}
\]
\end{column}
\begin{column}{0.4\textwidth}
\begin{center}
\scalebox{1}{\psset{unit=0.7,algebraic}
\begin{pspicture*}(-3,-1)(3,4)
\psaxes[labelFontSize=\scriptstyle,ticksize=-3pt 0,labelsep=2pt]{<->}(0,0)(-3,-0.5)(3,4)
\psplot[linecolor=blue]{-3}{-1.00001}{1/sqrt(x^2-1)}
\psplot[linecolor=blue]{1.00001}{3}{1/sqrt(x^2-1)}
\end{pspicture*}}
\end{center}
\end{column}
\end{columns}
\end{frame}


%---------------------------------------------------------------------slide----
\begin{frame}
\frametitle{Límites que no existen (II)}
Cuando los límites laterales no coinciden entonces no existe el límite

\structure{\textbf{Ejemplo}} Consideremos la función $f(x)=\dfrac{|x|}{x}$ y veamos que pasa cuando $x\rightarrow 0$:
\begin{columns}
\begin{column}{0.6\textwidth}
\[
\begin{array}{c}
\underbrace{\begin{array}{ccc}
\textrm{Por la izquierda} & \qquad & \textrm{Por la derecha }\\
\begin{array}{|l|l|}
\hline
\multicolumn{1}{|c|}{x}      & \multicolumn{1}{c|}{f(x)}   \\
\hline\hline
 -0.1   & -1       \\
\hline
 -0.01   & -1     \\
\hline
 -0.001  & -1   \\
\hline
\end{array}
& &
\begin{array}{|l|l|}
\hline
\multicolumn{1}{|c|}{x}      & \multicolumn{1}{c|}{f(x)}   \\
\hline\hline
 0.1    & 1       \\
\hline
 0.01   & 1    \\
\hline
 0.001  & 1   \\
\hline
\end{array}\\
\Downarrow & & \Downarrow\\
\displaystyle \lim_{x\rightarrow 0^-}\frac{|x|}{x}=-1
&\neq &
\displaystyle \lim_{x\rightarrow 0^+}\frac{|x|}{x}=1
\end{array}}\\
\Downarrow\\
\displaystyle \textrm{No existe }\lim_{x\rightarrow 0}\frac{|x|}{x}
\end{array}
\]
\end{column}
\begin{column}{0.4\textwidth}
\begin{center}
\scalebox{1}{\psset{unit=1}
\begin{pspicture*}(-2,-1.5)(2,1.5)
\psaxes[labelFontSize=\scriptstyle,ticksize=-3pt 0,labelsep=2pt]{<->}(0,0)(-2,-1.5)(2,1.5)
\psplot[linecolor=blue]{-2}{0}{-1}
\psplot[linecolor=blue]{0}{2}{1}
\end{pspicture*}}
\end{center}
\end{column}
\end{columns}
\end{frame}

%---------------------------------------------------------------------slide----
\begin{frame}
\frametitle{Límites que no existen (III)}
A veces, cuando $x\rightarrow a$ los valores de $f(x)$ crecen o decrecen infinitamente y entonces no existe el límite. En este caso se dice que la función \emph{diverge} y se escribe
\[\lim_{x\rightarrow a}f(x)=\pm \infty\]

\structure{\textbf{Ejemplo}} Veamos la tendencia de la función $f(x)=\dfrac{1}{x^2}$ cuando $x\rightarrow 0$:
\begin{columns}
\begin{column}{0.6\textwidth}
\[
\footnotesize
\begin{array}{c}
\underbrace{\begin{array}{ccc}
\textrm{Por la izquierda} & \qquad & \textrm{Por la derecha }\\
\begin{array}{|l|r|}
\hline
\multicolumn{1}{|c|}{x}      & \multicolumn{1}{c|}{f(x)}   \\
\hline\hline
 -0.1   & 100       \\
\hline
 -0.01   & 10000     \\
\hline
 -0.001  & 1000000   \\
\hline
\end{array}
& &
\begin{array}{|l|r|}
\hline
\multicolumn{1}{|c|}{x}      & \multicolumn{1}{c|}{f(x)}   \\
\hline\hline
 0.1    & 100       \\
\hline
 0.01   & 10000    \\
\hline
 0.001  & 1000000   \\
\hline
\end{array}\\
\Downarrow & & \Downarrow\\
\displaystyle \lim_{x\rightarrow 0^-}\frac{1}{x^2}=+\infty
& &
\displaystyle \lim_{x\rightarrow 0^+}\frac{1}{x^2}=+\infty
\end{array}}\\
\Downarrow\\
\displaystyle \textrm{No existe }\lim_{x\rightarrow 0}\frac{1}{x^2}=\infty
\end{array}
\]
\end{column}
\begin{column}{0.4\textwidth}
\begin{center}
\scalebox{1}{\psset{unit=0.7,algebraic}
\begin{pspicture*}(-3,-1)(3,4)
\psaxes[labelFontSize=\scriptstyle,ticksize=-3pt 0,labelsep=2pt]{<->}(0,0)(-3,-1)(3,4)
\psplot[linecolor=blue]{-3}{-0.01}{1/x^2}
\psplot[linecolor=blue]{0.01}{3}{1/x^2}
\end{pspicture*}}
\end{center}
\end{column}
\end{columns}
\end{frame}


%---------------------------------------------------------------------slide----
\begin{frame}
\frametitle{Límites que no existen (IV)}
A veces, el límite de un función en un punto puede no existir porque la función oscila rápidamente al acercarnos a dicho punto.

\structure{\textbf{Ejemplo}} Consideremos la función $f(x)=\sen \dfrac{1}{x}$ y veamos que pasa cuando $x\rightarrow 0$:
\begin{columns}
\begin{column}{0.6\textwidth}
\footnotesize
\[
\begin{array}{ccc}
\textrm{Por la izquierda} & \qquad & \textrm{Por la derecha }\\
\begin{array}{|l|l|}
\hline
\multicolumn{1}{|c|}{x}      & \multicolumn{1}{c|}{f(x)}   \\
\hline\hline
 -0.1   & -0.1736       \\
\hline
 -0.01   & -0.9848     \\
\hline
-0.005 & 0.3420 \\
\hline
 -0.001  & 0.9848  \\
\hline
-0.0005  & 0.3420\\
\hline
-0.0001 & 0.9848 \\
\hline
\end{array}
& &
\begin{array}{|l|l|}
\hline
\multicolumn{1}{|c|}{x}      & \multicolumn{1}{c|}{f(x)}   \\
\hline\hline
 0.1   & 0.1736       \\
\hline
 0.01   & 0.9848     \\
\hline
0.005 & -0.3420 \\
\hline
 0.001  & -0.9848  \\
\hline
0.0005  & -0.3420\\
\hline
0.0001 & -0.9848 \\
\hline
\end{array}\\
\Downarrow & & \Downarrow\\
\displaystyle \textrm{No existe }\lim_{x\rightarrow 0^-}\sen \frac{1}{x}
& &
\displaystyle \textrm{No existe }\lim_{x\rightarrow 0^+}\sen \frac{1}{x}
\end{array}
\]
\end{column}
\begin{column}{0.4\textwidth}
\begin{center}
\scalebox{1}{\psset{xunit=2,algebraic}
\begin{pspicture*}(-1,-1.5)(1,1.5)
\psaxes[labelFontSize=\scriptstyle,ticksize=-3pt 0,labelsep=2pt]{<->}(0,0)(-1,-1.5)(1,1.5)
\psplot[linecolor=blue,plotpoints=1000]{-1}{1}{sin(1/x)}
\end{pspicture*}}
\end{center}
\end{column}
\end{columns}
\end{frame}


%---------------------------------------------------------------------slide----
\begin{frame}
\frametitle{Límites en el infinito}
Si $f(x)$ tiende a $l$ cuando $x$ crece infinitamente, entonces se dice que $l$ es el \emph{límite en el infinito} de $f(x)$ cuando $x\rightarrow +\infty$, y se escribe
\[\lim_{x\rightarrow +\infty}f(x)=l.\]

Si $f(x)$ tiende a $l$ cuando $x$ decrece infinitamente, entonces se dice que $l$ es el \emph{límite en el infinito} de $f(x)$ cuando $x\rightarrow -\infty$, y se escribe
\[\lim_{x\rightarrow -\infty}f(x)=l.\]

\structure{\textbf{Ejemplo}} Estudiemos la tendencia de $f(x)=\dfrac{1}{x}$ cuando $x\rightarrow \pm\infty$:
\begin{columns}
\begin{column}{0.6\textwidth}
\[
\footnotesize
\begin{array}{ccc}
x\rightarrow +\infty & \qquad & x\rightarrow -\infty\\
\begin{array}{|r|l|}
\hline
\multicolumn{1}{|c|}{x}      & \multicolumn{1}{c|}{f(x)=1/x}   \\
\hline\hline
 1000    & 0.001       \\
\hline
 10000   & 0.0001     \\
\hline
 100000  & 0.00001   \\
\hline
\end{array}
& &
\begin{array}{|r|l|}
\hline
\multicolumn{1}{|c|}{x}      & \multicolumn{1}{c|}{f(x)=1/x}   \\
\hline\hline
 -1000    & -0.001       \\
\hline
 -10000   & -0.0001     \\
\hline
 -100000  & -0.00001   \\
\hline
\end{array}\\
\Downarrow & & \Downarrow\\
\lim_{x\rightarrow +\infty}\frac{1}{x}=0
& &
\lim_{x\rightarrow -\infty}\frac{1}{x}=0
\end{array}
\]
\end{column}
\begin{column}{0.4\textwidth}
\begin{center}
\scalebox{1}{\psset{unit=0.5,algebraic}
\begin{pspicture*}(-4,-3)(4,3)
\psaxes[labelFontSize=\scriptstyle,ticksize=-3pt 0,labelsep=2pt]{<->}(0,0)(-4,-3)(4,3)
\psplot[linecolor=blue]{-3.5}{-0.0001}{1/x}
\psplot[linecolor=blue]{0.0001}{3.5}{1/x}
\end{pspicture*}}
\end{center}
\end{column}
\end{columns}
\end{frame}


%---------------------------------------------------------------------slide----
\begin{frame}
\frametitle{Definición de límite}
\begin{definicion}[Límite de una función en un punto]
Se dice que el límite de la función $f$ cuando $x\rightarrow a$ es $l$, y se escribe
\[\lim_{x\rightarrow a} f(x) =l\]
si para cualquier valor $\varepsilon>0$ existe un número $\delta>0$ tal que, $|f(x)-l|<\varepsilon$ siempre que $0<|x-a|<\delta$.
\end{definicion}

\begin{center}
\scalebox{1}{\psset{unit=1.4,algebraic}
\begin{pspicture*}(-1,-0.5)(2,2)
\psaxes[ticks=none,labels=none]{<->}(0,0)(-0.5,-0.5)(2,2)
\psplot[linecolor=blue]{0.4}{1.6}{x^1.5}
\footnotesize
\psxTick[ticksize=-3pt 0,labelsep=3pt](1){a}
\psyTick[ticksize=-3pt 0,labelsep=3pt](1){l}
\psline[linecolor=red,linestyle=dotted](1,0)(1,1)(0,1)
\uncover<2->{
\psyTick[ticksize=-3pt 0,labelsep=3pt](1.6565){l+\varepsilon}
\psyTick[ticksize=-3pt 0,labelsep=3pt](0.4648){l-\varepsilon}
}
\uncover<3->{
\psline[linecolor=red,linestyle=dashed](1.4,0)(1.4,1.6565)(0,1.6565)
\psline[linecolor=red,linestyle=dashed](0.6,0)(0.6,0.4648)(0,0.4648)
}
\uncover<4->{
\psxTick[ticksize=-3pt 0,labelsep=3pt](1.4){a+\delta}
\psxTick[ticksize=-3pt 0,labelsep=3pt](0.6){a-\delta}
}
\end{pspicture*}}
\end{center}
\end{frame}


%---------------------------------------------------------------------slide----
\begin{frame}
\frametitle{Definición de límite en el infinito}
\begin{definicion}[Límite de una función en el infinito]
Se dice que el límite de la función $f$ cuando $x\rightarrow +\infty$ es $l$, y se escribe
\[\lim_{x\rightarrow +\infty} f(x) =l\]
si para cualquier valor $\varepsilon>0$ existe un número $\delta>0$ tal que, $|f(x)-l|<\varepsilon$ siempre que $x>\delta$.

Se dice que el límite de la función $f$ cuando $x\rightarrow +\infty$ es $l$, y se escribe
\[\lim_{x\rightarrow +\infty} f(x) =l\]
si para cualquier valor $\varepsilon>0$ existe un número $\delta<0$ tal que, $|f(x)-l|<\varepsilon$ siempre que $x<\delta$.
\end{definicion}
\end{frame}



\subsection{Álgebra de límites}
%---------------------------------------------------------------------slide----
\begin{frame}
\frametitle{Álgebra de límites}
Dadas dos funciones $f(x)$ y $g(x)$, tales que existe $\lim_{x\rightarrow a}f(x)$ y $\lim_{x\rightarrow a}g(x)$, entonces se cumple que
\begin{enumerate}
\item $\displaystyle \lim_{x\rightarrow a}c f(x)=c\lim_{x\rightarrow a}f(x)$, siendo $c$ constante. 
\item $\displaystyle \lim_{x\rightarrow a}(f(x)\pm g(x))=\lim_{x\rightarrow a}f(x)\pm \lim_{x\rightarrow a}g(x)$.
\item $\displaystyle \lim_{x\rightarrow a}(f(x)\cdot g(x))=\lim_{x\rightarrow a}f(x)\cdot \lim_{x\rightarrow a}g(x)$.
\item $\displaystyle \lim_{x\rightarrow a}\frac{f(x)}{g(x)}=\frac{\displaystyle \lim_{x\rightarrow
a}f(x)}{\displaystyle \lim_{x\rightarrow a}g(x)}$ si $\displaystyle \lim_{x\rightarrow a}g(x)\neq 0$.
\end{enumerate}
\end{frame}

%---------------------------------------------------------------------slide----
\begin{frame}
\frametitle{Límites de las funciones elementales}
\begin{itemize}
\item \structure{\textbf{Funciones polinómicas}}. Si $f$ es un polinomio, entonces existe el límite de $f$ en cualquier
punto $a\in \mathbb{R}$ y $\lim_{x\rightarrow a}f(x)=f(a)$. 
\item \structure{\textbf{Funciones racionales}}. Si $f(x)=\dfrac{p(x)}{q(x)}$ con $p(x)$ y $q(x)$ dos polinomios,
entonces existe el límite de $f$ en cualquier punto $a\in \mathbb{R}$ que no sea una raíz de $q(x)$, y $\lim_{x\rightarrow a}f(x)=f(a)$. Si $a$ es una raíz de $q(x)$ entonces el límite puede existir o no.
\item \structure{\textbf{Funciones potenciales}}. Si $f(x)=x^r$ con $r\in \mathbb{R}$, entonces existe el límite de $f$
en cualquier punto $a$ tal que exista un intervalo $(a-\delta,a+\delta)\subset \textrm{Dom}(f)$ para algún $\delta >0$, y en ese caso, $\lim_{x\rightarrow a}f(x)=f(a)$.
\item \structure{\textbf{Funciones exponenciales}}. Si $f(x)=c^x$ con $c\in \mathbb{R}$ entonces existe el límite de $f$
en cualquier punto $a\in \mathbb{R}$ y $\lim_{x\rightarrow a}f(x)=f(a)$.
\item \structure{\textbf{Funciones logarítmicas}}. Si $f(x)=\log_cx$ con $c\in \mathbb{R}$, entonces existe el límite de
$f$ en cualquier  punto $a\in \mathbb{R}^+$ y $\lim_{x\rightarrow a}f(x)=f(a)$.
\item \structure{\textbf{Funciones trigonométricas}}. Si $f(x)$ es una función trigonométrica, entonces existe el límite
de $f$ en cualquier punto $a\in \textrm{Dom}(f)$ y $\lim_{x\rightarrow a}f(x)=f(a)$.
\end{itemize}
\end{frame}



\subsection{Indeterminaciones y su resolución}
%---------------------------------------------------------------------slide----
\begin{frame}
\frametitle{Indeterminaciones}
Al calcular límites pueden aparecer las siguientes indeterminaciones:
\begin{itemize}
\item \structure{\textbf{Tipo cociente}}. Si $\lim_{x\rightarrow a} f(x)=0$ y $\lim_{x\rightarrow a} g(x)=0$, entonces
$\dfrac{f(x)}{g(x)}$ presenta una indeterminación del tipo \alert{$\dfrac{0}{0}$} cuando $x\rightarrow a$.

Si $\lim_{x\rightarrow a} f(x)=\pm\infty$ y $\lim_{x\rightarrow a} g(x)=\pm\infty$, entonces $\dfrac{f(x)}{g(x)}$ presenta una indeterminación del tipo \alert{$\pm\dfrac{\infty}{\infty}$} cuando $x\rightarrow a$.

\item \structure{\textbf{Tipo producto}}. Si $\lim_{x\rightarrow a} f(x)=0$ y $\lim_{x\rightarrow a} g(x)=\pm\infty$,
 entonces $f(x)\cdot g(x)$ presenta una indeterminación del tipo \alert{$0\cdot \pm\infty$} cuando $x\rightarrow a$.
\item \structure{\textbf{Tipo potencia}}. Si $\lim_{x\rightarrow a} f(x)=1$ y $\lim_{x\rightarrow a} g(x)=\infty$,
entonces $f(x)^{g(x)}$ presenta una indeterminación del tipo \alert{$1^\infty$} cuando $x\rightarrow a$.

Si $\lim_{x\rightarrow a} f(x)=0$ y $\lim_{x\rightarrow a} g(x)=0$, entonces $f(x)^{g(x)}$ presenta una indeterminación del tipo \alert{$0^0$} cuando $x\rightarrow a$.

Si $\lim_{x\rightarrow a} f(x)=\infty$ y $\lim_{x\rightarrow a} g(x)=0$, entonces $f(x)^{g(x)}$ presenta una indeterminación del tipo \alert{$\infty^0$} cuando $x\rightarrow a$.

\item \structure{\textbf{Tipo diferencia}}. Si $\lim_{x\rightarrow a} f(x)=\infty$ y $\lim_{x\rightarrow a}
 g(x)=\infty$, entonces $f(x)-g(x)$ presenta una indeterminación del tipo \alert{$\infty-\infty$} cuando $x\rightarrow a$.
\end{itemize}
\end{frame}


%---------------------------------------------------------------------slide----
\begin{frame}
\frametitle{Resolución de una indeterminación de tipo cociente}
Existen diferentes técnicas para resolver una indeterminación del tipo $\dfrac{0}{0}$ o $\dfrac{\infty}{\infty}$:
\begin{itemize}
\item Factorización de polinomios en funciones racionales.
\item División por el términos de mayor orden en funciones racionales.
\item Infinitésimos equivalentes.
\item Regla de L'Hôpital.
\end{itemize}
\end{frame}


%---------------------------------------------------------------------slide----
\begin{frame}
\frametitle{Resolución de una indeterminación de tipo cociente}
\framesubtitle{Factorización de polinomios en funciones racionales}
Si $f(x)=\dfrac{p(x)}{q(x)}$ es una función racional que presenta una indeterminación de tipo cociente cuando $x\rightarrow a$, y $a$ es una raíz de $p(x)$ y $q(x)$, se puede resolver la indeterminación factorizando los polinomios y simplificando.

\structure{\textbf{Ejemplo}} La función $f(x)=\dfrac{x^3-3x+2}{x^4-4x+3}\rightarrow \dfrac{0}{0}$ cuando $x\rightarrow 1$.

Para resolver la indeterminación factorizamos los polinomios
\begin{align*}
  x^3-3x+2 &= (x+2)(x-1)^2,\\
  x^4-4x+3 &= (x^2+2x+3)(x-1)^2.
\end{align*}
Como el factor $(x-1)^2$ es común, podemos simplificar la función en el cálculo del límite:
\begin{align*}
\lim_{x\rightarrow 1}\frac{x^3-3x+2}{x^4-4x+3} &=
\lim_{x\rightarrow 1}\frac{(x+2)(x-1)^2}{(x^2+2x+3)(x-1)^2} =
\lim_{x\rightarrow 1}\frac{(x+2)}{(x^2+2x+3)} =\frac{3}{6}=0.5.
\end{align*}\footnote{Se pude simplificar porque aunque $x\rightarrow 1$, $x\neq 1$ y por tanto el denominador no se anula.}
\end{frame}


%---------------------------------------------------------------------slide----
\begin{frame}
\frametitle{Resolución de una indeterminación de tipo cociente}
\framesubtitle{División por el término de mayor orden en funciones racionales}
Si $f(x)=\dfrac{p(x)}{q(x)}$ es una función racional que presenta una indeterminación de tipo cociente cuando $x\rightarrow \pm\infty$, entonces se puede resolver dividendo $p(x)$ y $q(x)$ por el término de mayor grado de ambos polinomios.

\structure{\textbf{Ejemplo}} La función $f(x)=\dfrac{x^3-3x+2}{x^4-4x+3}\rightarrow \dfrac{\infty}{\infty}$ cuando $x\rightarrow \infty$.

Para resolver la indeterminación dividimos numerador y denominador por $x^4$ que es el término de mayor grado:
\begin{align*}
\lim_{x\rightarrow \infty}\frac{x^3-3x+2}{x^4-4x+3} &=
\lim_{x\rightarrow \infty}\frac{\frac{x^3-3x+2}{x^4}}{\frac{x^4-4x+3}{x^4}} =
\lim_{x\rightarrow \infty}\frac{\frac{1}{x}-\frac{3}{x^3}+\frac{2}{x^4}}{1-\frac{4}{x^3}+\frac{3}{x^4}} =\frac{0}{1}=0
\end{align*}
En general, si $f(x)=\dfrac{a_0+a_1x+\cdots a_nx^n}{b_0+b_1x+\cdots b_mx^m}$, entonces:
\begin{itemize}
\item[--] Si $n>m$ entonces $\lim_{x\rightarrow \pm \infty}f(x)=\pm\infty$.
\item[--] Si $n<m$ entonces $\lim_{x\rightarrow \pm \infty}f(x)=0$.
\item[--] Si $n=m$ entonces $\lim_{x\rightarrow \pm \infty}f(x)=\dfrac{a_n}{b_m}$.
\end{itemize}
\end{frame}


%---------------------------------------------------------------------slide----
\begin{frame}
\frametitle{Infinitésimos equivalentes}
\begin{definicion}[Infinitésimos equivalentes]
Si $f(x)\rightarrow 0$ y $g(x)\rightarrow 0$ cuando $x\rightarrow a$, entonces se dice que $f$ y $g$ son \emph{infinitésimos equivalentes} cuando $x\rightarrow a$ si se cumple
\[
\lim_{x\rightarrow a}\frac{f(x)}{g(x)}=1
\]
En tal caso se escribe $f(x)\approx g(x)$ cuando $x\rightarrow a$.
\end{definicion}

Si $f(x)\approx g(x)$ cuando $x\rightarrow a$ entonces $f(x)$ y $g(x)$ son magnitudes equivalentes cuando $x\rightarrow a$.

Infinitésimos equivalentes cuando $x\rightarrow 0$:
\[
\begin{array}{c}
\sen x \approx x \approx \tg x\\
1-\cos x \approx \dfrac{x^2}{2}\\
\arctg x \approx x\\
e^x-1 \approx x\\
\log(1+x) \approx x\\
\end{array}
\]
\end{frame}


%---------------------------------------------------------------------slide----
\begin{frame}
\frametitle{Resolución de una indeterminación de tipo cociente}
\framesubtitle{Infinitésimos equivalentes}
A veces se puede resolver una indeterminación cuando $x\rightarrow a$ sustituyendo cualquier subexpresión de la función por un infinitésimo equivalente cuando $x\rightarrow a$.

\structure{\textbf{Ejemplo}} La función $f(x)=\dfrac{\sen x(1- \cos x)}{x^3}\rightarrow \dfrac{0}{0}$ cuando $x\rightarrow 0$.

Como $\sen x \approx x$ y $1-\cos x\approx \dfrac{x^2}{2}$ cuando $x\rightarrow 0$, para resolver la indeterminación sustituimos $\sen x$ por $x$ y $1-\cos x$ por $\dfrac{x^2}{2}$:
\begin{align*}
\lim_{x\rightarrow 0}\frac{\sen x(1- \cos x)}{x^3}&=
\lim_{x\rightarrow 0}\dfrac{x\frac{x^2}{2}}{x^3} =
\lim_{x\rightarrow 0}\dfrac{\frac{x^3}{2}}{x^3} = \lim_{x\rightarrow 0}\dfrac{1}{2} =0.5.
\end{align*}
\end{frame}


%---------------------------------------------------------------------slide----
\begin{frame}
\frametitle{Resolución de una indeterminación de tipo cociente}
%\framesubtitle{Regla de L'Hôpital}
\begin{teorema}[Regla de L'Hôpital]
Si $\dfrac{f(x)}{g(x)}\rightarrow \dfrac{0}{0}$ o $\dfrac{\infty}{\infty}$ cuando $x\rightarrow a$, entonces si existe el límite de $\dfrac{f'(x)}{g'(x)}$ cuando $x\rightarrow a$ se cumple
\[
\lim_{x\rightarrow a}\frac{f(x)}{g(x)}=\lim_{x\rightarrow a}\frac{f'(x)}{g'(x)}.
\]
\end{teorema}

\alert{¡Ojo!} Para que exista $\lim_{x\rightarrow a}\dfrac{f'(x)}{g'(x)}$ es necesario que que $f$ y $g$ sean derivables en un entorno de $a$.

\structure{\textbf{Ejemplo}} Sea $f(x)=\dfrac{\log(x^2-1)}{x+2}\rightarrow \dfrac{\infty}{\infty}$ cuando $x\rightarrow \infty$.

Para resolver la indeterminación aplicamos la regla de L'Hôpital:
\begin{align*}
\lim_{x\rightarrow \infty}\frac{\log(x^2-1)}{x+2} &= \lim_{x\rightarrow
\infty}\frac{\left(\log(x^2-1)\right)'}{\left(x+2\right)'}= \lim_{x\rightarrow \infty}\frac{\frac{2x}{x^2-1}}{1}=\\
&=\lim_{x\rightarrow \infty}\frac{2x}{x^2-1}= \lim_{x\rightarrow \infty}\frac{\left(2x\right)'}{\left(x^2-1\right)'}=
\lim_{x\rightarrow \infty}\frac{2}{2x}=0.
\end{align*}
\end{frame}


%---------------------------------------------------------------------slide----
\begin{frame}
\frametitle{Resolución de una indeterminación de tipo producto}
Si $f(x)\rightarrow 0$ y $g(x)\rightarrow \pm\infty$ cuando $x\rightarrow a$, entonces la indeterminación $f(x)\cdot g(x)\rightarrow 0\cdot \pm\infty$ puede convertirse en una de tipo cociente mediante la transformación:
\[
f(x)\cdot g(x) = \frac{f(x)}{1/g(x)}\rightarrow \frac{0}{0}.
\]
\structure{\textbf{Ejemplo}} Sea $f(x)=x^2e^{1/x^2}\rightarrow 0\cdot\infty$ cuando $x\rightarrow 0$.
\begin{align*}
\lim_{x\rightarrow 0}x^2e^{1/x^2} &= \lim_{x\rightarrow 0}\frac{e^{1/x^2}}{1/x^2}\rightarrow \frac{\infty}{\infty}
\end{align*}
Aplicando ahora la regla de L´Hôpital tenemos:
\begin{align*}
\lim_{x\rightarrow 0}\frac{e^{1/x^2}}{1/x^2} &= \lim_{x\rightarrow
0}\frac{\left(e^{1/x^2}\right)'}{\left(1/x^2\right)'} = \lim_{x\rightarrow
0}\frac{e^{1/x^2}\frac{-2}{x^3}}{\frac{-2}{x^3}} = \lim_{x\rightarrow 0}e^{1/x^2}=\infty.
\end{align*}
\end{frame}


%---------------------------------------------------------------------slide----
\begin{frame}
\frametitle{Resolución de una indeterminación de tipo potencia}
Si $f(x)^{g(x)}$ presenta una indeterminación de tipo potencia cuando $x\rightarrow a$, entonces la indeterminación puede convertirse en una de tipo producto mediante la transformación:
\[
\exp\left(\log f(x)^{g(x)}\right) = \exp\left(g(x)\cdot \log f(x)\right).
\]
\structure{\textbf{Ejemplo}} Sea $f(x)=\left(1+\dfrac{1}{x}\right)^{x} \rightarrow 1^\infty$ cuando $x\rightarrow 0$.
\begin{align*}
\lim_{x\rightarrow 0}\left(1+\dfrac{1}{x}\right)^{x} &= \lim_{x\rightarrow
0}\exp\left(\log\left(1+\frac{1}{x}\right)^{x}\right) = \exp\left(\lim_{x\rightarrow
0}x\log\left(1+\frac{1}{x}\right)\right) =\\
&= \exp\left(\lim_{x\rightarrow 0}\frac{\log\left(1+\frac{1}{x}\right)}{1/x}\right)
\end{align*}
Aplicando ahora la regla de L´Hôpital tenemos:
\begin{align*}
\exp\left(\lim_{x\rightarrow 0}\frac{\left(\log\left(1+\frac{1}{x}\right)\right)'}{\left(1/x\right)'}\right) &=
\exp\left(\lim_{x\rightarrow 0}\frac{\frac{1}{1+1/x}\frac{-1}{x^2}}{\frac{-1}{x^2}}\right) =
\exp\left(\lim_{x\rightarrow 0}\frac{1}{1+\frac{1}{x}}\right)=\exp(1)=e.
\end{align*}
\end{frame}


%---------------------------------------------------------------------slide----
\begin{frame}
\frametitle{Resolución de una indeterminación de tipo diferencia}
Si $f(x)\rightarrow \infty$ y $g(x)\rightarrow \infty$ cuando $x\rightarrow a$, entonces la indeterminación $f(x)-g(x)$ puede convertirse en una de tipo cociente mediante la transformación:
\[
f(x)-g(x)=\frac{\frac{1}{g(x)}-\frac{1}{f(x)}}{\frac{1}{f(x)g(x)}}\rightarrow \frac{0}{0}.
\]
\structure{\textbf{Ejemplo}} Sea $f(x)=\dfrac{1}{\sen x}-\dfrac{1}{x} \rightarrow \infty-\infty$ cuando $x\rightarrow 0$.
\begin{align*}
\lim_{x\rightarrow 0} \frac{1}{\sen x}-\frac{1}{x} &=
\lim_{x\rightarrow 0} \frac{x-\sen x}{x\sen x} \rightarrow \frac{0}{0}
\end{align*}
Aplicando ahora la regla de L´Hôpital tenemos:
\begin{align*}
\lim_{x\rightarrow 0} \frac{x-\sen x}{x\sen x} & =
\lim_{x\rightarrow 0} \frac{\left(x-\sen x\right)'}{\left(x\sen x\right)'}=
\lim_{x\rightarrow 0} \frac{1-\cos x}{\sen x +x\cos x} = \\
&= \lim_{x\rightarrow 0} \frac{\left(1-\cos x\right)'}{\left(\sen x +x\cos x\right)'} =
\lim_{x\rightarrow 0} \frac{\sen x}{\cos x +\cos x-x\sen x}=\frac{0}{2}=0.
\end{align*}
\end{frame}



\subsection{Asíntotas de una función}
%---------------------------------------------------------------------slide----
\begin{frame}
\frametitle{Asíntota de una función}
Una asíntota de una función es una recta a la que tiende la función en el infinito, es decir, que la distancia entre la recta y la función es cada vez menor.

Existen tres tipos de asíntotas:
\begin{itemize}
\item \structure{\textbf{Asíntota vertical}}: $x=a$,
\item \structure{\textbf{Asíntota horizontal}}: $y=a$,
\item \structure{\textbf{Asíntota oblicua}}: $y=a+bx$.
\end{itemize}
\end{frame}


%---------------------------------------------------------------------slide----
\begin{frame}
\frametitle{Asíntotas verticales}
\begin{definicion}[Asíntota vertical]
Se dice que una recta $x=a$ es una \emph{asíntota vertical} de una función $f$ si se cumple
\[ \lim_{x\rightarrow a^-}f(x)=\pm \infty \quad \textrm{o} \quad \lim_{x\rightarrow a^-}f(x)=\pm \infty \]
\end{definicion}

Las asíntotas verticales deben buscarse en los puntos donde no está definida la función, pero si lo está en las proximidades.

\structure{\textbf{Ejemplo}} La recta $x=2$ es una asíntota vertical de $f(x)=\dfrac{x+1}{x-2}$ ya que
\begin{columns}
\begin{column}{0.3\textwidth}
\begin{align*}
\lim_{x\rightarrow 2^-}\frac{x+1}{x-2}&=-\infty, \textrm{ y}\\
\lim_{x\rightarrow 2^+}\frac{x+1}{x-2}&=\infty
\end{align*}
\end{column}
\begin{column}{0.4\textwidth}
\begin{center}
\scalebox{1}{\psset{unit=0.45,algebraic}
\begin{pspicture*}(-3,-3)(7,5)
\psaxes[labelFontSize=\scriptstyle,ticksize=-3pt 0,labelsep=2pt]{<->}(0,0)(-3,-3)(7,5)
\psplot[linecolor=blue]{-3}{1.9999}{(x+1)/(x-2)}
\psplot[linecolor=blue]{2.0001}{7}{(x+1)/(x-2)}
\psline[linecolor=red](2,-3)(2,5)
\footnotesize
\rput[l](3.5,4){$f(x)=\dfrac{x+1}{x-2}$}
\rput[l](2.1,-2){$x=2$}
\end{pspicture*}}
\end{center}
\end{column}
\end{columns}
\end{frame}


%---------------------------------------------------------------------slide----
\begin{frame}
\frametitle{Asíntotas horizontales}
\begin{definicion}[Asíntota horizontal]
Se dice que una recta $y=a$ es una \emph{asíntota horizontal} de una función $f$ si se cumple
\[ \lim_{x\rightarrow +\infty}f(x)=a \quad \textrm{o} \quad \lim_{x\rightarrow \infty}f(x)=a \]
\end{definicion}

\structure{\textbf{Ejemplo}} La recta $y=1$ es una asíntota horizontal de $f(x)=\dfrac{x+1}{x-2}$ ya que
\begin{columns}
\begin{column}{0.5\textwidth}
\begin{align*}
\lim_{x\rightarrow -\infty}\frac{x+1}{x-2}&= \lim_{x\rightarrow -\infty}1+\frac{3}{x-2} = 1, \textrm{ y}\\
\lim_{x\rightarrow +\infty}\frac{x+1}{x-2}&= \lim_{x\rightarrow +\infty}1+\frac{3}{x-2} = 1
\end{align*}
\end{column}
\begin{column}{0.5\textwidth}
\begin{center}
\scalebox{1}{\psset{unit=0.5,algebraic}
\begin{pspicture*}(-3,-3)(7,5)
\psaxes[labelFontSize=\scriptstyle,ticksize=-3pt 0,labelsep=2pt]{<->}(0,0)(-3,-3)(7,5)
\psplot[linecolor=blue]{-3}{1.9999}{(x+1)/(x-2)}
\psplot[linecolor=blue]{2.0001}{7}{(x+1)/(x-2)}
\psline[linecolor=red](-3,1)(7,1)
\footnotesize
\rput[l](3.5,4){$f(x)=\dfrac{x+1}{x-2}$}
\rput[b](-2,1.1){$y=1$}
\end{pspicture*}}
\end{center}
\end{column}
\end{columns}
\end{frame}


%---------------------------------------------------------------------slide----
\begin{frame}
\frametitle{Asíntotas oblicuas}
\begin{definicion}[Asíntota oblicua]
Se dice que una recta $y=a+bx$ es una \emph{asíntota oblicua} de una función $f$ si se cumple
\[
\lim_{x\rightarrow \pm\infty}\frac{f(x)}{x}=b \quad \textrm{y} \quad \lim_{x\rightarrow \pm\infty}f(x)-bx=a.
\]
%o bien,
%\[
%\lim_{x\rightarrow -\infty}\frac{f(x)}{x}=b \quad \textrm{y} \quad \lim_{x\rightarrow -\infty}f(x)-bx=a
%\]
\end{definicion}

\structure{\textbf{Ejemplo}} La recta $y=x+1$ es una asíntota oblicua de $f(x)=\dfrac{x^2}{x-1}$
\begin{columns}
\begin{column}{0.5\textwidth}
\begin{align*}
\lim_{x\rightarrow \pm\infty}\frac{\frac{x^2}{x-1}}{x}&=
\lim_{x\rightarrow \pm\infty}\frac{x^2}{x^2-x} = 1, \textrm{ y}\\
\lim_{x\rightarrow \pm\infty}\frac{x^2}{x-1}-x &=
\lim_{x\rightarrow \pm\infty}1+\frac{x}{x-1} = 1
\end{align*}
\end{column}
\begin{column}{0.5\textwidth}
\begin{center}
\scalebox{1}{\psset{unit=0.5,algebraic}
\begin{pspicture*}(-4,-3)(6,6)
\psaxes[labelFontSize=\scriptstyle,ticksize=-3pt 0,labelsep=2pt]{<->}(0,0)(-4,-3)(6,6)
\psplot[linecolor=blue]{-4}{0.9999}{x^2/(x-1)}
\psplot[linecolor=blue]{1.0001}{6}{x^2/(x-1)}
\psplot[linecolor=red]{-4}{6}{x+1}
\footnotesize
\rput[l](1.5,-2){$f(x)=\dfrac{x^2}{x-1}$}
\rput[l](3,3){$y=x+1$}
\end{pspicture*}}
\end{center}
\end{column}
\end{columns}
\end{frame}



\subsection{Continuidad}
%---------------------------------------------------------------------slide----
\begin{frame}
\frametitle{Continuidad}
\begin{definicion}[Función continua en un punto]
Se dice que una función $f$ es \emph{continua} en el punto $a$ si
\[ \lim_{x\rightarrow a}f(x)=f(a).\]
\end{definicion}

 De esta definición se deducen tres condiciones necesarias para la continuidad:
\begin{enumerate}
 $f(a)\in \textrm{Dom}(f)$.
 Existe $\displaystyle \lim_{x\rightarrow a}f(x)$.
 $\displaystyle \lim_{x\rightarrow a}f(x)=f(a)$.
\end{enumerate}

Si se rompe alguna de estas condiciones, se dice que la función presenta una discontinuidad en $a$.

\begin{definicion}[Función continua en un intervalo]
Se dice que una función $f$ es \emph{continua} en un intervalo si lo es en cada uno de los puntos del intervalo.
\end{definicion}

La gráfica de una función continua en un intervalo puede dibujarse sin levantar el lápiz.
\end{frame}



\subsection{Tipos de discontinuidades}
%---------------------------------------------------------------------slide----
\begin{frame}
\frametitle{Tipos de discontinuidades}
Dependiendo de la condición de continuidad que se rompa, existen distintos tipos de  discontinuidades:
\begin{itemize}
\item Discontinuidad evitable.
\item Discontinuidad de 1ª especie de salto finito.
\item Discontinuidad de 1ª especie de salto infinito.
\item Discontinuidad de 2ª especie.
\end{itemize}
\end{frame}


%---------------------------------------------------------------------slide----
\begin{frame}
\frametitle{Discontinuidad evitable}
\begin{definicion}[Discontinuidad evitable]
Se dice que una función $f$ tiene una \emph{discontinuidad evitable} en el punto $a$ si
existe el límite de $f(x)$ cuando $x\rightarrow a$ pero $\displaystyle \lim_{x\rightarrow a}f(x)\neq f(a)$.
\end{definicion}

\structure{\textbf{Ejemplo}} La función $f(x)=\dfrac{x^2-1}{x-1}$ tiene una discontinuidad evitable en $x=1$ ya que
\begin{columns}
\begin{column}{0.4\textwidth}
La función no está definida en $x=1$ pero
\begin{align*}
\lim_{x\rightarrow 2}\frac{x^2-1}{x-1}&= \lim_{x\rightarrow 2}x+1=2.
\end{align*}
\end{column}
\begin{column}{0.5\textwidth}
\begin{center}
\scalebox{1}{\psset{unit=0.6,algebraic}
\begin{pspicture*}(-2,-1.5)(4,5.5)
\psaxes[labelFontSize=\scriptstyle,ticksize=-3pt 0,labelsep=2pt]{<->}(0,0)(-2,-1.5)(4,5.5)
\psplot[linecolor=blue]{-4}{0.98}{(x^2-1)/(x-1)}
\psplot[linecolor=blue]{1.02}{6}{(x^2-1)/(x-1)}
\footnotesize
\rput[l](1.4,2){$f(x)=\dfrac{x^2-1}{x-1}$}
\end{pspicture*}}
\end{center}
\end{column}
\end{columns}


\end{frame}


%---------------------------------------------------------------------slide----
\begin{frame}
\frametitle{Discontinuidad de 1ª especie de salto finito}
\begin{definicion}[Discontinuidad de 1ª especie de salto finito]
Se dice que una función $f$ tiene una \emph{discontinuidad de 1ª especie de salto finito} en el punto $a$ si existen los límites laterales de $f(x)$ cuando $x\rightarrow a$ pero
\[\lim_{x\rightarrow a^-}f(x)\neq \lim_{x\rightarrow a^+}f(x).\]
A la diferencia entre ambos límite se le lama \emph{salto} de la discontinuidad.
\end{definicion}

\structure{\textbf{Ejemplo}} La función $f(x)=\dfrac{|x|}{x}$ tiene una discontinuidad de 1ª especie de salto finito en $x=0$ ya que
\begin{columns}
\begin{column}{0.4\textwidth}
\begin{align*}
\lim_{x\rightarrow 0^-}\frac{|x|}{x}&= -1\\
\lim_{x\rightarrow 0^+}\frac{|x|}{x}&= 1
\end{align*}
Salto $= 1-(-1)=2$.
\end{column}
\begin{column}{0.5\textwidth}
\begin{center}
\scalebox{1}{\psset{unit=1,algebraic}
\begin{pspicture*}(-2,-1.5)(2,1.5)
\psaxes[labelFontSize=\scriptstyle,ticksize=-3pt 0,labelsep=2pt]{<->}(0,0)(-2,-1.5)(2,1.5)
\psplot[linecolor=blue]{-2}{0}{-1}
\psplot[linecolor=blue]{0}{2}{1}
\footnotesize
\rput[l](0.5,0.5){$f(x)=\dfrac{|x|}{x}$}
\uncover<2->{
\psline[arrows=|-|,linecolor=red](0,-1)(0,1)
\rput[r](-0.1,0.5){Salto}
}
\end{pspicture*}}
\end{center}
\end{column}
\end{columns}
\end{frame}


%---------------------------------------------------------------------slide----
\begin{frame}
\frametitle{Discontinuidad de 1ª especie de salto infinito}
\begin{definicion}[Discontinuidad de 1ª especie de salto infinito]
Se dice que una función $f$ tiene una \emph{discontinuidad de 1ª especie de salto infinito} en el punto $a$ si
\[\lim_{x\rightarrow a^-}f(x)=\pm\infty \quad \textrm{o} \quad \lim_{x\rightarrow a^+}f(x)=\pm\infty.\]
\end{definicion}

Si $f$ tienen una discontinuidad de 1ª especie de salto infinito en un punto $a$, entonces $f$ tienen una asíntota vertical $x=a$.

\structure{\textbf{Ejemplo}} La función $f(x)=e^{1/x}$ tiene una discontinuidad de 1ª especie de salto infinito en $x=0$ ya que
\begin{columns}
\begin{column}{0.3\textwidth}
\begin{align*}
\lim_{x\rightarrow 0^-}e^{1/x}&= 0\\
\lim_{x\rightarrow 0^+}e^{1/x}&= \infty
\end{align*}
\end{column}
\begin{column}{0.5\textwidth}
\begin{center}
\scalebox{1}{\psset{unit=0.5,algebraic}
\begin{pspicture*}(-3,-1)(3,5)
\psaxes[labelFontSize=\scriptstyle,ticksize=-3pt 0,labelsep=2pt]{<->}(0,0)(-3,-1)(3,5)
\psplot[linecolor=blue]{-3}{-0.0001}{2.7182^(1/x)}
\psplot[linecolor=blue]{0.1}{3}{2.7182^(1/x)}
\footnotesize
\rput[l](0.5,0.5){$f(x)=e^{1/x}$}
\end{pspicture*}}
\end{center}
\end{column}
\end{columns}
\end{frame}


%---------------------------------------------------------------------slide----
\begin{frame}
\frametitle{Discontinuidad de 2ª especie}
\begin{definicion}[Discontinuidad de 2ª especie]
Se dice que una función $f$ tiene una \emph{discontinuidad de 2ª especie} en el punto $a$ si
no existe alguno de los límites laterales y tampoco se trata de una discontinuidad de 1ª especie de salto infinito.
\end{definicion}

Normalmente la discontinuidades de 2ª especie se dan en puntos donde la función no definida en sus proximidades.

\structure{\textbf{Ejemplo}} La función $f(x)=\dfrac{1}{\sqrt{x^2-1}}$ tiene una discontinuidad de 2ª especie en $x=0$ ya que
\begin{columns}
\begin{column}{0.3\textwidth}
\begin{align*}
& \lim_{x\rightarrow 1^-}\frac{1}{\sqrt{x^2-1}} \textrm{ no existe}  \\
& \lim_{x\rightarrow 1^+}\frac{1}{\sqrt{x^2-1}}=\infty
\end{align*}
\end{column}
\begin{column}{0.5\textwidth}
\begin{center}
\scalebox{1}{\psset{unit=0.55,algebraic}
\begin{pspicture*}(-3,-2)(3,4)
\psaxes[labelFontSize=\scriptstyle,ticksize=-3pt 0,labelsep=2pt]{<->}(0,0)(-3,-0.5)(3,4)
\psplot[linecolor=blue]{-3}{-1.00001}{1/sqrt(x^2-1)}
\psplot[linecolor=blue]{1.00001}{3}{1/sqrt(x^2-1)}
\footnotesize
\rput[t](0,-0.6){$f(x)=\dfrac{1}{\sqrt{x^2-1}}$}
\end{pspicture*}}
\end{center}
\end{column}
\end{columns}
\end{frame} 
